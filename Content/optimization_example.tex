
Before introducing an example of equivalent sequences of instructions in
EVM bytecode we need to give a simple introduction on programming in the
EVM to be able to follow the example. EVM instructions modify the state
of the EVM in two main ways:
\begin{itemize}
    \item Manipulating the values in the stack. \\
    In our simulation of the execution of a program this will be represented
    by showing the state of the stack before and after the execution of the 
    instruction. For example, consider the 
    \textcolor{red}{\texttt{ADD}} instruction which pushes to the top of the 
    stack the addition of its first two elements.
    \begin{Verbatim}[commandchars=\\\{\}]
    [x0,x1] 
     ↓ \textcolor{red}{ADD}
    [x0+x1]
    \end{Verbatim}

    \item Producing an effect. \\
    An effect is any change which is not reflected in the stack but can be 
    observed by an external process. The only instruction we are interested in
    which produces an effect is the \textcolor{red}{\texttt{MSTORE}} 
    instruction, which stores at the address referenced by the top element in 
    the stack the value at the position after it. Notice that the instruction
    besides producing an effect also manipulates the values in the stack. We
    represent effects, if present, by putting them in parenthesis next to the 
    state of the stack.
    \begin{Verbatim}[commandchars=\\\{\}]
    [x0,x1]
     ↓ \textcolor{red}{MSTORE}
    []   \textcolor{teal}{(MEM[x0] = x1)}
    \end{Verbatim}
\end{itemize}

Besides \textcolor{red}{\texttt{ADD}} and \textcolor{red}{\texttt{MSTORE}} we also
describe the following instructions.
\begin{itemize}
    \item \textcolor{red}{\texttt{SWAP1}} exchanges the top element in the stack with the
    second element bellow it.
    \begin{Verbatim}[commandchars=\\\{\}]
    [x0,x1]
     ↓ \textcolor{red}{SWAP1}
    [x1,x0]
    \end{Verbatim}

    \item \textcolor{red}{\texttt{SWAP2} } exchanges the top element in the stack with the
    second element bellow it.
    \begin{Verbatim}[commandchars=\\\{\}]
    [x0,x1,x2]
     ↓ \textcolor{red}{SWAP2}
    [x2,x1,x0]
    \end{Verbatim}

    \item \textcolor{red}{\texttt{SWAP3}} exchanges the top element in the stack with the
    third element bellow it.
    \begin{Verbatim}[commandchars=\\\{\}]
    [x0,x1,x2,x3]
     ↓ \textcolor{red}{SWAP3}
    [x3,x1,x2,x0]
    \end{Verbatim}

    \item \textcolor{red}{\texttt{POP}} discards the top element of the stack.
    \begin{Verbatim}[commandchars=\\\{\}]
    [x0] 
     ↓ \textcolor{red}{POP}
    []
    \end{Verbatim}
\end{itemize}

With these we can finally move on to analyze the following example. Consider
these two sequences of EVM instructions.

\begin{listing}[!ht]
{\color{red}
\begin{verbatim}
SWAP3 SWAP1 SWAP2 MSTORE SWAP1 SWAP2 ADD SWAP1 MSTORE
\end{verbatim}
}
\caption{Original EVM program}
\label{lst:evm-original}
\end{listing}

\begin{listing}[!ht]
{\color{red}
\begin{verbatim}
MSTORE MSTORE POP 
\end{verbatim}
}
\caption{Optimized EVM program}
\label{lst:evm-optimized}
\end{listing}

Even though we refer to them as ``Original'' and ``Optimized'', at first glance these 
two programs don't appear to be equivalent. Let's consider the results of their
executions.

Consider the execution of the first program (\ref{lst:evm-original}).

\begin{Verbatim}[commandchars=\\\{\}]
[x0,x1,x2,x3,x4,x5]
 ↓ \textcolor{red}{SWAP3}
[x3,x1,x2,x0,x4,x5]
 ↓ \textcolor{red}{SWAP1}
[x1,x3,x2,x0,x4,x5]
 ↓ \textcolor{red}{SWAP2}
[x2,x3,x1,x0,x4,x5]
 ↓ \textcolor{red}{MSTORE}
[x1,x0,x4,x5] \textcolor{teal}{(MEM[x2] = x3)}
 ↓ \textcolor{red}{SWAP1}
[x0,x1,x4,x5] \textcolor{teal}{(MEM[x2] = x3)}
 ↓ \textcolor{red}{SWAP2}
[x4,x1,x0,x5] \textcolor{teal}{(MEM[x2] = x3)}
 ↓ \textcolor{red}{ADD}
[x4+x1,x0,x5] \textcolor{teal}{(MEM[x2] = x3)}
 ↓ \textcolor{red}{SWAP1}
[x0,x4+x1,x5] \textcolor{teal}{(MEM[x2] = x3)}
 ↓ \textcolor{red}{MSTORE}
[x5]          \textcolor{teal}{(MEM[x2] = x3; MEM[x0] = x4+x1)}
\end{Verbatim}

Finally, consider the execution of the second program (\ref{lst:evm-optimized})

\begin{Verbatim}[commandchars=\\\{\}]
[x0,x1,x2,x3,x4,x5]
 ↓ \textcolor{red}{MSTORE}
[x2,x3,x4,x5] \textcolor{teal}{(MEM[x0] = x1)}
 ↓ \textcolor{red}{MSTORE}
[x4,x5]       \textcolor{teal}{(MEM[x0] = x1; MEM[x2] = x3)}
 ↓ \textcolor{red}{POP}
[x5]          \textcolor{teal}{(MEM[x0] = x1; MEM[x2] = x3)}
\end{Verbatim}

While the end result of the stacks are equivalent for both executions, we can't ensure that the
effects they produce are equivalent. For starters, we cannot ensure that the memory accesses can
be commuted. After all, if \verb|x0| $=$ \verb|x2| $= 0$ then after the first program we'd have
{\textcolor{teal}{\texttt{MEM[0]~=~x4+x1}}} but after the second we'd have 
{\textcolor{teal}{\texttt{MEM[0]~=~x3}}}, which need not be the same if \verb|x3| $\ne$ \verb|x4+x1|.
Furthermore, we'd need that \verb|x4+x1| $=$ \verb|x1|, which is only possible if \verb|x4| $= 0$.

However, some of these conditions may be fulfilled if we consider the contextual information.
For example, consider the following contextual information, a list of added constraints to the
previous variables.

\begin{itemize}
    \item \verb|x4| $=$ \verb|x5|
    \item \verb|x5| $=$ 0
    \item \verb|x0| $\ge$ \verb|x2| $+ 128$
\end{itemize}

From the last constraint we can derive that the memory accesses to the offsets
referenced by \verb|x0| and \verb|x2| are disjoint, since the word size of the EVM is of 32 bytes and 
$32 < 128$, and therefore we can reorder both writes while preserving the semantics of the code.
From the first two constraints we can derive that \verb|x4| $= 0$ which ensures that \verb|x4| $+$ 
\verb|x1| is in fact equivalent to \verb|x1|.
