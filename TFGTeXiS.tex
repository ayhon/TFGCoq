% ----------------------------------------------------------------------
%
%                            TFMTesis.tex
%
%----------------------------------------------------------------------
%
% Este fichero contiene el "documento maestro" del documento. Lo único
% que hace es configurar el entorno LaTeX e incluir los ficheros .tex
% que contienen cada sección.
%
%----------------------------------------------------------------------
%
% Los ficheros necesarios para este documento son:
%
%       TeXiS/* : ficheros de la plantilla TeXiS.
%       Cascaras/* : ficheros con las partes del documento que no
%          son capítulos ni apéndices (portada, agradecimientos, etc.)
%       Capitulos/*.tex : capítulos de la tesis
%       Apendices/*.tex: apéndices de la tesis
%       constantes.tex: constantes LaTeX
%       config.tex : configuración de la "compilación" del documento
%       guionado.tex : palabras con guiones
%
% Para la bibliografía, además, se necesitan:
%
%       *.bib : ficheros con la información de las referencias
%
% ---------------------------------------------------------------------

\documentclass[12pt,a4paper,twoside]{book}

%
% Definimos  el   comando  \compilaCapitulo,  que   luego  se  utiliza
% (opcionalmente) en config.tex. Quedaría  mejor si también se definiera
% en  ese fichero,  pero por  el modo  en el  que funciona  eso  no es
% posible. Puedes consultar la documentación de ese fichero para tener
% más  información. Definimos también  \compilaApendice, que  tiene el
% mismo  cometido, pero  que se  utiliza para  compilar  únicamente un
% apéndice.
%
%
% Si  queremos   compilar  solo   una  parte  del   documento  podemos
% especificar mediante  \includeonly{...} qué ficheros  son los únicos
% que queremos  que se incluyan.  Esto  es útil por  ejemplo para sólo
% compilar un capítulo.
%
% El problema es que todos aquellos  ficheros que NO estén en la lista
% NO   se  incluirán...  y   eso  también   afecta  a   ficheros  de
% la plantilla...
%
% Total,  que definimos  una constante  con los  ficheros  que siempre
% vamos a querer compilar  (aquellos relacionados con configuración) y
% luego definimos \compilaCapitulo.
\newcommand{\ficherosBasicosTeXiS}{%
TeXiS/TeXiS_pream,TeXiS/TeXiS_cab,TeXiS/TeXiS_bib,TeXiS/TeXiS_cover%
}
\newcommand{\ficherosBasicosTexto}{%
constantes,guionado,Cascaras/bibliografia,config%
}
\newcommand{\compilaCapitulo}[1]{%
\includeonly{\ficherosBasicosTeXiS,\ficherosBasicosTexto,Capitulos/#1}%
}

\newcommand{\compilaApendice}[1]{%
\includeonly{\ficherosBasicosTeXiS,\ficherosBasicosTexto,Apendices/#1}%
}

%- - - - - - - - - - - - - - - - - - - - - - - - - - - - - - - - - - -
%            Preámbulo del documento. Configuraciones varias
%- - - - - - - - - - - - - - - - - - - - - - - - - - - - - - - - - - -

% Define  el  tipo  de  compilación que  estamos  haciendo.   Contiene
% definiciones  de  constantes que  cambian  el  comportamiento de  la
% compilación. Debe incluirse antes del paquete TeXiS/TeXiS.sty
%---------------------------------------------------------------------
%
%                          config.tex
%
%---------------------------------------------------------------------
%
% Contiene la  definición de constantes  que determinan el modo  en el
% que se compilará el documento.
%
%---------------------------------------------------------------------
%
% En concreto, podemos  indicar si queremos "modo release",  en el que
% no  aparecerán  los  comentarios  (creados  mediante  \com{Texto}  o
% \comp{Texto}) ni los "por  hacer" (creados mediante \todo{Texto}), y
% sí aparecerán los índices. El modo "debug" (o mejor dicho en modo no
% "release" muestra los índices  (construirlos lleva tiempo y son poco
% útiles  salvo  para   la  versión  final),  pero  sí   el  resto  de
% anotaciones.
%
% Si se compila con LaTeX (no  con pdflatex) en modo Debug, también se
% muestran en una esquina de cada página las entradas (en el índice de
% palabras) que referencian  a dicha página (consulta TeXiS_pream.tex,
% en la parte referente a show).
%
% El soporte para  el índice de palabras en  TeXiS es embrionario, por
% lo  que no  asumas que  esto funcionará  correctamente.  Consulta la
% documentación al respecto en TeXiS_pream.tex.
%
%
% También  aquí configuramos  si queremos  o  no que  se incluyan  los
% acrónimos  en el  documento final  en la  versión release.  Para eso
% define (o no) la constante \acronimosEnRelease.
%
% Utilizando \compilaCapitulo{nombre}  podemos también especificar qué
% capítulo(s) queremos que se compilen. Si no se pone nada, se compila
% el documento  completo.  Si se pone, por  ejemplo, 01Introduccion se
% compilará únicamente el fichero Capitulos/01Introduccion.tex
%
% Para compilar varios  capítulos, se separan sus nombres  con comas y
% no se ponen espacios de separación.
%
% En realidad  la macro \compilaCapitulo  está definida en  el fichero
% principal tesis.tex.
%
%---------------------------------------------------------------------


% Comentar la línea si no se compila en modo release.
% TeXiS hará el resto.
% ¡¡¡Si cambias esto, haz un make clean antes de recompilar!!!
\def\release{1}


% Descomentar la linea si se quieren incluir los
% acrónimos en modo release (en modo debug
% no se incluirán nunca).
% ¡¡¡Si cambias esto, haz un make clean antes de recompilar!!!
%\def\acronimosEnRelease{1}


% Descomentar la línea para establecer el capítulo que queremos
% compilar

% \compilaCapitulo{01Introduccion}
% \compilaCapitulo{02EstructuraYGeneracion}
% \compilaCapitulo{03Edicion}
% \compilaCapitulo{04Imagenes}
% \compilaCapitulo{05Bibliografia}
% \compilaCapitulo{06Makefile}

% \compilaApendice{01AsiSeHizo}

% Variable local para emacs, para  que encuentre el fichero maestro de
% compilación y funcionen mejor algunas teclas rápidas de AucTeX
%%%
%%% Local Variables:
%%% mode: latex
%%% TeX-master: "./Tesis.tex"
%%% End:


% Paquete de la plantilla
\usepackage{TeXiS/TeXiS}

% Incluimos el fichero con comandos de constantes
%---------------------------------------------------------------------
%
%                          constantes.tex
%
%---------------------------------------------------------------------
%
% Fichero que  declara nuevos comandos LaTeX  sencillos realizados por
% comodidad en la escritura de determinadas palabras
%
%---------------------------------------------------------------------

%%%%%%%%%%%%%%%%%%%%%%%%%%%%%%%%%%%%%%%%%%%%%%%%%%%%%%%%%%%%%%%%%%%%%%
% Comando: 
%
%       \titulo
%
% Resultado: 
%
% Escribe el título del documento.
%%%%%%%%%%%%%%%%%%%%%%%%%%%%%%%%%%%%%%%%%%%%%%%%%%%%%%%%%%%%%%%%%%%%%%
\def\titulo{\textsc{TeXiS}: Verificación y Certificación de Optimizaciones de EVM}

%%%%%%%%%%%%%%%%%%%%%%%%%%%%%%%%%%%%%%%%%%%%%%%%%%%%%%%%%%%%%%%%%%%%%%
% Comando: 
%
%       \autor
%
% Resultado: 
%
% Escribe el autor del documento.
%%%%%%%%%%%%%%%%%%%%%%%%%%%%%%%%%%%%%%%%%%%%%%%%%%%%%%%%%%%%%%%%%%%%%%
\def\autor{Marco Antonio y Pedro Pablo G\'omez Mart\'in}

% Variable local para emacs, para  que encuentre el fichero maestro de
% compilación y funcionen mejor algunas teclas rápidas de AucTeX

%%%
%%% Local Variables:
%%% mode: latex
%%% TeX-master: "tesis.tex"
%%% End:


% Sacamos en el log de la compilación el copyright
%\typeout{Copyright Marco Antonio and Pedro Pablo Gomez Martin}

%
% "Metadatos" para el PDF
%
\ifpdf\hypersetup{%
    pdftitle = {\titulo},
    pdfsubject = {Plantilla de Tesis},
    pdfkeywords = {Plantilla, LaTeX, tesis, trabajo de
      investigación, trabajo de Master},
    pdfauthor = {\textcopyright\ \autor},
    pdfcreator = {\LaTeX\ con el paquete \flqq hyperref\frqq},
    pdfproducer = {pdfeTeX-0.\the\pdftexversion\pdftexrevision},
    }
    \pdfinfo{/CreationDate (\today)}
\fi

% \begin{document}
% \begin{otherlanguage}{english}
% \chapter{Introduction}
\label{cap:introduction}

% \chapterquote{Frase célebre dicha por alguien inteligente}{Autor}
\chapterquote
{Ethereum, taken as a whole, can be viewed as a transaction-based state machine: we begin with a genesis state and incrementally execute transactions to morph it into some current state. It is this current state which we accept as the canonical “version” of the world of Ethereum.}
{Ethereum Yellow Paper}

\section{Motivation}
\label{sect:motivation}

% A distinctive feature of Ethereum is that transactions are programs, smart contracts, and computing a state transition requires to run the contract code to compute the next state. This capability
% is provided by the Ethereum Virtual Machine (EVM) that can execute programs written in EVM
% bytecode.


In the field of code optimization, EVM bytecode stands as a language with much to benefit from, as it
has some unique requirements which open up new dimensions for optimization. For general purpose
programming languages the optimization's focus are usually on execution time, memory usage or code size.
However, in the case of programs compiled to EVM bytecode, called \emph{smart contracts}, one must also
take into consideration other factors, such as the size of the compiled binary or the cost of the
programs execution. Every EVM instruction requires a fee (called \emph{gas}\footnote{See 
\url{https://www.evm.codes/}}) to be paid for its execution, and these prices can vary greatly in 
magnitude.

% https://costa.fdi.ucm.es/papers/costa/AlbertGHR22.pdf

However, the optimization of EVM bytecode is not as widespread as one could imagine. This is in part
due to the higher stakes that EVM programs, called \emph{smart contracts}, tend to work under. A bug
in a \emph{smart contract} tends to have a greater impact than most other programs, often resulting
in large monetary loses\footnote{See \url{https://www.gemini.com/cryptopedia/the-dao-hack-makerdao}}.
It's therefore understandable that \emph{smart contract} developers do not want to risk
being responsible of the semantics of their code being contaminated by a third party's tool.

To mitigate this risk, the \verb|FORVES|
\footnote{FORmaly Verified EVM optimizationS \url{https://github.com/costa-group/forves2/}}
project aims to develop a verifier which is able to guarantee that an optimization of a jump-free
sequence of EVM bytecode retains the same semantics as the original unoptimized version. This, however,
does not by itself suffice to appease the worries of the \emph{smart contract} developer, since we've
merely pushed the burden of trust from the optimizer to the verifier. If the verifier were to have a
bug, it could fail to recognize a valid optimization or, in the worst case, guarantee that an erroneous
optimization is in fact valid, impacting its users.

The solution is to, again, shift the burden of proof to another tool. For the \verb|FORVES| project the 
verifier has been written in Coq, a proof assistant and programming language which allows its users to 
define a formal specification and prove that their program adheres to it. The benefit of using a tool
such as Coq is that the burden of proof is placed on a small core subset of its features, which have
been deeply scrutinized by mathematicians, and out of which all the other functionalities are built.
Therefore, to trust that a Coq-certified program is correct is equivalent to trusting that the Coq  
kernel is correct, which is a much greater assurance. 
% TODO: add link to article on verifying Coq in Coq: https://www.openaccessgovernment.org/proof-assistants-2/80852/
Many different projects have taken advantage of this fact to develop pieces of critical software which
can be trusted upon \citep{ConCert}
% TODO: add references to other tools verified in Coq

The purpose of this project is to aid in the development of \verb|FORVES2|, a successor of \verb|FORVES|
which improves on its reasoning power by allowing it to take into account contextual information to
decide if two pieces of code are equivalent. 

The \verb|FORVES| verifier was devised to certify optimizations of the \verb|GASOL| superoptimizer. 
Superoptimization is a compilation technique that searches, for a given jump-free sequence of 
instructions, a semantically equivalent sequence of instructions which is optimal in some metric, like
memory usage or execution cost. Since the superoptimizer requires these sequences of instructions to not
have bifurcations, to optimize a program it first extracts all the sequences of instructions which don't 
perform jumps and optimizes those separately before reassembling the program. In doing so, it remembers
which conditions triggered those jumps so it can gain more information on which states of the program
are possible for each section of the program. Consider the following example

\begin{center}
\begin{tikzpicture}[auto,
  node distance = 12mm and 20mm,
  start chain = going below,
  box/.style = {draw,rounded corners,blur shadow,fill=white,
        on chain,align=center}]
 \node[box] (b2)    {\verb|x0|\mintinline{haskell}{ = ...} \\ \verb|x1|\mintinline{haskell}{ = ...} \\ \verb|x0| - \verb|x1| $\ge$ 128?};      
 \node[box] (b3)    {\mintinline{haskell}{MEM[x0] = ...} \\ \mintinline{haskell}{MEM[x1] = ...}};  
 \node[box,right=of b2] (b4)    {\dots};     
 \begin{scope}[rounded corners,-latex]
 \path 
  (b2) edge node{\color{red}{\texttt{false}}} (b4)
  (b2) edge node{\mintinline{coq}{true}} (b3);
 \end{scope}
\end{tikzpicture}
\end{center}

The third block is only reachable if the jump condition is met, which means that \verb|x0| $\ge$ 
\verb|x1| $+ 128$. This information can be useful when optimizing a piece of code, and must therefore
be taken into consideration when certifying that two sequences of instructions are semantically 
equivalent. 


Before introducing an example of equivalent sequences of instructions in
EVM bytecode we need to give a simple introduction on programming in the
EVM to be able to follow the example. EVM instructions modify the state
of the EVM in two main ways:
\begin{itemize}
    \item Manipulating the values in the stack. \\
    In our simulation of the execution of a program this will be represented
    by showing the state of the stack before and after the execution of the 
    instruction. For example, consider the 
    \textcolor{red}{\texttt{ADD}} instruction which pushes to the top of the 
    stack the addition of its first two elements.
    \begin{Verbatim}[commandchars=\\\{\}]
    [x0,x1] 
     ↓ \textcolor{red}{ADD}
    [x0+x1]
    \end{Verbatim}

    \item Producing an effect. \\
    An effect is any change which is not reflected in the stack but can be 
    observed by an external process. The only instruction we are interested in
    which produces an effect is the \textcolor{red}{\texttt{MSTORE}} 
    instruction, which stores at the address referenced by the top element in 
    the stack the value at the position after it. Notice that the instruction
    besides producing an effect also manipulates the values in the stack. We
    represent effects, if present, by putting them in parenthesis next to the 
    state of the stack.
    \begin{Verbatim}[commandchars=\\\{\}]
    [x0,x1]
     ↓ \textcolor{red}{MSTORE}
    []   \textcolor{teal}{(MEM[x0] = x1)}
    \end{Verbatim}
\end{itemize}

Besides \textcolor{red}{\texttt{ADD}} and \textcolor{red}{\texttt{MSTORE}} we also
describe the following instructions.
\begin{itemize}
    \item \textcolor{red}{\texttt{SWAP1}} exchanges the top element in the stack with the
    second element bellow it.
    \begin{Verbatim}[commandchars=\\\{\}]
    [x0,x1]
     ↓ \textcolor{red}{SWAP1}
    [x1,x0]
    \end{Verbatim}

    \item \textcolor{red}{\texttt{SWAP2} } exchanges the top element in the stack with the
    second element bellow it.
    \begin{Verbatim}[commandchars=\\\{\}]
    [x0,x1,x2]
     ↓ \textcolor{red}{SWAP2}
    [x2,x1,x0]
    \end{Verbatim}

    \item \textcolor{red}{\texttt{SWAP3}} exchanges the top element in the stack with the
    third element bellow it.
    \begin{Verbatim}[commandchars=\\\{\}]
    [x0,x1,x2,x3]
     ↓ \textcolor{red}{SWAP3}
    [x3,x1,x2,x0]
    \end{Verbatim}

    \item \textcolor{red}{\texttt{POP}} discards the top element of the stack.
    \begin{Verbatim}[commandchars=\\\{\}]
    [x0] 
     ↓ \textcolor{red}{POP}
    []
    \end{Verbatim}
\end{itemize}

With these we can finally move on to analyze the following example. Consider
these two sequences of EVM instructions.

\begin{listing}[!ht]
{\color{red}
\begin{verbatim}
SWAP3 SWAP1 SWAP2 MSTORE SWAP1 SWAP2 ADD SWAP1 MSTORE
\end{verbatim}
}
\caption{Original EVM program}
\label{lst:evm-original}
\end{listing}

\begin{listing}[!ht]
{\color{red}
\begin{verbatim}
MSTORE MSTORE POP 
\end{verbatim}
}
\caption{Optimized EVM program}
\label{lst:evm-optimized}
\end{listing}

Even though we refer to them as ``Original'' and ``Optimized'', at first glance these 
two programs don't appear to be equivalent. Let's consider the results of their
executions.

Consider the execution of the first program (\ref{lst:evm-original}).

\begin{Verbatim}[commandchars=\\\{\}]
[x0,x1,x2,x3,x4,x5]
 ↓ \textcolor{red}{SWAP3}
[x3,x1,x2,x0,x4,x5]
 ↓ \textcolor{red}{SWAP1}
[x1,x3,x2,x0,x4,x5]
 ↓ \textcolor{red}{SWAP2}
[x2,x3,x1,x0,x4,x5]
 ↓ \textcolor{red}{MSTORE}
[x1,x0,x4,x5] \textcolor{teal}{(MEM[x2] = x3)}
 ↓ \textcolor{red}{SWAP1}
[x0,x1,x4,x5] \textcolor{teal}{(MEM[x2] = x3)}
 ↓ \textcolor{red}{SWAP2}
[x4,x1,x0,x5] \textcolor{teal}{(MEM[x2] = x3)}
 ↓ \textcolor{red}{ADD}
[x4+x1,x0,x5] \textcolor{teal}{(MEM[x2] = x3)}
 ↓ \textcolor{red}{SWAP1}
[x0,x4+x1,x5] \textcolor{teal}{(MEM[x2] = x3)}
 ↓ \textcolor{red}{MSTORE}
[x5]          \textcolor{teal}{(MEM[x2] = x3; MEM[x0] = x4+x1)}
\end{Verbatim}

Finally, consider the execution of the second program (\ref{lst:evm-optimized})

\begin{Verbatim}[commandchars=\\\{\}]
[x0,x1,x2,x3,x4,x5]
 ↓ \textcolor{red}{MSTORE}
[x2,x3,x4,x5] \textcolor{teal}{(MEM[x0] = x1)}
 ↓ \textcolor{red}{MSTORE}
[x4,x5]       \textcolor{teal}{(MEM[x0] = x1; MEM[x2] = x3)}
 ↓ \textcolor{red}{POP}
[x5]          \textcolor{teal}{(MEM[x0] = x1; MEM[x2] = x3)}
\end{Verbatim}

While the end result of the stacks are equivalent for both executions, we can't ensure that the
effects they produce are equivalent. For starters, we cannot ensure that the memory accesses can
be commuted. After all, if \verb|x0| $=$ \verb|x2| $= 0$ then after the first program we'd have
{\textcolor{teal}{\texttt{MEM[0]~=~x4+x1}}} but after the second we'd have 
{\textcolor{teal}{\texttt{MEM[0]~=~x3}}}, which need not be the same if \verb|x3| $\ne$ \verb|x4+x1|.
Furthermore, we'd need that \verb|x4+x1| $=$ \verb|x1|, which is only possible if \verb|x4| $= 0$.

However, some of these conditions may be fulfilled if we consider the contextual information.
For example, consider the following contextual information, a list of added constraints to the
previous variables.

\begin{itemize}
    \item \verb|x4| $=$ \verb|x5|
    \item \verb|x5| $=$ 0
    \item \verb|x0| $\ge$ \verb|x2| $+ 128$
\end{itemize}

From the last constraint we can derive that the memory accesses to the offsets
referenced by \verb|x0| and \verb|x2| are disjoint, since the word size of the EVM is of 32 bytes and 
$32 < 128$, and therefore we can reorder both writes while preserving the semantics of the code.
From the first two constraints we can derive that \verb|x4| $= 0$ which ensures that \verb|x4| $+$ 
\verb|x1| is in fact equivalent to \verb|x1|.


The purpose of this project is to aid in the development of \verb|FORVES2|, a successor of 
\verb|FORVES| which is able to take into account this kind of contextual information to verify whether
two sequences of instructions are equivalent. To do so we need to be able to reason if a set of
constraints are met from the context, which is what this project sets out to do, by developing a 
certified ``implication checker'' which tests whether some constraints can be derived from the current
context.

\section{Objectives}
\label{sect:objectives}

\begin{itemize}
    \item Objective 1
    \item Objective 2
    \item \emph{and most importantly...} Objective 3!
\end{itemize}

\section{Work plan}
\label{sect:work-plan}

In order to achieve the previous objectives, the following plan was developed.

\begin{itemize}
    \item Define what contextual information is useful for the verifier
    \item Define the specification of an implementation checker
    \item Develop and certify a basic implementation checker
    \item Develop and certify a more complex implementation checker.
\end{itemize}
% \end{otherlanguage}
% \end{document}

%- - - - - - - - - - - - - - - - - - - - - - - - - - - - - - - - - - -
%                        Documento
%- - - - - - - - - - - - - - - - - - - - - - - - - - - - - - - - - - -
\begin{document}

% Incluimos el  fichero de definición de guionado  de algunas palabras
% que LaTeX no ha dividido como debería
%----------------------------------------------------------------
%
%                          guionado.tex
%
%----------------------------------------------------------------
%
% Fichero con algunas divisiones de palabras que LaTeX no
% hace correctamente si no se le da alguna ayuda.
%
%----------------------------------------------------------------

\hyphenation{
% a
abs-trac-to
abs-trac-tos
abs-trac-ta
abs-trac-tas
ac-tua-do-res
a-gra-de-ci-mien-tos
ana-li-za-dor
an-te-rio-res
an-te-rior-men-te
apa-rien-cia
a-pro-pia-do
a-pro-pia-dos
a-pro-pia-da
a-pro-pia-das
a-pro-ve-cha-mien-to
a-que-llo
a-que-llos
a-que-lla
a-que-llas
a-sig-na-tu-ra
a-sig-na-tu-ras
a-so-cia-da
a-so-cia-das
a-so-cia-do
a-so-cia-dos
au-to-ma-ti-za-do
% b
batch
bi-blio-gra-fía
bi-blio-grá-fi-cas
bien
bo-rra-dor
boo-l-ean-expr
% c
ca-be-ce-ra
call-me-thod-ins-truc-tion
cas-te-lla-no
cir-cuns-tan-cia
cir-cuns-tan-cias
co-he-ren-te
co-he-ren-tes
co-he-ren-cia
co-li-bri
co-men-ta-rio
co-mer-cia-les
co-no-ci-mien-to
cons-cien-te
con-si-de-ra-ba
con-si-de-ra-mos
con-si-de-rar-se
cons-tan-te
cons-trucción
cons-tru-ye
cons-tru-ir-se
con-tro-le
co-rrec-ta-men-te
co-rres-pon-den
co-rres-pon-dien-te
co-rres-pon-dien-tes
co-ti-dia-na
co-ti-dia-no
crean
cris-ta-li-zan
cu-rri-cu-la
cu-rri-cu-lum
cu-rri-cu-lar
cu-rri-cu-la-res
% d
de-di-ca-do
de-di-ca-dos
de-di-ca-da
de-di-ca-das
de-rro-te-ro
de-rro-te-ros
de-sa-rro-llo
de-sa-rro-llos
de-sa-rro-lla-do
de-sa-rro-lla-dos
de-sa-rro-lla-da
de-sa-rro-lla-das
de-sa-rro-lla-dor
de-sa-rro-llar
des-cri-bi-re-mos
des-crip-ción
des-crip-cio-nes
des-cri-to
des-pués
de-ta-lla-do
de-ta-lla-dos
de-ta-lla-da
de-ta-lla-das
di-a-gra-ma
di-a-gra-mas
di-se-ños
dis-po-ner
dis-po-ni-bi-li-dad
do-cu-men-ta-da
do-cu-men-to
do-cu-men-tos
% e
edi-ta-do
e-du-ca-ti-vo
e-du-ca-ti-vos
e-du-ca-ti-va
e-du-ca-ti-vas
e-la-bo-ra-do
e-la-bo-ra-dos
e-la-bo-ra-da
e-la-bo-ra-das
es-co-llo
es-co-llos
es-tu-dia-do
es-tu-dia-dos
es-tu-dia-da
es-tu-dia-das
es-tu-dian-te
e-va-lua-cio-nes
e-va-lua-do-res
exis-ten-tes
exhaus-ti-va
ex-pe-rien-cia
ex-pe-rien-cias
% f
for-ma-li-za-do
% g
ge-ne-ra-ción
ge-ne-ra-dor
ge-ne-ra-do-res
ge-ne-ran
% h
he-rra-mien-ta
he-rra-mien-tas
% i
i-dio-ma
i-dio-mas
im-pres-cin-di-ble
im-pres-cin-di-bles
in-de-xa-do
in-de-xa-dos
in-de-xa-da
in-de-xa-das
in-di-vi-dual
in-fe-ren-cia
in-fe-ren-cias
in-for-ma-ti-ca
in-gre-dien-te
in-gre-dien-tes
in-me-dia-ta-men-te
ins-ta-la-do
ins-tan-cias
% j
% k
% l
len-gua-je
li-be-ra-to-rio
li-be-ra-to-rios
li-be-ra-to-ria
li-be-ra-to-rias
li-mi-ta-do
li-te-ra-rio
li-te-ra-rios
li-te-ra-ria
li-te-ra-rias
lo-tes
% m
ma-ne-ra
ma-nual
mas-que-ra-de
ma-yor
me-mo-ria
mi-nis-te-rio
mi-nis-te-rios
mo-de-lo
mo-de-los
mo-de-la-do
mo-du-la-ri-dad
mo-vi-mien-to
% n
na-tu-ral
ni-vel
nues-tro
% o
obs-tan-te
o-rien-ta-do
o-rien-ta-dos
o-rien-ta-da
o-rien-ta-das
% p
pa-ra-le-lo
pa-ra-le-la
par-ti-cu-lar
par-ti-cu-lar-men-te
pe-da-gó-gi-ca
pe-da-gó-gi-cas
pe-da-gó-gi-co
pe-da-gó-gi-cos
pe-rio-di-ci-dad
per-so-na-je
plan-te-a-mien-to
plan-te-a-mien-tos
po-si-ción
pre-fe-ren-cia
pre-fe-ren-cias
pres-cin-di-ble
pres-cin-di-bles
pri-me-ra
pro-ble-ma
pro-ble-mas
pró-xi-mo
pu-bli-ca-cio-nes
pu-bli-ca-do
% q
% r
rá-pi-da
rá-pi-do
ra-zo-na-mien-to
ra-zo-na-mien-tos
re-a-li-zan-do
re-fe-ren-cia
re-fe-ren-cias
re-fe-ren-cia-da
re-fe-ren-cian
re-le-van-tes
re-pre-sen-ta-do
re-pre-sen-ta-dos
re-pre-sen-ta-da
re-pre-sen-ta-das
re-pre-sen-tar-lo
re-qui-si-to
re-qui-si-tos
res-pon-der
res-pon-sa-ble
% s
se-pa-ra-do
si-guien-do
si-guien-te
si-guien-tes
si-guie-ron
si-mi-lar
si-mi-la-res
si-tua-ción
% t
tem-pe-ra-ments
te-ner
trans-fe-ren-cia
trans-fe-ren-cias
% u
u-sua-rio
Unreal-Ed
% v
va-lor
va-lo-res
va-rian-te
ver-da-de-ro
ver-da-de-ros
ver-da-de-ra
ver-da-de-ras
ver-da-de-ra-men-te
ve-ri-fi-ca
% w
% x
% y
% z
}
% Variable local para emacs, para que encuentre el fichero
% maestro de compilación
%%%
%%% Local Variables:
%%% mode: latex
%%% TeX-master: "./Tesis.tex"
%%% End:


% Marcamos  el inicio  del  documento para  la  numeración de  páginas
% (usando números romanos para esta primera fase).
\frontmatter
\pagestyle{empty}

%---------------------------------------------------------------------
%
%                          configCover.tex
%
%---------------------------------------------------------------------
%
% cover.tex
% Copyright 2009 Marco Antonio Gomez-Martin, Pedro Pablo Gomez-Martin
%
% This file belongs to the TeXiS manual, a LaTeX template for writting
% Thesis and other documents. The complete last TeXiS package can
% be obtained from http://gaia.fdi.ucm.es/projects/texis/
%
% Although the TeXiS template itself is distributed under the 
% conditions of the LaTeX Project Public License
% (http://www.latex-project.org/lppl.txt), the manual content
% uses the CC-BY-SA license that stays that you are free:
%
%    - to share & to copy, distribute and transmit the work
%    - to remix and to adapt the work
%
% under the following conditions:
%
%    - Attribution: you must attribute the work in the manner
%      specified by the author or licensor (but not in any way that
%      suggests that they endorse you or your use of the work).
%    - Share Alike: if you alter, transform, or build upon this
%      work, you may distribute the resulting work only under the
%      same, similar or a compatible license.
%
% The complete license is available in
% http://creativecommons.org/licenses/by-sa/3.0/legalcode
%
%---------------------------------------------------------------------
%
% Fichero que contiene la configuración de la portada y de la 
% primera hoja del documento.
%
%---------------------------------------------------------------------


% Pueden configurarse todos los elementos del contenido de la portada
% utilizando comandos.

%%%%%%%%%%%%%%%%%%%%%%%%%%%%%%%%%%%%%%%%%%%%%%%%%%%%%%%%%%%%%%%%%%%%%%
% Título del documento:
% \tituloPortada{titulo}
% Nota:
% Si no se define se utiliza el del \titulo. Este comando permite
% cambiar el título de forma que se especifiquen dónde se quieren
% los retornos de carro cuando se utilizan fuentes grandes.
%%%%%%%%%%%%%%%%%%%%%%%%%%%%%%%%%%%%%%%%%%%%%%%%%%%%%%%%%%%%%%%%%%%%%%
\tituloPortada{%
Verificación y Certificación de Optimizaciones de EVM
}


%%%%%%%%%%%%%%%%%%%%%%%%%%%%%%%%%%%%%%%%%%%%%%%%%%%%%%%%%%%%%%%%%%%%%%
% Título del documento en inglés:
% \tituloPortadaEng{titulo}
% Nota:
% Si no se define se utiliza el del \titulo. Este comando permite
% cambiar el título de forma que se especifiquen dónde se quieren
% los retornos de carro cuando se utilizan fuentes grandes.
%%%%%%%%%%%%%%%%%%%%%%%%%%%%%%%%%%%%%%%%%%%%%%%%%%%%%%%%%%%%%%%%%%%%%%
\tituloPortadaEng{%
Verification and Certification of EVM Optimizations
}

%%%%%%%%%%%%%%%%%%%%%%%%%%%%%%%%%%%%%%%%%%%%%%%%%%%%%%%%%%%%%%%%%%%%%%
% Autor del documento:
% \autorPortada{Nombre}
% Se utiliza en la portada y en el valor por defecto del
% primer subtítulo de la segunda portada.
%%%%%%%%%%%%%%%%%%%%%%%%%%%%%%%%%%%%%%%%%%%%%%%%%%%%%%%%%%%%%%%%%%%%%%
\autorPortada{Fernando Isaías Leal Sánchez}

%%%%%%%%%%%%%%%%%%%%%%%%%%%%%%%%%%%%%%%%%%%%%%%%%%%%%%%%%%%%%%%%%%%%%%
% Fecha de publicación:
% \fechaPublicacion{Fecha}
% Puede ser vacío. Aparece en la última línea de ambas portadas
%%%%%%%%%%%%%%%%%%%%%%%%%%%%%%%%%%%%%%%%%%%%%%%%%%%%%%%%%%%%%%%%%%%%%%
% Descomentar para que ponga siempre la fecha actual
\fechaPublicacion{\today}
%\fechaPublicacion{\textcolor{red}{DIA de MES de AÑO}}

%%%%%%%%%%%%%%%%%%%%%%%%%%%%%%%%%%%%%%%%%%%%%%%%%%%%%%%%%%%%%%%%%%%%%%
% Imagen de la portada (y escala)
% \imagenPortada{Fichero}
% \escalaImagenPortada{Numero}
% Si no se especifica, se utiliza la imagen TODO.pdf
%%%%%%%%%%%%%%%%%%%%%%%%%%%%%%%%%%%%%%%%%%%%%%%%%%%%%%%%%%%%%%%%%%%%%%
% imagen en blanco y negro
%\imagenPortada{Imagenes/Vectorial/escudoUCM}
%imagen en color
\imagenPortada{Imagenes/Bitmap/escudoUCMcolor}
\escalaImagenPortada{.2}

%%%%%%%%%%%%%%%%%%%%%%%%%%%%%%%%%%%%%%%%%%%%%%%%%%%%%%%%%%%%%%%%%%%%%%
% Tipo de documento.
% \tipoDocumento{Tipo}
% Para el texto justo debajo del escudo.
% Si no se indica, se utiliza "TESIS DOCTORAL".
%%%%%%%%%%%%%%%%%%%%%%%%%%%%%%%%%%%%%%%%%%%%%%%%%%%%%%%%%%%%%%%%%%%%%%
\tipoDocumento{Trabajo de Fin de Grado}

%%%%%%%%%%%%%%%%%%%%%%%%%%%%%%%%%%%%%%%%%%%%%%%%%%%%%%%%%%%%%%%%%%%%%%
% Institución/departamento asociado al documento.
% \institucion{Nombre}
% Puede tener varias líneas. Se utiliza en las dos portadas.
% Si no se indica aparecerá vacío.
%%%%%%%%%%%%%%%%%%%%%%%%%%%%%%%%%%%%%%%%%%%%%%%%%%%%%%%%%%%%%%%%%%%%%%
\institucion{%
Grado en {Ingeniería Informática} y Matemáticas\\[0.2em]
Facultad de Informática\\[0.2em]
Universidad Complutense de Madrid
}

%%%%%%%%%%%%%%%%%%%%%%%%%%%%%%%%%%%%%%%%%%%%%%%%%%%%%%%%%%%%%%%%%%%%%%
% Director del trabajo.
% \directorPortada{Nombre}
% Se utiliza para el valor por defecto del segundo subtítulo, donde
% se indica quién es el director del trabajo.
% Si se fuerza un subtítulo distinto, no hace falta definirlo.
%%%%%%%%%%%%%%%%%%%%%%%%%%%%%%%%%%%%%%%%%%%%%%%%%%%%%%%%%%%%%%%%%%%%%%
\directorPortada{
Samir Genaim
\\
Enrique Martín Martín
}


%%%%%%%%%%%%%%%%%%%%%%%%%%%%%%%%%%%%%%%%%%%%%%%%%%%%%%%%%%%%%%%%%%%%%%
% Colaborador en la dirección del trabajo.
% \colaboradorPortada{Nombre}
% Se utiliza para el valor por defecto del segundo subtítulo, donde
% se indica quién es el colaborador en la dirección del trabajo.
% Si se fuerza un subtítulo distinto, no hace falta definirlo.
%%%%%%%%%%%%%%%%%%%%%%%%%%%%%%%%%%%%%%%%%%%%%%%%%%%%%%%%%%%%%%%%%%%%%%
\colaboradorPortada{NA}


%%%%%%%%%%%%%%%%%%%%%%%%%%%%%%%%%%%%%%%%%%%%%%%%%%%%%%%%%%%%%%%%%%%%%%
% Texto del primer subtítulo de la segunda portada.
% \textoPrimerSubtituloPortada{Texto}
% Para configurar el primer "texto libre" de la segunda portada.
% Si no se especifica se indica "Memoria que presenta para optar al
% título de Doctor en Informática" seguido del \autorPortada.
%%%%%%%%%%%%%%%%%%%%%%%%%%%%%%%%%%%%%%%%%%%%%%%%%%%%%%%%%%%%%%%%%%%%%%
\textoPrimerSubtituloPortada{%
\textbf{
Memoria del trabajo de fin de grado
de ingeniería informática del Doble Grado de Ingeniería Informática y Matemáticas
}\\ [0.3em]
}

%%%%%%%%%%%%%%%%%%%%%%%%%%%%%%%%%%%%%%%%%%%%%%%%%%%%%%%%%%%%%%%%%%%%%%
% Texto del segundo subtítulo de la segunda portada.
% \textoSegundoSubtituloPortada{Texto}
% Para configurar el segundo "texto libre" de la segunda portada.
% Si no se especifica se indica "Dirigida por el Doctor" seguido
% del \directorPortada.
%%%%%%%%%%%%%%%%%%%%%%%%%%%%%%%%%%%%%%%%%%%%%%%%%%%%%%%%%%%%%%%%%%%%%%
\textoSegundoSubtituloPortada{%
\textbf{Convocatoria: }{Mayo} \the\year%\\[0.2em]
%\textbf{Calificación: }\textit{\textcolor{red}{Nota}}
}

%%%%%%%%%%%%%%%%%%%%%%%%%%%%%%%%%%%%%%%%%%%%%%%%%%%%%%%%%%%%%%%%%%%%%%
% \explicacionDobleCara
% Si se utiliza, se aclara que el documento está preparado para la
% impresión a doble cara.
%%%%%%%%%%%%%%%%%%%%%%%%%%%%%%%%%%%%%%%%%%%%%%%%%%%%%%%%%%%%%%%%%%%%%%
%\explicacionDobleCara

%%%%%%%%%%%%%%%%%%%%%%%%%%%%%%%%%%%%%%%%%%%%%%%%%%%%%%%%%%%%%%%%%%%%%%
% \isbn
% Si se utiliza, aparecerá el ISBN detrás de la segunda portada.
%%%%%%%%%%%%%%%%%%%%%%%%%%%%%%%%%%%%%%%%%%%%%%%%%%%%%%%%%%%%%%%%%%%%%%
%\isbn{978-84-692-7109-4}


%%%%%%%%%%%%%%%%%%%%%%%%%%%%%%%%%%%%%%%%%%%%%%%%%%%%%%%%%%%%%%%%%%%%%%
% \copyrightInfo
% Si se utiliza, aparecerá información de los derechos de copyright
% detrás de la segunda portada.
%%%%%%%%%%%%%%%%%%%%%%%%%%%%%%%%%%%%%%%%%%%%%%%%%%%%%%%%%%%%%%%%%%%%%%
%\copyrightInfo{\autor}


%%
%% Creamos las portadas
%%
\makeCover

% Variable local para emacs, para que encuentre el fichero
% maestro de compilación
%%%
%%% Local Variables:
%%% mode: latex
%%% TeX-master: "../Tesis.tex"
%%% End:

%\include{Cascaras/autorizacion}
% +--------------------------------------------------------------------+
% | Dedication Page (Optional)
% +--------------------------------------------------------------------+

\chapter*{Dedicatoria}

\begin{flushright}
\begin{minipage}[c]{8.5cm}
\flushright{\textit{A Pedro Pablo y Marco Antonio, por crear TeXiS e iluminar nuestro camino}}
\flushright{\textit{A mis padres y hermano.}}
\end{minipage}
\end{flushright}
% +--------------------------------------------------------------------+
% | Acknowledgements Page (Optional)                                   |
% +--------------------------------------------------------------------+

\chapter*{Agradecimientos}

A Guillermo, por el tiempo empleado en hacer estas plantillas. A Adrián, Enrique y Nacho, por sus comentarios para mejorar lo que hicimos. Y a Narciso, a quien no le ha hecho falta el Anillo Único para coordinarnos a todos.

\chapter*{Acknowledgements}
To Johannes Hostert, Ike Mulder, Théo Winterhalter, Meven Lennon-Bertrand, Pierre-Marie Pédrot, Pierre Roux and Kenji Maillard, for their time and help when I asked questions in Coq's Zulip chat.












\chapter*{Resumen}

\section*{\tituloPortadaVal}


La optimización del \emph{bytecode} de la máquina virtual de Ethereum (EVM) presenta retos y
oportunidades únicos debido a sus requisitos específicos, que incluyen consideraciones como 
el coste de ejecución y el tamaño del ejecutable compilado. A pesar de sus potenciales 
beneficios, la optimización en este ámbito sigue siendo limitada, principalmente debido a los 
altos riesgos asociados con los programas EVM, llamados contratos inteligentes, donde incluso
errores menores pueden resultar en pérdidas financieras significativas. Para abordar estas 
preocupaciones, el proyecto FORVES desarrolla un verificador, aprovechando las técnicas de 
verificación formal para garantizar que el código de bytes optimizado conserve su semántica
original. Al emplear Coq, un asistente de pruebas y lenguaje de programación rigurosamente 
verificado, FORVES aumenta la confianza en los procesos de optimización, trasladando la carga 
de la prueba del optimizador al verificador. Este trabajo ocurre en el entorno de \verb|FORVES2|,
un sucesor del proyecto anterior que incorpora información contextual para evaluar la equivalencia
del código. En concreto, el proyecto pretende diseñar un verificador de implicaciones capaz de 
comprobar si ciertas restricciones octogonales se cumple en contextos específicos, empleando para
su validación un algoritmo de cálculo del cierre transitivo.

\section*{Palabras clave}

\noindent Coq, Restricciones octogonales, Programación funcional, Smart contracts, Ethereum, EVM, Blockchain
\begin{otherlanguage}{english}
\chapter*{Abstract}

\section*{\tituloPortadaEngVal}

An abstract in English, half a page long, including the title in English. Below, a list with no more than 10 keywords.


\section*{Keywords}

\noindent Coq, Octogonal restrictions, Functional programming
% \noindent 10 keywords max., separated by commas.




% Si el trabajo se escribe en inglés, comentar esta línea y descomentar
% otra igual que hay justo antes de \end{document}
% \end{otherlanguage}

\ifx\generatoc\undefined
\else
%---------------------------------------------------------------------
%
%                          TeXiS_toc.tex
%
%---------------------------------------------------------------------
%
% TeXiS_toc.tex
% Copyright 2009 Marco Antonio Gomez-Martin, Pedro Pablo Gomez-Martin
%
% This file belongs to TeXiS, a LaTeX template for writting
% Thesis and other documents. The complete last TeXiS package can
% be obtained from http://gaia.fdi.ucm.es/projects/texis/
%
% This work may be distributed and/or modified under the
% conditions of the LaTeX Project Public License, either version 1.3
% of this license or (at your option) any later version.
% The latest version of this license is in
%   http://www.latex-project.org/lppl.txt
% and version 1.3 or later is part of all distributions of LaTeX
% version 2005/12/01 or later.
%
% This work has the LPPL maintenance status `maintained'.
% 
% The Current Maintainers of this work are Marco Antonio Gomez-Martin
% and Pedro Pablo Gomez-Martin
%
%---------------------------------------------------------------------
%
% Contiene  los  comandos  para  generar los  índices  del  documento,
% entendiendo por índices las tablas de contenidos.
%
% Genera  el  índice normal  ("tabla  de  contenidos"),  el índice  de
% figuras y el de tablas. También  crea "marcadores" en el caso de que
% se esté compilando con pdflatex para que aparezcan en el PDF.
%
%---------------------------------------------------------------------


% Primero un poquito de configuración...


% Pedimos que inserte todos los epígrafes hasta el nivel \subsection en
% la tabla de contenidos.
\setcounter{tocdepth}{2} 

% Le  pedimos  que nos  numere  todos  los  epígrafes hasta  el  nivel
% \subsubsection en el cuerpo del documento.
\setcounter{secnumdepth}{3} 


% Creamos los diferentes índices.

% Lo primero un  poco de trabajo en los marcadores  del PDF. No quiero
% que  salga una  entrada  por cada  índice  a nivel  0...  si no  que
% aparezca un marcador "Índices", que  tenga dentro los otros tipos de
% índices.  Total, que creamos el marcador "Índices".
% Antes de  la creación  de los índices,  se añaden los  marcadores de
% nivel 1.

\ifpdf
   \pdfbookmark{Índices}{indices}
\fi

% Tabla de contenidos.
%
% La  inclusión  de '\tableofcontents'  significa  que  en la  primera
% pasada  de  LaTeX  se  crea   un  fichero  con  extensión  .toc  con
% información sobre la tabla de contenidos (es conceptualmente similar
% al  .bbl de  BibTeX, creo).  En la  segunda ejecución  de  LaTeX ese
% documento se utiliza para  generar la verdadera página de contenidos
% usando la  información sobre los  capítulos y demás guardadas  en el
% .toc
\ifpdf
   \pdfbookmark[1]{Tabla de Contenidos}{tabla de contenidos}
\fi

\cabeceraEspecial{\'Indice}

\tableofcontents

\newpage 

% Índice de figuras
%
% La idea es semejante que para  el .toc del índice, pero ahora se usa
% extensión .lof (List Of Figures) con la información de las figuras.

\ifpdf
   \pdfbookmark[1]{Índice de figuras}{indice de figuras}
\fi

\cabeceraEspecial{\'Indice de figuras}

%\listoffigures

\newpage

% Índice de tablas
% Como antes, pero ahora .lot (List Of Tables)

\ifpdf
   \pdfbookmark[1]{Índice de tablas}{indice de tablas}
\fi

\cabeceraEspecial{\'Indice de tablas}

%\listoftables

\newpage

% Variable local para emacs, para  que encuentre el fichero maestro de
% compilación y funcionen mejor algunas teclas rápidas de AucTeX

%%%
%%% Local Variables:
%%% mode: latex
%%% TeX-master: "../Tesis.tex"
%%% End:

\fi

% Marcamos el  comienzo de  los capítulos (para  la numeración  de las
% páginas) y ponemos la cabecera normal
\mainmatter

\pagestyle{fancy}
\restauraCabecera


\chapter{Introduction}
\label{cap:introduction}

% \chapterquote{Frase célebre dicha por alguien inteligente}{Autor}
\chapterquote
{Ethereum, taken as a whole, can be viewed as a transaction-based state machine: we begin with a genesis state and incrementally execute transactions to morph it into some current state. It is this current state which we accept as the canonical “version” of the world of Ethereum.}
{Ethereum Yellow Paper}

\section{Motivation}
\label{sect:motivation}

% A distinctive feature of Ethereum is that transactions are programs, smart contracts, and computing a state transition requires to run the contract code to compute the next state. This capability
% is provided by the Ethereum Virtual Machine (EVM) that can execute programs written in EVM
% bytecode.


In the field of code optimization, EVM bytecode stands as a language with much to benefit from, as it
has some unique requirements which open up new dimensions for optimization. For general purpose
programming languages the optimization's focus are usually on execution time, memory usage or code size.
However, in the case of programs compiled to EVM bytecode, called \emph{smart contracts}, one must also
take into consideration other factors, such as the size of the compiled binary or the cost of the
programs execution. Every EVM instruction requires a fee (called \emph{gas}\footnote{See 
\url{https://www.evm.codes/}}) to be paid for its execution, and these prices can vary greatly in 
magnitude.

% https://costa.fdi.ucm.es/papers/costa/AlbertGHR22.pdf

However, the optimization of EVM bytecode is not as widespread as one could imagine. This is in part
due to the higher stakes that EVM programs, called \emph{smart contracts}, tend to work under. A bug
in a \emph{smart contract} tends to have a greater impact than most other programs, often resulting
in large monetary loses\footnote{See \url{https://www.gemini.com/cryptopedia/the-dao-hack-makerdao}}.
It's therefore understandable that \emph{smart contract} developers do not want to risk
being responsible of the semantics of their code being contaminated by a third party's tool.

To mitigate this risk, the \verb|FORVES|
\footnote{FORmaly Verified EVM optimizationS \url{https://github.com/costa-group/forves2/}}
project aims to develop a verifier which is able to guarantee that an optimization of a jump-free
sequence of EVM bytecode retains the same semantics as the original unoptimized version. This, however,
does not by itself suffice to appease the worries of the \emph{smart contract} developer, since we've
merely pushed the burden of trust from the optimizer to the verifier. If the verifier were to have a
bug, it could fail to recognize a valid optimization or, in the worst case, guarantee that an erroneous
optimization is in fact valid, impacting its users.

The solution is to, again, shift the burden of proof to another tool. For the \verb|FORVES| project the 
verifier has been written in Coq, a proof assistant and programming language which allows its users to 
define a formal specification and prove that their program adheres to it. The benefit of using a tool
such as Coq is that the burden of proof is placed on a small core subset of its features, which have
been deeply scrutinized by mathematicians, and out of which all the other functionalities are built.
Therefore, to trust that a Coq-certified program is correct is equivalent to trusting that the Coq  
kernel is correct, which is a much greater assurance. 
% TODO: add link to article on verifying Coq in Coq: https://www.openaccessgovernment.org/proof-assistants-2/80852/
Many different projects have taken advantage of this fact to develop pieces of critical software which
can be trusted upon \citep{ConCert}
% TODO: add references to other tools verified in Coq

The purpose of this project is to aid in the development of \verb|FORVES2|, a successor of \verb|FORVES|
which improves on its reasoning power by allowing it to take into account contextual information to
decide if two pieces of code are equivalent. 

The \verb|FORVES| verifier was devised to certify optimizations of the \verb|GASOL| superoptimizer. 
Superoptimization is a compilation technique that searches, for a given jump-free sequence of 
instructions, a semantically equivalent sequence of instructions which is optimal in some metric, like
memory usage or execution cost. Since the superoptimizer requires these sequences of instructions to not
have bifurcations, to optimize a program it first extracts all the sequences of instructions which don't 
perform jumps and optimizes those separately before reassembling the program. In doing so, it remembers
which conditions triggered those jumps so it can gain more information on which states of the program
are possible for each section of the program. Consider the following example

\begin{center}
\begin{tikzpicture}[auto,
  node distance = 12mm and 20mm,
  start chain = going below,
  box/.style = {draw,rounded corners,blur shadow,fill=white,
        on chain,align=center}]
 \node[box] (b2)    {\verb|x0|\mintinline{haskell}{ = ...} \\ \verb|x1|\mintinline{haskell}{ = ...} \\ \verb|x0| - \verb|x1| $\ge$ 128?};      
 \node[box] (b3)    {\mintinline{haskell}{MEM[x0] = ...} \\ \mintinline{haskell}{MEM[x1] = ...}};  
 \node[box,right=of b2] (b4)    {\dots};     
 \begin{scope}[rounded corners,-latex]
 \path 
  (b2) edge node{\color{red}{\texttt{false}}} (b4)
  (b2) edge node{\mintinline{coq}{true}} (b3);
 \end{scope}
\end{tikzpicture}
\end{center}

The third block is only reachable if the jump condition is met, which means that \verb|x0| $\ge$ 
\verb|x1| $+ 128$. This information can be useful when optimizing a piece of code, and must therefore
be taken into consideration when certifying that two sequences of instructions are semantically 
equivalent. 


Before introducing an example of equivalent sequences of instructions in
EVM bytecode we need to give a simple introduction on programming in the
EVM to be able to follow the example. EVM instructions modify the state
of the EVM in two main ways:
\begin{itemize}
    \item Manipulating the values in the stack. \\
    In our simulation of the execution of a program this will be represented
    by showing the state of the stack before and after the execution of the 
    instruction. For example, consider the 
    \textcolor{red}{\texttt{ADD}} instruction which pushes to the top of the 
    stack the addition of its first two elements.
    \begin{Verbatim}[commandchars=\\\{\}]
    [x0,x1] 
     ↓ \textcolor{red}{ADD}
    [x0+x1]
    \end{Verbatim}

    \item Producing an effect. \\
    An effect is any change which is not reflected in the stack but can be 
    observed by an external process. The only instruction we are interested in
    which produces an effect is the \textcolor{red}{\texttt{MSTORE}} 
    instruction, which stores at the address referenced by the top element in 
    the stack the value at the position after it. Notice that the instruction
    besides producing an effect also manipulates the values in the stack. We
    represent effects, if present, by putting them in parenthesis next to the 
    state of the stack.
    \begin{Verbatim}[commandchars=\\\{\}]
    [x0,x1]
     ↓ \textcolor{red}{MSTORE}
    []   \textcolor{teal}{(MEM[x0] = x1)}
    \end{Verbatim}
\end{itemize}

Besides \textcolor{red}{\texttt{ADD}} and \textcolor{red}{\texttt{MSTORE}} we also
describe the following instructions.
\begin{itemize}
    \item \textcolor{red}{\texttt{SWAP1}} exchanges the top element in the stack with the
    second element bellow it.
    \begin{Verbatim}[commandchars=\\\{\}]
    [x0,x1]
     ↓ \textcolor{red}{SWAP1}
    [x1,x0]
    \end{Verbatim}

    \item \textcolor{red}{\texttt{SWAP2} } exchanges the top element in the stack with the
    second element bellow it.
    \begin{Verbatim}[commandchars=\\\{\}]
    [x0,x1,x2]
     ↓ \textcolor{red}{SWAP2}
    [x2,x1,x0]
    \end{Verbatim}

    \item \textcolor{red}{\texttt{SWAP3}} exchanges the top element in the stack with the
    third element bellow it.
    \begin{Verbatim}[commandchars=\\\{\}]
    [x0,x1,x2,x3]
     ↓ \textcolor{red}{SWAP3}
    [x3,x1,x2,x0]
    \end{Verbatim}

    \item \textcolor{red}{\texttt{POP}} discards the top element of the stack.
    \begin{Verbatim}[commandchars=\\\{\}]
    [x0] 
     ↓ \textcolor{red}{POP}
    []
    \end{Verbatim}
\end{itemize}

With these we can finally move on to analyze the following example. Consider
these two sequences of EVM instructions.

\begin{listing}[!ht]
{\color{red}
\begin{verbatim}
SWAP3 SWAP1 SWAP2 MSTORE SWAP1 SWAP2 ADD SWAP1 MSTORE
\end{verbatim}
}
\caption{Original EVM program}
\label{lst:evm-original}
\end{listing}

\begin{listing}[!ht]
{\color{red}
\begin{verbatim}
MSTORE MSTORE POP 
\end{verbatim}
}
\caption{Optimized EVM program}
\label{lst:evm-optimized}
\end{listing}

Even though we refer to them as ``Original'' and ``Optimized'', at first glance these 
two programs don't appear to be equivalent. Let's consider the results of their
executions.

Consider the execution of the first program (\ref{lst:evm-original}).

\begin{Verbatim}[commandchars=\\\{\}]
[x0,x1,x2,x3,x4,x5]
 ↓ \textcolor{red}{SWAP3}
[x3,x1,x2,x0,x4,x5]
 ↓ \textcolor{red}{SWAP1}
[x1,x3,x2,x0,x4,x5]
 ↓ \textcolor{red}{SWAP2}
[x2,x3,x1,x0,x4,x5]
 ↓ \textcolor{red}{MSTORE}
[x1,x0,x4,x5] \textcolor{teal}{(MEM[x2] = x3)}
 ↓ \textcolor{red}{SWAP1}
[x0,x1,x4,x5] \textcolor{teal}{(MEM[x2] = x3)}
 ↓ \textcolor{red}{SWAP2}
[x4,x1,x0,x5] \textcolor{teal}{(MEM[x2] = x3)}
 ↓ \textcolor{red}{ADD}
[x4+x1,x0,x5] \textcolor{teal}{(MEM[x2] = x3)}
 ↓ \textcolor{red}{SWAP1}
[x0,x4+x1,x5] \textcolor{teal}{(MEM[x2] = x3)}
 ↓ \textcolor{red}{MSTORE}
[x5]          \textcolor{teal}{(MEM[x2] = x3; MEM[x0] = x4+x1)}
\end{Verbatim}

Finally, consider the execution of the second program (\ref{lst:evm-optimized})

\begin{Verbatim}[commandchars=\\\{\}]
[x0,x1,x2,x3,x4,x5]
 ↓ \textcolor{red}{MSTORE}
[x2,x3,x4,x5] \textcolor{teal}{(MEM[x0] = x1)}
 ↓ \textcolor{red}{MSTORE}
[x4,x5]       \textcolor{teal}{(MEM[x0] = x1; MEM[x2] = x3)}
 ↓ \textcolor{red}{POP}
[x5]          \textcolor{teal}{(MEM[x0] = x1; MEM[x2] = x3)}
\end{Verbatim}

While the end result of the stacks are equivalent for both executions, we can't ensure that the
effects they produce are equivalent. For starters, we cannot ensure that the memory accesses can
be commuted. After all, if \verb|x0| $=$ \verb|x2| $= 0$ then after the first program we'd have
{\textcolor{teal}{\texttt{MEM[0]~=~x4+x1}}} but after the second we'd have 
{\textcolor{teal}{\texttt{MEM[0]~=~x3}}}, which need not be the same if \verb|x3| $\ne$ \verb|x4+x1|.
Furthermore, we'd need that \verb|x4+x1| $=$ \verb|x1|, which is only possible if \verb|x4| $= 0$.

However, some of these conditions may be fulfilled if we consider the contextual information.
For example, consider the following contextual information, a list of added constraints to the
previous variables.

\begin{itemize}
    \item \verb|x4| $=$ \verb|x5|
    \item \verb|x5| $=$ 0
    \item \verb|x0| $\ge$ \verb|x2| $+ 128$
\end{itemize}

From the last constraint we can derive that the memory accesses to the offsets
referenced by \verb|x0| and \verb|x2| are disjoint, since the word size of the EVM is of 32 bytes and 
$32 < 128$, and therefore we can reorder both writes while preserving the semantics of the code.
From the first two constraints we can derive that \verb|x4| $= 0$ which ensures that \verb|x4| $+$ 
\verb|x1| is in fact equivalent to \verb|x1|.


The purpose of this project is to aid in the development of \verb|FORVES2|, a successor of 
\verb|FORVES| which is able to take into account this kind of contextual information to verify whether
two sequences of instructions are equivalent. To do so we need to be able to reason if a set of
constraints are met from the context, which is what this project sets out to do, by developing a 
certified ``implication checker'' which tests whether some constraints can be derived from the current
context.

\section{Objectives}
\label{sect:objectives}

\begin{itemize}
    \item Objective 1
    \item Objective 2
    \item \emph{and most importantly...} Objective 3!
\end{itemize}

\section{Work plan}
\label{sect:work-plan}

In order to achieve the previous objectives, the following plan was developed.

\begin{itemize}
    \item Define what contextual information is useful for the verifier
    \item Define the specification of an implementation checker
    \item Develop and certify a basic implementation checker
    \item Develop and certify a more complex implementation checker.
\end{itemize}
\chapter*{Introducción}
\label{cap:introduccion}
\addcontentsline{toc}{chapter}{Introduction}

Introduction to the subject area. This chapter contains the translation of Chapter \ref{cap:introduccion}.











\chapter{State of the art}
\label{cap:state-of-the-art}

Something something
\chapter{Modelization of the problem in Coq}
\label{cap:definitions}

% X Define model
% X Define constraints (and literals)
% X Define contextual information (disjunctive normal form of constraitns)
% X Define implication between constraints
% X Define implication checker (both conj_imp_checker and imp_checker)
% X Define equivalence of imp_checker and conj_imp_checker
%    - Their difference is in terms of interface, but one can always derived from the other
%   This is a central theorem of chapter

% X Talk about the trivial implication checker

The purpose of \mintinline{coq}{FORVES} is to certify that two sequences of EVM instructions are 
semantically equivalent. To do so, we need to define what we mean by semantic equivalence.


\section{Semantic equivalence of programs}

We can represent a program as a function which transform state where by state we mean an assignment of 
variables to values. In generic programming languages state is usually visualized with a map or a
dictionary.

\begin{minted}{python}
state = {
    "x": 10,
    "y": 3,
}
\end{minted}

The reason we use maps to represent these objects is that they are memory efficient, since their 
cost in efficiency is proportional to the number of elements they assign. However, the canonical 
way in which we represent assignments in mathematics is through functions, and this is also the 
method preferred by Coq since it eases its use in proofs. Therefore, the state our program's 
transform can be though of as a function with the signature \mintinline{coq}{Variable -> Value}.

The Ethereum Virtual Machine is a stack-based virtual machine with access to memory. Therefore,
variable values can be either regions of memory or elements which have been pushed into the stack. 
Since the Ethereum Virtual Machine is a low-level virtual machine, it works on the byte level, usually
by grouping them in sequences of 32 called words. These words cover all possible values our variables 
can take.

\begin{minted}{coq}
Definition EVMWord := word EVMWordSize. 
(* EVMWordSize = BytesInEVMWord * EVMByteSize
               =             32 * 8
               =               256             *)
\end{minted}

Therefore, we can define the state of an EVM application to be a function from \mintinline{coq}{nat} 
to \mintinline{coq}{EVMWord}, which is precisely how they are defined in \verb|constraints.v|.

\begin{minted}{coq}
Definition assignment : Type := nat -> EVMWord.
\end{minted}

When a program is executed in the EVM a new stack is created for its execution, destroying it at the
end with whichever values it had left over. To make changes persistent these need to be assigned to
memory. We can then represent our EVM programs as functions of type
\mintinline{coq}{assignmet -> assignment} where \mintinline{coq}{assignment} represents the status of
memory. We can think of our variables as the regions in memory we are interested about.

Since a program is then just a function we can define semantic equality through the principle of
function extensionality, which says that if for all possible inputs two functions output the same
values then the functions must be the same.

\begin{minted}{coq}
Axiom functional_extensionality : forall {X Y: Type}
                                    {f g : X → Y},
  (forall (x:X), f x = g x) → f = g.
\end{minted}

Functional extensionality is not native to Coq, since it cannot be proven directly. Therefore, if we
wanted to use it we would need to add it as an axiom. Fortunately, adding this axiom is known to keep
Coq's logic consistent, which means that we cannot prove a proposition simultaneously to be true and 
false. Since it is harmless to add it, it is defined in Coq's standard library, under 
\mintinline{coq}{Coq.Logic.FunctionalExtensionality}, althought not in the form described before. Coq's functional
extensionality axiom is tweaked to also work with functions which return dependent types. Our 
\mintinline{coq}{functional_extensionality} proposition is then derived from this axiom.

\begin{minted}{coq}
Axiom functional_extensionality_dep : forall {A} {B : A -> Type},
  forall (f g : forall x : A, B x),
  (forall x, f x = g x) -> f = g.
\end{minted}

Unfortunately, proceeding with this approach is unfeasible since testing all possible assignments for
the complete program is too complex. Therefore, the approach followed by \verb|FORVES| is different.
Instead, it records the effects of the execution of both programs on a generic model of the state and
simplifies  them to a canonical state to see if they're equivalent, by applying simplification rules.
Furthermore, \verb|FORVES| doesn't test the transformation performed by the whole program, but only of 
jump-free sequences of instructions. This had the added benefit of not having to worry about 
conditionals with the added cost that we can no longer make any assumptions about the stack. However 
this is easily fixed by extending our \mintinline{coq}{assignment} definition to also include 
variables representing elements in the stack, not only offsets in memory.

Furthermore, we may also not need to consider all states, since not all states are reachable at a 
particular point in the code. For instance, if a piece of code is only reachable through an if 
condition which checks for an invariant $P$, we can assume that the invariant holds for the final
assignment. Consider the following snippets of code.

\begin{minted}{c}
if(x > y){
    z = x-y;
}
\end{minted}
\begin{minted}{c}
if(x > y){
    z = abs(x-y);
}
\end{minted}

The result of executing the section inside the \mintinline{c}{if} statement in the first snippet is 
setting the variable \mintinline{coq}{z} to \mintinline{coq}{x-y}, while in the second snippet it's
set to \mintinline{coq}{abs(x-y)}. Both of these effects are not equivalent if we consider an initial
state of \mintinline{js}{{x:0, y:1}} since then ${x-y} = -1 \ne 1 = \lvert-1\rvert$. However,
the condition in the \mintinline{coq}{if} statement discards these types of states. We can represent
this by making the condition in the \mintinline{coq}{if} statement a requirement for the model to 
follow.

Therefore, for some optimizations we may want to ask for some special condition to be applied.
For example, in the previous example we may consider simplifying \mintinline{coq}{abs(z)} to
\mintinline{coq}{z} if we're able to show that \mintinline{coq}{z > 0}. Sometimes, these conditions
are not explicitly stated in the context, but can instead be derived from it. Consider an optimization
\mintinline{coq}{add_zero}, which says that if we have \mintinline{coq}{x0 + x1} and 
\mintinline{coq}{x0} is zero then this is equivalent to just having \mintinline{coq}{x1}. This 
optimization can only be applied if we have the condition \mintinline{coq}{x0 = 0}. Now, consider the
following snippet.

\begin{minted}{c}
if (x > 0) f();
else if (x < 0) g();
else {
    y = y + x;
}
\end{minted}

When we reach the \mintinline{coq}{else} branch the contextual information we've recorded is that 
$x \le 0$ and $x \ge 0$, but not $x = 0$. However, we know that having $x \le 0$ and $x \ge 0$ implies 
$x = 0$. Therefore, to apply the optimization \mintinline{coq}{add_zero} we need to be able to show 
that the current contextual information we have implies the conditions we need, not that it's present
explicitly.

\section{Representation of constraints}

How do we represent this in Coq? For starters, we need to define what type of contextual information
are we working with. Since the Ethereum Virtual Machine works at the byte level, all its comparisons
are arithmetic, which means the only conditions we are interested in are equalities and inequalities.

\begin{minted}{coq}
Inductive constraint : Type :=
  | C_LT (l r : cliteral)
  | C_EQ (l r : cliteral)
  | C_LE (l r : cliteral).
\end{minted}

We call \mintinline{coq}{cliteral}s to the elements we compare. In general, we are interested in 
comparisons between variables, constants and variables offset by a constant. These are what our 
\mintinline{coq}{cliteral}s represent.

\begin{minted}{coq}
Inductive cliteral : Type :=
  | C_VAR (n : nat) (* x *)
  | C_VAL (n : N) (* c *)
  | C_VAR_DELTA (n: nat)(delta : N).(* x + c *)
\end{minted}

Armed with these definitions we can start to represent what we mean by contextual information. This 
information is obtained from the optimizer when it is performing the static analysis of the program, 
in particular from the branches in the code. After each conditional jump, if the jump was taken we 
can add the condition tested to the context, and if it was not taken we add the negation of the 
condition. If we have multiple nested conditions, these can be combined logically by using an 
\mintinline{coq}{and} connective. It could also happen that one same region of the code is reachable 
from different jump instructions, which themselves carry different contextual information. This is the 
case, for example, with function calls. In this case we can combine the contextual information through 
an \mintinline{coq}{or} connective. 

To ease the representation of contextual information, we represent these combined conditions in 
disjunctive normal form. In code, this is modeled as a list of lists, where the inner lists represent 
conjunctions and the outer list represents one big disjunction.

\begin{minted}{coq}
Notation conjunction := (list constraint).
Notation disjuntion := (list conjunction).
Definition constraints : Type := disjuntion.
\end{minted}

\section{Implication between constraints}

The type \mintinline{coq}{constraints} represents the context that is available at one point in the 
code. What we want to be able to do is check if some set of constraints imply a different one. But in 
order to define this we first need to define what it means that a set of constraints imply another.

In logic, implication is usually expressed by the symbol $\rightarrow$, although the symbol $\subset$ 
is sometimes also used for the same purpose. This last symbol sheds more light into how we might want 
to define implication in this case.

For any state, a condition \mintinline{coq}{c} may or may not hold. We say that a model satisfies a 
single constraint if after substituting the variables by the values assigned to them by the model the 
condition holds.

\begin{minted}{coq}
Definition satisfies_single_constraint 
    (model: assignment) (c: constraint) : bool :=
  let get_value := cliteral_to_nat model in 
  match c with
  | C_EQ l r => (get_value l =? get_value r)%N
  | C_LT l r => (get_value l <? get_value r)%N
  | C_LE l r => (get_value l <=? get_value r)%N
  end.
\end{minted}

We can extend this definition to conjunctions of constraints by requiring that the model satisfies all 
constraints at the same time.

\begin{minted}{coq}
Definition satisfies_conjunction 
    (model: assignment) (conj: conjunction): bool :=
  forallb (satisfies_single_constraint model) conj.
\end{minted}

Finally, we extend the definition to also hold for generic combinations of constraints in disjunctive 
normal form by specifying that at least one conjunction must hold for the whole formula to be true.

\begin{minted}{coq}
Definition satisfies_constraints 
    (model: assignment) (cs: constraints): bool :=
  forallb (satisfies_conjunction model) cs.
\end{minted}

Now, we can take inspiration from the $\subset$ notation of implication to define implication of 
constraints. We say that a set of constraints in disjunctive normal form \mintinline{coq}{cs} implies a constraints \mintinline{coq}{c} if,
for every model which satisfies \mintinline{coq}{cs}, it also satisfies \mintinline{coq}{c}. That is, if $State_{\texttt{cs}}$ is the set of
all states which satisfy $cs$ and $State_{\texttt{c}}$ is the set of all states which satisfy $c$, then
$State_{\texttt{cs}} \subset State_{\texttt{c}}$. In Coq, we would write

\begin{minted}{coq}
Definition imply(cs: constraints)(c: constraint) := forall (m: assignment),
  satisfies_constraints m cs = true -> satisfies_single_constraint m c = true.
\end{minted}

A similar definition follows for conjunctions instead of terms in DNF.
\begin{minted}{coq}
Definition conj_imply(cs: conjunction)(c: constraint) := forall (m: assignment),
  satisfies_conjunction m cs = true -> satisfies_single_constraint m c = true.
\end{minted}

\section{Implication checker}

We can finally define what we mean by an implication checker. In the end, an implication checker is 
just a function which takes some constraints as contextual information, which we will call the 
hypothesis, and a constraint we will call the thesis, and returns the boolean true if it is able to 
show that the hypothesis imply the thesis.

In a different programming language we would implement implication checkers as functions with the 
signature \mintinline{coq}{constaints -> constraint -> bool}, but in Coq we can go further and encode 
the implication checker's invariant as part of the type.

\begin{minted}{coq}
Record imp_checker: Type := 
  { imp_checker_fun: constraints -> constraint -> bool
  ; imp_checker_snd: forall (cs: constraints) (c: constraint),
      imp_checker_fun cs c = true -> imply cs c
  }.
\end{minted}

An implication checker is not just the function which checks for the implication, but also the proof of
that if the checking function says the implication is true, then it is a fact that the hypothesis 
imply the thesis.

To implement the implication checker, we need to handle the constraints in disjunctive normal form. 
However, we observe that in general we can consider the different conjunctions of constraints 
independently. The intuition behind it is that, if we need to prove that $A \vee B \rightarrow C$, 
then we need to prove $A \rightarrow C$ and $B \rightarrow C$ separately. This can actually be easily 
proven by using the identities $A \rightarrow B \equiv \neg A \vee B$ and the distributive property of 
$\wedge$ and $\vee$.

\begin{align*}
A \vee B \rightarrow C &\equiv \neg (A \vee B) \vee C \\
                       &\equiv \neg A \wedge \neg B \vee C \\
                       &\equiv (\neg A \vee C)\wedge (\neg B \vee C) \\
                       &\equiv (A \rightarrow C)\wedge (B \rightarrow C)
\end{align*}

% ~ (A \/ B) \/ C = ~A /\ ~B \/ C = (~ A \/ C) /\ (~ B \/ C)

In Coq we represent this by defining a different type of implication checker, one that only works with 
conjunctions, and showing that a conjunction implication checker can be used as a regular implication
checker and vice versa.

\begin{minted}{coq}
Record conj_imp_checker: Type := 
  { conj_imp_checker_fun: conjunction -> constraint -> bool
  ; conj_imp_checker_snd: forall (cs: conjunction) (c: constraint),
      conj_imp_checker_fun cs c = true -> conj_imply cs c
  }.
\end{minted}

That a \mintinline{coq}{imp_checker} can work as a \mintinline{coq}{conj_imp_checker} is obvious and 
not really useful, since a conjunction is trivially in disjunctive normal form, where there are no 
disjunctions. The interesting side of this equivalence is the converse, showing that given a 
\mintinline{coq}{conj_imp_checker} we can create an \mintinline{coq}{imp_checker}. We do this by 
defining the \mintinline{coq}{mk_imp_checker} function, of signature 
\mintinline{coq}{conj_imp_checker -> imp_checker}.

\begin{minted}{coq}
Program Definition mk_imp_checker 
    (checker: conj_imp_checker): imp_checker := {|
  imp_checker_fun (cs : constraints) c := 
    match cs with
    | [] => false
    | _ => forallb (fun conj => conj_imp_checker_fun checker conj c) cs
    end
|}.
\end{minted}

Since we included the implication checkers invariant inside its type, Coq wo not let us finish the definition
of this function until we have proven that this function preserves the invariant. That is why we need to use
the \mintinline{coq}{Program} prefix to this definition, to reassure Coq that we know that the definition is not finished 
with just the function, and that we will follow with the proof of the invariant.

As we saw before, given a conjunction implication checker we can define a more general implication checker
by running the conjunction implication checker on each conjunction in the hypothesis and making sure they 
all hold true. We know need to show that, given that the conjunction implication checker is sound, the
implication checker derived from it is sound as well. Let's look at the proof of this.

We use the \mintinline{coq}{Next Obligation} command to enter in proof mode after the \mintinline{coq}{Program} definition we used before.

\begin{minted}{coq}
Next Obligation.
  (* First we unfold the definition of imply and introduce the model in
     its definition *)
  unfold imply; intros model.
  (* Then we obtain the conjunction's checking function and soundness proof
     from the constructor of the conjunction implication checker *)
  destruct checker as [checker checker_snd].
  (* Next Obligation automatically introduces as many terms as possible. In
     particular, the goal it asks to prove is whether an assignment called 
     model satisfies the constraint c given that:
       H : the checker function we have derived says that cs imply c
     We will rename H to full_checker__cs_imp_c. Full checker is the
     checking function we have derived from the conjunction implication
     checker's *)
  rename H into full_checker__cs_imp_c.
  (* We proceed over induction on c. The base case is not reachable,
     so we discriminate. We're only left with the inductive case. *)
  induction cs as [|c' cs' IHcs']; try discriminate.
  simpl in *.
  (* Since we're left in the case where cs = c' :: cs' and the full
     checker said that cs imply c, by definition that means that
     c' imply c and cs' imply c. Notice that:
        cs': list list constraint (DNF)
         c': list constraint (conjunction)
         c : constraint
     *)
  apply Bool.andb_true_iff in full_checker__cs_imp_c 
    as [checker__c'_imp_c checker__cs'_imp_c].
  (* Moreso, to show that the disjuntion c' \⁄ cs'' implies c we
     need to that the model satisfies c both when it satisfies c' 
     and it satisfies cs'. *)
  intros h; apply Bool.orb_true_iff in h as [c'_sat | cs'_sat].
  - (* If the model satisfies c' then since the conjugation checker
       said that c' implies c, we have our result.*)
    exact (checker_snd _ _ checker__c'_imp_c model c'_sat).
  - (* Otherwise, we apply the induction hypothesis on cs' *)
    unfold is_model in cs'_sat.
    (* We can assume that cs' is nonempty, otherwise the proof would
       have concluded earlier *)
    destruct cs' as [|c'' cs'']; try discriminate.
    exact (IHcs' checker__cs'_imp_c cs'_sat).
Qed.
\end{minted}

Thanks to this function, which we have now proven that correctly yields
implication checkers from conjunction implication checkers, serves to ease the 
development of implication checkers. To define a conjunction implication checker
we no longer need to deal with DNF formulas, just with conjunctions. We can later
derive the full implication checker from the conjunction one.

Before we conclude this chapter, we must highlight an important fact. We have only
required our implication checkers to be sound, but we have not said anything about
completeness. This means that if an implication checker returns false we cannot 
assume that the implication is in fact false. Proving completeness is quite a hard
endeavour, and was therefore considered out of scope for this project.

This means that there is some type of partial order between implication checkers, 
depending on if how close to completeness they are. We can implement some trivial
implication checkers which are pretty far away from completeness. 

One such example would be the implication checker that always returns false. But a
more interesting example is the implication checker which just checks whether the
thesis is in the hypothesis.

\begin{minted}{coq}
Program Definition inclusion_conj_imp_checker: conj_imp_checker := {| 
  conj_imp_checker_fun := fun cs c => existsb (eqc c) cs
|}.
(* Proof left as an exercise to the reader *)
Definition inclusion_imp_checker := mk_imp_checker inclusion_conj_imp_checker.
\end{minted}

While we have defined our implication checkers to be best-effort, we of course would
like to have the best implication checker we could possible have. That is the 
purpose of the next chapter.
\chapter{Implication checker from transitive closure}
\label{cap:imp-checker}

Having defined what an implication checker is, we are now left with the task of
developing the most advanced implication checker we can muster. Originally I
decided to approach this with the idea of modelling constraints as a graph, where
our \mintinline{coq}{cliteral}s are the nodes of the graph and the constraints its edges. With 
this representation we would transform the problem of testing whether an constraint is
implied into finding a path in a graph, for which many solutions are already 
known.

Unfortunately, this original approach had a problem, and it had to do with 
allowing our \mintinline{coq}{cliteral}s to include variables with an offset.
By allowing $x$ and $x + d$ for $d \in \mathbb{N}$ to be represented by 
different nodes, we no longer have a cheap way of expressing that if $x + d \ge y + d'$ then $x + d + k \ge y + d' + k$ for every $k\in \mathbb{Z}$. The graph
idea may be salvaged, but our definition of \mintinline{coq}{cliteral} could not be
its nodes.

While investigating the prior work we came across \cite{TransitiveClosure}, where
the author describes a $O(n^3)$ algorithm to compute the tightened transitive 
closure of a set of octagonal constraints.

\section{Tightened transitive closure of octagonal constraints}
\label{sect:tightened-transitive-closure}

Octagonal constraints are constraints of the form $\pm x - \pm y \le d$ or 
$\pm x \le d$ where $x,y,d \in \mathbb{Z}$. We say that constraints of the form
$\pm x - \pm y \le d$ are addition constraints and of the form $\pm x \le d$ are
bound constraints.

Our constraints can be translated to octagonal constraints through the following 
transformations. First, we exchange every \mintinline{coq}{C_EQ} constraint with two
\mintinline{coq}{C_LE} constraints and every \mintinline{coq}{LT} constraints with a 
\mintinline{coq}{LE} constraint by taking advantage of the fact that since these are 
integer constraints, if $x < y$ then $x \le y-1$. Afterwards, we get rid of all 
inequalities of the form $d \le d'$ where $d$ ad $d'$ are constant values and not
variables. After all, these inequalities do not reveal any information, since they
are always true if the constraints are satisfiable, and we assume our constraints
are indeed satisfiable. Finally, we transform all \mintinline{coq}{cliteral}s of the
\mintinline{coq}{C_VAR} constructor to a \mintinline{coq}{C_VAR_DELTA} constructor, where
the \mintinline{coq}{delta} field is left as 0. From there, if our constraints are not
already in the form of an octagonal constraint, it is enough to multiple both sides
by $-1$ or move around terms from one side of the inequality to the other until 
they do.

Given an octagonal addition constraint of the form $ax + by\le d$ where $a,b\in
\{-1,1\}$, we say that $ax + by \le d$ \emph{trivially implies} $ax + by \le d'$
if $d \le d'$. We say that a set of constraints $C'$ is the transitive closure of
$C$ if every constraint implied by $C$ is trivially implied by a constraint in 
$C'$.

Therefore, if we had an algorithm to compute the transitive closure of a set of
constraints $C$ we could implement a conjunction implication checker testing
whether the thesis is trivially implied by any of the constraints in the 
transitive closure of the hypothesis. This algorithm is our final goal.

In \cite{TransitiveClosure} another simpler algorithm is mentioned to compute the
transitive closure of a set of octagonal constraints, first described in 
\cite{HarveyStuckey}. This algorithm relies on iteratively adding to an already 
transitively close set new additive constraints, along with all the constraints needed
to keep the set transitively closed.

In particular, if $A$ is a set of addition constraints and $B$ is a set of bound 
constraints, where $A \cup B$ is transitively closed, then 
${A \cup B \cup \{ax+by \le d\} \cup A' \cup B'}$ is also transitively closed, where
$A'$ is formed by the addition constraints
\begin{enumerate}[label={\small \boxed{\textbf{A{\arabic*}}}}]
    \item\label{rule:A1} $ez + by \le d + d'$ if $-ax + ez \le d'$, $z \ne y$
    \item\label{rule:A2} $ax + ft \le d + d'$ if $ft - by \le d'$, $t \ne x$
    \item\label{rule:A3} $ex + ft \le d + d' + d''$ if $-ax + ex \le d'$, $-by + ft \le d''$ and $t$, $z$, $x$ and $yz$ are all different.
\end{enumerate}

\noindent and $B'$ is formed by the bound constraints

\begin{enumerate}[label={\small \boxed{\textbf{B{\arabic*}}}}]
    \item\label{rule:B1} $by \le d + d'$ if $-ax \le d'$
    \item\label{rule:B2} $ax \le d + d'$ if $-by \le d'$
    \item\label{rule:B3} $ez \le d + d' + d''$ if $-ax \le d'$ and $-by + ex \le d''$ with $z \ne x$
    \item\label{rule:B4} $ft \le d + d' + d''$ if $-by \le d'$ and $-ax + ft \le d''$ with $t \ne y$
    \item\label{rule:B5} $by \le \lfloor \frac{d + d'}{2}\rfloor$ if $-ax + by \le d'$
    \item\label{rule:B6} $ax \le \lfloor \frac{d + d'}{2}\rfloor$ if $-by + ax \le d'$
    \item\label{rule:B7} $ez \le \lfloor \frac{d + d' + d''}{2}\rfloor$ if $-ax + ez \le d'$ and $-by + ez \le d''$
\end{enumerate}

With this result one can derive an algorithm to obtain the transitive closure of any
set of octogonal constraints. Given $C_A$ and $C_B$ sets of addition and bound 
constraints respectively, we start with $B = C_B$ and 
$A = \{ax + by \le d + d'\ :\  \forall ax\le d\in C_B, by\le d'\in C_B\}$ and 
iteratively add the elements in $C_A$ using the rules above, taking care to remove
any constraint which may be trivially implied by another.

Based on these rules we implemented a simpler algorithm which is just as powerful but
with the downside of requiring more iterations to derive the transitive closure.
The reason is that these rules are useful if we want to prove the completeness of the 
transitive closure, but since we are only interested in the soundness of the algorithm,
we can notice that some of these rules are actually the result of applying one or more
of the other rules.

In particular, we remove the rules:
\begin{itemize}
    \item  \ref{rule:A3} since it can be derived from successive uses of \ref{rule:A1}
           or \ref{rule:A2}.
    \item  \ref{rule:B3} since it can be derived from successive uses of \ref{rule:B1}
           and \ref{rule:B2}.
           $$
           \begin{array}{llll}
           -ax \le d'&\quad \phantom{-}ax + by \le d &\overset{\ref{rule:B1}}{\Longrightarrow} &by \le d + d' \\
           \\
           \phantom{-}bx \le d + d'&\quad -by + ex \le d'' &\overset{\ref{rule:B2}}{\Longrightarrow} &ez \le d + d' + d'' \\
           \end{array}
           $$
    \item  \ref{rule:B4} since it's the symmetric of \ref{rule:B3}.
    \item  \ref{rule:B7} since it can be obtained by successively applying \ref{rule:A1} and
           \ref{rule:B6} or \ref{rule:A2} and \ref{rule:B5}.
           $$
           \begin{array}{llll}
           -ax + ex \le d' & \phantom{-}ax + by \le d &\overset{\ref{rule:A1}}{\Longrightarrow} &
             ex + by \le d + d' \\
           \phantom{-}ex + by \le d + d' & -by + ex \le d'' &\overset{\ref{rule:B6}}{\Longrightarrow} &
             ex \le \lfloor\frac{d + d' + d''}{2}\rfloor \\
           \end{array}
           $$
\end{itemize}

If we eliminate the requirement that after each new constraint added the set must be 
transitively closed, we can actually go without these rules. After adding a new 
constraint we would just have to apply the rules also with the constraint we derived at
that step until no new constraints can be added. And since we no longer care if the
intermediate set of constraints is transitively closed or not, we can drop the 
requirement of adding the constraints iteratively and simply, for our currently built
set of constraints, test for every pair whether they can be combined and add the 
result if its not trivially implied by any of the other constraints already in the 
set.

Therefore, the final rules we have are:
\begin{enumerate}[label={\small \boxed{\textbf{R{\arabic*}}}}]
    \item\label{rule:R1}
    $ax + by \le d \quad -by + ez \le d' \quad\Longrightarrow ax + ez \le d + d'$
    \item\label{rule:R2}
    $ax + by \le d \quad -ax + ez \le d' \quad\Longrightarrow by + ez \le d + d'$
    \item\label{rule:R3}
    $ax + by \le d \quad \phantom{-}ez - ax  \le d' \quad\Longrightarrow by + ez \le d + d'$
    \item\label{rule:R4}
    $ax + by \le d \quad \phantom{-}ez - bx  \le d' \quad\Longrightarrow by + ez \le d + d'$
    \item\label{rule:R5}
    $ax + by \le d \quad -ax \le d' \quad\Longrightarrow by \le d + d'$
    \item\label{rule:R6}
    $ax + by \le d \quad -by \le d' \quad\Longrightarrow ax \le d + d'$
    \item\label{rule:R7}
    $ax \le d \quad by \le d' \quad\Longrightarrow ax + by \le d + d'$
\end{enumerate}

The algorithm, in pseudocode, would look like this
\begin{minted}{lua}
function TransitiveClosure(C: constraints) -> constraints
    repeat
        new_constraints := [
            combine(c1,c2) for c1 in C for c2 in C
        ]
        C = join(C, new_constraints)
    until C not changed
    return C
\end{minted}
\noindent We have to pay attention to not include unnecessary constraints when adding
new ones to the set, which is why a \mintinline{coq}{join} function is explicitly used.
Applying the rules is left for the \mintinline{coq}{combine} function.

\section{Implementation of the implication checker}

Since the representation of the constraints is not exactly the same for our new 
algorithm, we have opted to include these definitions in its own file, aptly named
\verb|octagon.v|. Implementing an implication checker as defined before in terms of
the algorithm developed in this chapter is left as future work\footnote{See Section 
\ref{sect:future-work} for more information.}. First of all, we need to define in Coq 
what are our octagonal constraints. Since there are two types of octagonal constraints,
additive and bound constraints, we represent this with two different constructors.

\begin{minted}{coq}
Inductive Constraint :=
 | AddConstr (l r: term)(d: Z)
 | BndConstr (t: term)(d: Z).
\end{minted}

In this definition, the terms represent the $\pm x$ elements in our constraints. To
represent the $\pm$ in Coq we use the subset $\{1,-1\} \subset \mathbb{Z}$.

\begin{minted}{coq}
Definition pmUnit := { z : Z | z = 1 \/ z = - 1 }.
Record term := {a: pmUnit; x: nat}.
\end{minted}

Variables in this representation are still identified with natural numbers, but since
they can now take any integer values we need to also change our previous \mintinline{coq}{assignment}
type definition, which we call now the type of \mintinline{coq}{model}s.

\begin{minted}{coq}
Definition model : Type := nat -> Z.
\end{minted}

Therefore, we say that a model \mintinline{coq}{m} satisfies a single constraint \mintinline{coq}{m}
if the following function returns true.

\begin{minted}{coq}
Definition satisfies_single_constraint 
    (m: model) (c: Constraint): bool  := 
  match c with 
  | AddConstr l r d => term_value m l + term_value m r <=? d
  | BndConstr t d => term_value m t <=? d
  end.
\end{minted}

\mintinline{coq}{term_value} is a function which retrieves the value of the variable 
\mintinline{coq}{x} according to the model \mintinline{coq}{m} and multiplies it by 
\mintinline{coq}{a} to get the value of the term under the model.

\begin{minted}{coq}
Definition term_value(m: model)(t: term): Z := 
    proj1_sig t.(a) * (m t.(x)).
\end{minted}

% Explain as a standalone theory
% Leave integration of the `octagon.v` file to `constraints.v` for "future work"

% Start with approaches that did not work -> Graph traversal
% Explain WHY it does not work (constraints are invariant to translation)
% Introduce the papers which inspired us to follow this approach

In this new file we have also ported over a more general definition of implication of
constraints, which works over conjunctions. Since we showed in the previous chapter
% TODO: Add ref
that conjunction implication checkers are equivalent to implication checkers, in this
file we drop the requirement of disjunctions in the constraints and simply treat our
hypothesis as conjunctions of constraints.

We also include a more general version of implication which works with a conjunction
of thesis constraints, and add some notation to help us work with it more comfortably.

\begin{minted}{coq}
Definition implication(C: list Constraint)(C': list Constraint) :=
  forall m, 
    satisfies_constraints m C = true ->
    satisfies_constraints m C' = true.
Infix "==>>" := implication (at level 96, right associativity).
\end{minted}

In this definition, \mintinline{coq}{satisfies_constraints} is implemented as a 
\mintinline{coq}{forallb} over the \mintinline{coq}{satisfies_single_constraint} 
function. Its definition reads as ``for every model which satisfies $C$, it also
satisfies $C'$''.

The usual facts about implication are also true for \mintinline{coq}{==>>}, most notably.

\begin{itemize}
    \item Reflexivity
\begin{minted}{coq}
Theorem implication_refl(C: list Constraint):
  (C ==>> C).
\end{minted}

    \item Transitivity
\begin{minted}{coq}
Theorem implication_trans(C C' C'': list Constraint):
  (C ==>> C') -> (C' ==>> C'') -> (C ==>> C'').
\end{minted}
  
\end{itemize}

With these new definitions we proceed to define the functions which will help us 
compute the transitive closure, or at least given a list of constraints $C$ help bring
it closer to its transitive closure.

Since our algorithm consists of performing successively a given number of iterations
over the list we define an \mintinline{coq}{iterate} which we will call consecutively to
iteratively bring closer our constraints $C$ to its transitive closure.

\begin{minted}{coq}
Definition iterate(C: list Constraint) : list Constraint :=
  let C' := new_constraints C in
  let C'' := flatten C' in
  join C C''.
\end{minted}

If we de-sugar the definition, we will see that \mintinline{coq}{iterate} is in fact the composition of
three different functions: 

\begin{itemize}
    \item \mintinline{coq}{new_constraints}
    \item \mintinline{coq}{flatten}
    \item \mintinline{coq}{join C}
\end{itemize}

\mintinline{coq}{new_constraints} itself is just a function which takes every pair of constraints in 
\mintinline{coq}{C} and collects all the constraints obtained from their combination.
These combinations come from the rules in Harvey's and Stuckey's algorithm\cite{HarveyStuckey},
but simplified to make our life easier. The rules themselves are implemented in the 
\mintinline{coq}{combine} function.

\begin{minted}{coq}
Definition new_constraints(C: list Constraint): list Constraint :=
  flat_map (fun c => 
    flat_map (fun c' => 
        opt_to_list(combine c c')
    ) C
  ) C.
  
Definition combine(c c': Constraint): option Constraint :=
  match c, c' with
  | AddConstr l r d, AddConstr l' r' d' =>
      if      l =? (op l') then Some (AddConstr r r' (d + d'))
      else if l =? (op r') then Some (AddConstr r l' (d + d'))
      else if r =? (op l') then Some (AddConstr l r' (d + d'))
      else if r =? (op r') then Some (AddConstr l l' (d + d'))
      else None
  | AddConstr l r d, BndConstr t' d' =>
      if      l =? (op t') then Some (BndConstr r (d + d'))
      else if r =? (op t') then Some (BndConstr l (d + d'))
      else None
  | BndConstr t d, AddConstr l' r' d' =>
      if      l' =? (op t) then Some (BndConstr r' (d + d'))
      else if r' =? (op t) then Some (BndConstr l' (d + d'))
      else None
  | BndConstr t d, BndConstr t' d' => Some (AddConstr t t' (d + d'))
  end.
\end{minted}

The \mintinline{coq}{flatten} function is just a map over the function 
\mintinline{coq}{normalize_constraints}, which performs a very simple transformation, those of 
the constraints of the form $x + x \le d$, which turn into $x \le \lfloor \frac{d}{2}\rfloor$.

\begin{minted}{coq}
Definition normalize_constraint(c: Constraint): Constraint :=
   match c with
   | AddConstr l r d => 
       if l =? r 
       then BndConstr l (d / 2)
       else c
   | c => c
   end.
\end{minted}

In fact, this is an important transformation, referred to as \emph{tightening} in the
literature \cite{TransitiveClosure}. Since we know that our variables hold integer
values, if we have a constraint of the form $x + y \le d + \epsilon$ where $\epsilon$
is a real number in the interval $(0,1)$ and $x,y,d\in \mathbb{Z}$ then we know that 
$x + y \le d$, since that is its closest integer value to $d+\epsilon$. Sometimes this
can allow us to improve upon our original constraints. Consider, for example, that we 
had the following constraints.

\begin{align}
    \label{C1} x + y &\le 10  \\
    \label{C2} x - y &\le 3 \\
    \label{C3} y \le 3
\end{align}

Then, combining \ref{C1} and \ref{C2} we can deduce that $x + x \le 13$, which we can
tighten into the bound constraint $x \le 6$. Then, combining this with \ref{C3} we 
obtain the addition constraint $x + y \le 9$, which is an improvement over \ref{C1}.

Finally the \mintinline{coq}{join} is potentially one of the most complicated functions
in this file. Its purpose is to merge two lists of constraints into one, but getting 
rid of those constraints which are duplicated or no longer needed.

\begin{minted}{coq}
Definition joined(c': Constraint)(cs: list Constraint) :=
  c' :: filter (fun c => negb (trivial_impl c' c)) cs.

Definition join(C C': list Constraint): list Constraint :=
  fold_left (fun cs c' => 
    if forallb (fun c => negb (trivial_impl c c')) cs
    then joined c' cs
    else cs
  ) C' C.
\end{minted}

The property we want to prove for \mintinline{coq}{iterate} is the following.

\begin{minted}{coq}
Theorem iterate_implication(C T: list Constraint):
  (C ==>> iterate C).
\end{minted}

This is but a corollary of the following theorems.
\begin{minted}{coq}
Theorem flatten_implication(C T: list Constraint):
  (C ==>> T) -> (C ==>> flatten T).
Theorem join_implication(C T: list Constraint):
  (C ==>> T) -> (C ==>> join C T).
Theorem new_constraints_implication(C T: list Constraint):
  (C ==>> T) -> (C ==>> new_constraints T).
\end{minted}

Once these are proven, the proof of \mintinline{coq}{iterate_implication} is as simple as performing
successive applications of these theorems.
\begin{minted}{coq}
Theorem iterate_implication(C: list Constraint):
  (C ==>> iterate C).
Proof.
  apply join_implication.
  apply flatten_implication.
  apply new_constraints_implication.
  apply implication_refl.
Qed.
\end{minted}

The proofs for these theorems can be found in the file \verb|octagon.v| in \verb|FORVES2|'s GitHub
repository.

% DONE: Elaborate a bit more here
% DONE: Update references to new "future work" section
% DONE: Explain scopes (such as %N)
% DONE: Eliminate contractions from the whole text
% DONE: Check there is no overflow of mintedinline in a page

With this \mintinline{coq}{iterate} function defined and the \mintinline{coq}{iterate_implication}
theorem proven, the only piece left to construct our implication checker is a function which would
allow us to apply the \mintinline{coq}{iterate} function over the original constraints a given
number of times.

\begin{minted}{coq}
Fixpoint church_numeral{A: Type}(n: nat)(f: A -> A)(x: A):=
  match n with
  | O => x
  | S m => f (church_numeral m f x)
  end.
\end{minted}

We call this function \mintinline{coq}{church_numeral} since it actually implements a translation
between Peano-encoded naturals (the \mintinline{coq}{nat} type in Coq) and Church-encoded naturals,
where the number \mintinline{coq}{n} corresponds to the function which takes another function 
\mintinline{coq}{f} and a value \mintinline{coq}{x} and applies the function over the value 
\mintinline{coq}{n} times.

With these definitions it is now trivial to construct an implication checker. The only decision
left to make is how many times we want to \mintinline{coq}{iterate}. A good choice ends up
being twice the number of our hypothesis constraints. This is because when deriving the rules of
our algorithm (implemented in \mintinline{coq}{combine}) we took care in ensuring that any rule
from Harvey's and Stuckey's algorithm\cite{HarveyStuckey} could be derived by applying two of 
our own, and their algorithm is proven to be complete. Even if this completeness is not proven
in Coq, it ends up being the optimal choice.

\begin{minted}{coq}
Program Definition conj_trans_closure_checker
    (n: nat) : conj_imp_checker := {|
  conj_imp_checker_fun cs c := 
    let trans_closure := church_numeral n iterate cs in 
    existsb (fun c' => trivial_impl c' c) trans_closure
|}.
Next Obligation. 
(* The proof proceeds by induction. We informally
   prove it for the case n=1, assuming n=0 as proven.
   Let  c' be a constraint in the transitive closure
   such that [c'] ==>> [c]. Then, we have:
            cs ==>> iterate cs    from iterate_implication
    iterate cs ==>> [c']          from imply_refl
          [c'] ==>> [c]           from our hypothesis
   Through transitivity, we conclude that cs ==>> [c]. *)
Qed. 
Program Definition trans_closure_checker(n: nat) : imp_checker 
    := mk_imp_checker (conj_trans_closure_checker n).
\end{minted}

Take note that the \mintinline{coq}{mk_imp_checker} used in this module is not
the one we defined in Chapter \ref{cap:definitions}, since we are using a different definition
to represent the state and constraints. Most notably, in this section we have used
integers while we previously restricted ourselves to natural numbers. An explanation
on how to bridge this implementation with the definitions in Chapter \ref{cap:definitions}
will be described in Section \ref{sect:future-work}.
% \chapter{Introduction}
\label{cap:introduction}

% \chapterquote{Frase célebre dicha por alguien inteligente}{Autor}
\chapterquote
{Ethereum, taken as a whole, can be viewed as a transaction-based state machine: we begin with a genesis state and incrementally execute transactions to morph it into some current state. It is this current state which we accept as the canonical “version” of the world of Ethereum.}
{Ethereum Yellow Paper}

\section{Motivation}
\label{sect:motivation}

% A distinctive feature of Ethereum is that transactions are programs, smart contracts, and computing a state transition requires to run the contract code to compute the next state. This capability
% is provided by the Ethereum Virtual Machine (EVM) that can execute programs written in EVM
% bytecode.


In the field of code optimization, EVM bytecode stands as a language with much to benefit from, as it
has some unique requirements which open up new dimensions for optimization. For general purpose
programming languages the optimization's focus are usually on execution time, memory usage or code size.
However, in the case of programs compiled to EVM bytecode, called \emph{smart contracts}, one must also
take into consideration other factors, such as the size of the compiled binary or the cost of the
programs execution. Every EVM instruction requires a fee (called \emph{gas}\footnote{See 
\url{https://www.evm.codes/}}) to be paid for its execution, and these prices can vary greatly in 
magnitude.

% https://costa.fdi.ucm.es/papers/costa/AlbertGHR22.pdf

However, the optimization of EVM bytecode is not as widespread as one could imagine. This is in part
due to the higher stakes that EVM programs, called \emph{smart contracts}, tend to work under. A bug
in a \emph{smart contract} tends to have a greater impact than most other programs, often resulting
in large monetary loses\footnote{See \url{https://www.gemini.com/cryptopedia/the-dao-hack-makerdao}}.
It's therefore understandable that \emph{smart contract} developers do not want to risk
being responsible of the semantics of their code being contaminated by a third party's tool.

To mitigate this risk, the \verb|FORVES|
\footnote{FORmaly Verified EVM optimizationS \url{https://github.com/costa-group/forves2/}}
project aims to develop a verifier which is able to guarantee that an optimization of a jump-free
sequence of EVM bytecode retains the same semantics as the original unoptimized version. This, however,
does not by itself suffice to appease the worries of the \emph{smart contract} developer, since we've
merely pushed the burden of trust from the optimizer to the verifier. If the verifier were to have a
bug, it could fail to recognize a valid optimization or, in the worst case, guarantee that an erroneous
optimization is in fact valid, impacting its users.

The solution is to, again, shift the burden of proof to another tool. For the \verb|FORVES| project the 
verifier has been written in Coq, a proof assistant and programming language which allows its users to 
define a formal specification and prove that their program adheres to it. The benefit of using a tool
such as Coq is that the burden of proof is placed on a small core subset of its features, which have
been deeply scrutinized by mathematicians, and out of which all the other functionalities are built.
Therefore, to trust that a Coq-certified program is correct is equivalent to trusting that the Coq  
kernel is correct, which is a much greater assurance. 
% TODO: add link to article on verifying Coq in Coq: https://www.openaccessgovernment.org/proof-assistants-2/80852/
Many different projects have taken advantage of this fact to develop pieces of critical software which
can be trusted upon \citep{ConCert}
% TODO: add references to other tools verified in Coq

The purpose of this project is to aid in the development of \verb|FORVES2|, a successor of \verb|FORVES|
which improves on its reasoning power by allowing it to take into account contextual information to
decide if two pieces of code are equivalent. 

The \verb|FORVES| verifier was devised to certify optimizations of the \verb|GASOL| superoptimizer. 
Superoptimization is a compilation technique that searches, for a given jump-free sequence of 
instructions, a semantically equivalent sequence of instructions which is optimal in some metric, like
memory usage or execution cost. Since the superoptimizer requires these sequences of instructions to not
have bifurcations, to optimize a program it first extracts all the sequences of instructions which don't 
perform jumps and optimizes those separately before reassembling the program. In doing so, it remembers
which conditions triggered those jumps so it can gain more information on which states of the program
are possible for each section of the program. Consider the following example

\begin{center}
\begin{tikzpicture}[auto,
  node distance = 12mm and 20mm,
  start chain = going below,
  box/.style = {draw,rounded corners,blur shadow,fill=white,
        on chain,align=center}]
 \node[box] (b2)    {\verb|x0|\mintinline{haskell}{ = ...} \\ \verb|x1|\mintinline{haskell}{ = ...} \\ \verb|x0| - \verb|x1| $\ge$ 128?};      
 \node[box] (b3)    {\mintinline{haskell}{MEM[x0] = ...} \\ \mintinline{haskell}{MEM[x1] = ...}};  
 \node[box,right=of b2] (b4)    {\dots};     
 \begin{scope}[rounded corners,-latex]
 \path 
  (b2) edge node{\color{red}{\texttt{false}}} (b4)
  (b2) edge node{\mintinline{coq}{true}} (b3);
 \end{scope}
\end{tikzpicture}
\end{center}

The third block is only reachable if the jump condition is met, which means that \verb|x0| $\ge$ 
\verb|x1| $+ 128$. This information can be useful when optimizing a piece of code, and must therefore
be taken into consideration when certifying that two sequences of instructions are semantically 
equivalent. 


Before introducing an example of equivalent sequences of instructions in
EVM bytecode we need to give a simple introduction on programming in the
EVM to be able to follow the example. EVM instructions modify the state
of the EVM in two main ways:
\begin{itemize}
    \item Manipulating the values in the stack. \\
    In our simulation of the execution of a program this will be represented
    by showing the state of the stack before and after the execution of the 
    instruction. For example, consider the 
    \textcolor{red}{\texttt{ADD}} instruction which pushes to the top of the 
    stack the addition of its first two elements.
    \begin{Verbatim}[commandchars=\\\{\}]
    [x0,x1] 
     ↓ \textcolor{red}{ADD}
    [x0+x1]
    \end{Verbatim}

    \item Producing an effect. \\
    An effect is any change which is not reflected in the stack but can be 
    observed by an external process. The only instruction we are interested in
    which produces an effect is the \textcolor{red}{\texttt{MSTORE}} 
    instruction, which stores at the address referenced by the top element in 
    the stack the value at the position after it. Notice that the instruction
    besides producing an effect also manipulates the values in the stack. We
    represent effects, if present, by putting them in parenthesis next to the 
    state of the stack.
    \begin{Verbatim}[commandchars=\\\{\}]
    [x0,x1]
     ↓ \textcolor{red}{MSTORE}
    []   \textcolor{teal}{(MEM[x0] = x1)}
    \end{Verbatim}
\end{itemize}

Besides \textcolor{red}{\texttt{ADD}} and \textcolor{red}{\texttt{MSTORE}} we also
describe the following instructions.
\begin{itemize}
    \item \textcolor{red}{\texttt{SWAP1}} exchanges the top element in the stack with the
    second element bellow it.
    \begin{Verbatim}[commandchars=\\\{\}]
    [x0,x1]
     ↓ \textcolor{red}{SWAP1}
    [x1,x0]
    \end{Verbatim}

    \item \textcolor{red}{\texttt{SWAP2} } exchanges the top element in the stack with the
    second element bellow it.
    \begin{Verbatim}[commandchars=\\\{\}]
    [x0,x1,x2]
     ↓ \textcolor{red}{SWAP2}
    [x2,x1,x0]
    \end{Verbatim}

    \item \textcolor{red}{\texttt{SWAP3}} exchanges the top element in the stack with the
    third element bellow it.
    \begin{Verbatim}[commandchars=\\\{\}]
    [x0,x1,x2,x3]
     ↓ \textcolor{red}{SWAP3}
    [x3,x1,x2,x0]
    \end{Verbatim}

    \item \textcolor{red}{\texttt{POP}} discards the top element of the stack.
    \begin{Verbatim}[commandchars=\\\{\}]
    [x0] 
     ↓ \textcolor{red}{POP}
    []
    \end{Verbatim}
\end{itemize}

With these we can finally move on to analyze the following example. Consider
these two sequences of EVM instructions.

\begin{listing}[!ht]
{\color{red}
\begin{verbatim}
SWAP3 SWAP1 SWAP2 MSTORE SWAP1 SWAP2 ADD SWAP1 MSTORE
\end{verbatim}
}
\caption{Original EVM program}
\label{lst:evm-original}
\end{listing}

\begin{listing}[!ht]
{\color{red}
\begin{verbatim}
MSTORE MSTORE POP 
\end{verbatim}
}
\caption{Optimized EVM program}
\label{lst:evm-optimized}
\end{listing}

Even though we refer to them as ``Original'' and ``Optimized'', at first glance these 
two programs don't appear to be equivalent. Let's consider the results of their
executions.

Consider the execution of the first program (\ref{lst:evm-original}).

\begin{Verbatim}[commandchars=\\\{\}]
[x0,x1,x2,x3,x4,x5]
 ↓ \textcolor{red}{SWAP3}
[x3,x1,x2,x0,x4,x5]
 ↓ \textcolor{red}{SWAP1}
[x1,x3,x2,x0,x4,x5]
 ↓ \textcolor{red}{SWAP2}
[x2,x3,x1,x0,x4,x5]
 ↓ \textcolor{red}{MSTORE}
[x1,x0,x4,x5] \textcolor{teal}{(MEM[x2] = x3)}
 ↓ \textcolor{red}{SWAP1}
[x0,x1,x4,x5] \textcolor{teal}{(MEM[x2] = x3)}
 ↓ \textcolor{red}{SWAP2}
[x4,x1,x0,x5] \textcolor{teal}{(MEM[x2] = x3)}
 ↓ \textcolor{red}{ADD}
[x4+x1,x0,x5] \textcolor{teal}{(MEM[x2] = x3)}
 ↓ \textcolor{red}{SWAP1}
[x0,x4+x1,x5] \textcolor{teal}{(MEM[x2] = x3)}
 ↓ \textcolor{red}{MSTORE}
[x5]          \textcolor{teal}{(MEM[x2] = x3; MEM[x0] = x4+x1)}
\end{Verbatim}

Finally, consider the execution of the second program (\ref{lst:evm-optimized})

\begin{Verbatim}[commandchars=\\\{\}]
[x0,x1,x2,x3,x4,x5]
 ↓ \textcolor{red}{MSTORE}
[x2,x3,x4,x5] \textcolor{teal}{(MEM[x0] = x1)}
 ↓ \textcolor{red}{MSTORE}
[x4,x5]       \textcolor{teal}{(MEM[x0] = x1; MEM[x2] = x3)}
 ↓ \textcolor{red}{POP}
[x5]          \textcolor{teal}{(MEM[x0] = x1; MEM[x2] = x3)}
\end{Verbatim}

While the end result of the stacks are equivalent for both executions, we can't ensure that the
effects they produce are equivalent. For starters, we cannot ensure that the memory accesses can
be commuted. After all, if \verb|x0| $=$ \verb|x2| $= 0$ then after the first program we'd have
{\textcolor{teal}{\texttt{MEM[0]~=~x4+x1}}} but after the second we'd have 
{\textcolor{teal}{\texttt{MEM[0]~=~x3}}}, which need not be the same if \verb|x3| $\ne$ \verb|x4+x1|.
Furthermore, we'd need that \verb|x4+x1| $=$ \verb|x1|, which is only possible if \verb|x4| $= 0$.

However, some of these conditions may be fulfilled if we consider the contextual information.
For example, consider the following contextual information, a list of added constraints to the
previous variables.

\begin{itemize}
    \item \verb|x4| $=$ \verb|x5|
    \item \verb|x5| $=$ 0
    \item \verb|x0| $\ge$ \verb|x2| $+ 128$
\end{itemize}

From the last constraint we can derive that the memory accesses to the offsets
referenced by \verb|x0| and \verb|x2| are disjoint, since the word size of the EVM is of 32 bytes and 
$32 < 128$, and therefore we can reorder both writes while preserving the semantics of the code.
From the first two constraints we can derive that \verb|x4| $= 0$ which ensures that \verb|x4| $+$ 
\verb|x1| is in fact equivalent to \verb|x1|.


The purpose of this project is to aid in the development of \verb|FORVES2|, a successor of 
\verb|FORVES| which is able to take into account this kind of contextual information to verify whether
two sequences of instructions are equivalent. To do so we need to be able to reason if a set of
constraints are met from the context, which is what this project sets out to do, by developing a 
certified ``implication checker'' which tests whether some constraints can be derived from the current
context.

\section{Objectives}
\label{sect:objectives}

\begin{itemize}
    \item Objective 1
    \item Objective 2
    \item \emph{and most importantly...} Objective 3!
\end{itemize}

\section{Work plan}
\label{sect:work-plan}

In order to achieve the previous objectives, the following plan was developed.

\begin{itemize}
    \item Define what contextual information is useful for the verifier
    \item Define the specification of an implementation checker
    \item Develop and certify a basic implementation checker
    \item Develop and certify a more complex implementation checker.
\end{itemize}
% \chapter{State of the art}
\label{cap:state-of-the-art}

Something something
% \include{Capitulos/3_DescriptionOfTheProject}
%\include{Capitulos/Capitulo4}
%\include{Capitulos/Capitulo5}
\chapter{Conclusions and Future Work}
\label{cap:conclusions}

Conclusions and future lines of work. This chapter contains the translation of Chapter.



\chapter*{Conclusiones y Trabajo Futuro}
\label{cap:conclusiones}
\addcontentsline{toc}{chapter}{Conclusiones y Trabajo Futuro}

En el Capítulo \ref{cap:imp-checker} implementamos un algoritmo que, cuando recibe una
lista de hipótesis de restricciones octogonales y otra restricción octogonal como tesis,
determina si la tesis se puede obtener de las hipótesis. Si este algoritmo determina que
es posible, también hemos verificado formalmente que la implicación siempre es correcta,
y por lo tanto se puede utilizar en distintas partes del proyecto.

Ahora podemos emplear nuestro algoritmo para resolver el problema introducido en el
Capítulo \ref{cap:state-of-the-art}. En este vimos que los Fragmentos \ref{lst:evm-original} 
y \ref{lst:evm-optimized} dejan la pila de la EVM en el mismo estado tras su ejecución,
pero que no se podían considerar semánticamente equivalentes a no ser que se cumplieran
ciertas restricciones. En concreto, estas restricciones eran.
\begin{align}
    \label{restr:T1} x_4 &= 0 \\
    \label{restr:T2} x_0 &\ge x_2 + 32
\end{align}

En la Sección \ref{sect:optimization-smart-contracts} probamos que \ref{restr:T1} y 
\ref{restr:T2} se podían derivar de las siguientes restricciones.

\begin{align*}
    x_4 &= x_5 \\
    x_5 &= 0 \\
    x_0 &\ge x_2 + 128
\end{align*}

Ahora, sin embargo, podemos usar el comprobador de implicaciones implementado en
el Capítulo \ref{cap:imp-checker} para comprobarlo. Primero, tenemos que transformar
nuestras hipótesis en restricciones octogonales. Para ello, seguimos las instrucciones
descritas en el Capítulo \ref{cap:definitions}.

\begin{align*}
    x_4 = x_5 &\Longrightarrow x_4 - _5 \le 0 \ \wedge\  x_5 - x_4 \le 0
    \\
    x_5 = 0 &\Longrightarrow x_5 \le 0 \ \wedge\ -x_5 \le 0
    \\
    x_0 \ge x_2 + 128 &\Longrightarrow x_2 - x_0 \le -128
\end{align*}

Para facilitarnos la construcción de varias restricciones, usamos las siguientes
funciones de conveniencia.

\begin{minted}{coq}
Program Definition mkadd_pp(x: nat)(y: nat)(d: Z) :=
  AddConstr (Build_term 1 x) (Build_term 1 y) d.
Program Definition mkadd_pn(x: nat)(y: nat)(d: Z) :=
  AddConstr (Build_term 1 x) (Build_term (-1) y) d.
Program Definition mkadd_nn(x: nat)(y: nat)(d: Z) :=
  AddConstr (Build_term (-1) x) (Build_term (-1) y) d.
Program Definition mkbnd_p(x: nat)(d: Z) :=
  BndConstr (Build_term 1 x) d.
Program Definition mkbnd_n(x: nat)(d: Z) :=
  BndConstr (Build_term (-1) x) d.
\end{minted}

Con estas funciones, podemos definir la lista de restricciones que representan
la información del contexto de nuestra aplicación.

\begin{minted}{coq}
Local Definition C := 
  (* Obtained from [x4 = x5] *)
  [ Octagon.mkadd_pn 4 5 0   (*  x4 - x5 <= 0 *)
  ; Octagon.mkadd_pn 5 4 0   (*  x5 - x4 <= 0 *)
  (* Obtained from [x5 = 0] *)
  ; Octagon.mkbnd_p 5 0      (*  x5 <= 0*)
  ; Octagon.mkbnd_n 5 0      (* -x5 <= 0*)
  (* Obtained from [x0 >= x2 + 128] *)
  ; Octagon.mkadd_pn 2 0 (-128) (*  x2 - x0 <= -128 *)
].
\end{minted}

Esta transformación de nuestras hipótesis también necesita realizarse para
nuestra tesis. En concreto, \ref{restr:T1}  se transforma en las 
restricciones $x_4 \le 0$ y $-x_4 \le 0$,  por lo que tendremos que comprobar 
que ambas se cumplen para poder decir que \ref{restr:T1} se cumple.

\begin{minted}{coq}
Local Definition checker: Constraint -> bool := 
    conj_imp_checker_fun (conj_trans_closure_checker (2 * length C)) C.

(* x4 = 0 *)
Compute checker (Octagon.mkbnd_p 4 0) && checker (Octagon.mkbnd_n 4 0).
(* x0 >= x2 + 32 *)
Compute checker (Octagon.mkadd_pn 2 0 (-32)).
\end{minted}
\vspace{-\baselineskip*3/2}
\begin{minted}[bgcolor=outputbg]{coq}
checker is defined

= true
     : bool

= true
     : bool
\end{minted}

\section{Trabajo futuro}
\label{sect:future-work}

El objetivo original de este trabajo era el desarrollar el comprobador de 
implicaciones y verificar su corrección. Por lo tanto, el objetivo del 
trabajo se puede dar por concluido. Sin embargo, aún queda trabajo por
hacer para que este algoritmo se pueda usar directamente en el proyecto
\verb|FORVES2|. Mientras que el algoritmo se ha verificado, no se puede
usar con las definiciones descritas en el Capítulo \ref{cap:definitions},
que se pueden encontrar en el archivo \mintinline{coq}{constraints.v}. Por
lo tanto, sería necesario conectar estas definiciones con las usadas en
la teoría del Capítulo \ref{cap:imp-checker}.

Los primeras definiciones a conectar serían las del estado:
\mintinline{coq}{Octagon.model} y \mintinline{coq}{Constraints.assignment}.
Ambas representan funciones que toman números naturales, pero 
\mintinline{coq}{Octagon.model} devuelve enteros (\mintinline{coq}{Z}) mientras
\mintinline{coq}{Constraints.assignment} devuelve naturales (\mintinline{coq}{N}).
El primer paso sería mostrar que \mintinline{coq}{Octagon.model} contiene a
\mintinline{coq}{Constraints.assignment}, y dar una inyección explícita.
\begin{minted}{coq}
Definition translate_model:
    Constraint.assigment -> Octagon.model.
\end{minted}

También sería necesario implementar la traducción de restricciones descrita en el
Capítulo \ref{cap:definitions}, de \mintinline{coq}{Constraints.constraint}s a
\mintinline{coq}{Octagon.Constraint}s usando las reglas descritas en el Capítulo
\ref{cap:definitions}. Podemos asumir que el tipo de esta transformación sería similar al
siguiente.
\begin{minted}{coq}
Definition translate_constraints: 
    Constraint.conjunction -> list Octagon.Constraint.
\end{minted}

\noindent Finalmente deberíamos demostrar que la función \mintinline{coq}{translate_constraints}
preserva la información contenida en las restricciones.
Es decir, que un modelo satisface unas restricciones si y solamente si el modelo traducido
también satisface la transformación de las restricciones.

\begin{minted}{coq}
Lemma translate_preserves_information(C: list Constraint.constraint) :
  forall (m: Constraint.assigment),
    let m_oct = translate_model m in
    let C_oct = translate_constraints C in
      Constraint.satisfies_conjunction m C = true <->
      Octagon.satisfies_constraints m_oct C_oct = true.
\end{minted}

Con estos resultados podríamos definir un \mintinline{coq}{Constraint.conj_imp_checker}
que funcionaría traduciendo las restricciones del modulo \mintinline{coq}{Constraint} a las
de \mintinline{coq}{Octagon} y preguntando si se pueden derivar las tesis transformadas de las
hipótesis con \mintinline{coq}{Octagon.conj_trans_closure_checker}. Cabe recalcar que como 
nuestras transformaciones están definidas para conjuntos de restricciones, una restricción al 
traducirse puede resultar en varias restricciones. En estos casos se comprobaría que cada una
de las restricciones resultantes se cumple para verificar que se cumple la tesis. Usando el lema
\mintinline{coq}{translate_preserves_information} 
podríamos probar que la corrección de \mintinline{coq}{Octagon.conj_trans_closure_checker} 
implica la de nuestro comprobador.

Más allá de la conexión entre las teorías descritas en \mintinline{coq}{constraints.v} y
\mintinline{coq}{octagon.v}, también existe la posibilidad de mejorar el tiempo de ejecución
del algoritmo de clausura transitiva. Por ejemplo, \cite{TransitiveClosure} menciona un algoritmo
que utiliza matrices para representar una colección de restricciones. Además de ser una 
representación más eficiente, también simplificaría la implementación de la función
\mintinline{coq}{join}.

También se podría abordar la cuestión desde el punto de vista de desacoplar al algoritmo de la 
implementación del contenedor de restricciones, mediante la definición de un 
\mintinline{coq}{Module Type} que especifica las operaciones necesarias del contenedor y sus 
invariantes. De esta manera se podría desarrollar la teoría del comprobador de implicaciones
independientemente de la implementación del contenedor.

Por último, como se mencionó en el Capítulo \ref{cap:definitions}, el invariante \mintinline{coq}{imp_checker} 
solo especifica que cuando la función comprobadora devuelve \mintinline{coq}{true} entonces la implicación
se cumple. Sin embargo, sería útil tener el recíproco: si la implicación se cumple, entonces sabemos que
la función comprobadora devuelve \mintinline{coq}{true}. Esta propiedad podría usarse para desarrollar un 
comprobador de satisfabilidad, por ejemplo.

Hay que tener en cuenta que nuestro algoritmo construye restricciones octogonales aditivas
de la forma $x - x \le d$. Estas restricciones son ciertas o no dependiendo del valor de
la constate $d$ e independientemente del valor de la variable $x$. De hecho, \cite{HarveyStuckey}
prueba que derivar una restricción de la forma $x - x \le d$ con $d$ negativo es condición
suficiente y necesaria para concluir que las restricciones originales son insatisfacibles,
es decir, no son satisfacibles para ningún modelo.

Si tuviéramos el resultado de completitud, podríamos implementar un comprobador de satisfabilidad
comprobando si la restricción $x - x \le -1$ se puede derivar de las restricciones originales, para
cualquier variable $x$. De esta manera, si el comprobador es incapaz de decir que la implicación es
cierta, por la completitud podemos deducir que las restricciones son satisfacibles.

Aunque la completitud es una propiedad conveniente, esta nunca ha sido un requisito en el proyecto
\verb|FORVES2|. Gracias al proyecto SMTCoq\cite{EMT+17} podemos usar resolutores SMT externos para
probar que un conjunto de restricciones es satisfacible y verificar sus resultados. Es por esto
que en este trabajo optamos en concentrarnos en la correctitud antes que la completitud del algoritmo.
Aun así, es cierto que esta sería la extensión más natural de los conceptos tratados en este trabajo.

%
% Bibliografía
%
% Si el TFM se escribe en inglés, editar TeXiS/TeXiS_bib para cambiar el
% estilo de las referencias
%---------------------------------------------------------------------
%
%                      configBibliografia.tex
%
%---------------------------------------------------------------------
%
% bibliografia.tex
% Copyright 2009 Marco Antonio Gomez-Martin, Pedro Pablo Gomez-Martin
%
% This file belongs to the TeXiS manual, a LaTeX template for writting
% Thesis and other documents. The complete last TeXiS package can
% be obtained from http://gaia.fdi.ucm.es/projects/texis/
%
% Although the TeXiS template itself is distributed under the 
% conditions of the LaTeX Project Public License
% (http://www.latex-project.org/lppl.txt), the manual content
% uses the CC-BY-SA license that stays that you are free:
%
%    - to share & to copy, distribute and transmit the work
%    - to remix and to adapt the work
%
% under the following conditions:
%
%    - Attribution: you must attribute the work in the manner
%      specified by the author or licensor (but not in any way that
%      suggests that they endorse you or your use of the work).
%    - Share Alike: if you alter, transform, or build upon this
%      work, you may distribute the resulting work only under the
%      same, similar or a compatible license.
%
% The complete license is available in
% http://creativecommons.org/licenses/by-sa/3.0/legalcode
%
%---------------------------------------------------------------------
%
% Fichero  que  configura  los  parámetros  de  la  generación  de  la
% bibliografía.  Existen dos  parámetros configurables:  los ficheros
% .bib que se utilizan y la frase célebre que aparece justo antes de la
% primera referencia.
%
%---------------------------------------------------------------------


%%%%%%%%%%%%%%%%%%%%%%%%%%%%%%%%%%%%%%%%%%%%%%%%%%%%%%%%%%%%%%%%%%%%%%
% Definición de los ficheros .bib utilizados:
% \setBibFiles{<lista ficheros sin extension, separados por comas>}
% Nota:
% Es IMPORTANTE que los ficheros estén en la misma línea que
% el comando \setBibFiles. Si se desea utilizar varias líneas,
% terminarlas con una apertura de comentario.
%%%%%%%%%%%%%%%%%%%%%%%%%%%%%%%%%%%%%%%%%%%%%%%%%%%%%%%%%%%%%%%%%%%%%%
\setBibFiles{%
biblio%
}

%%%%%%%%%%%%%%%%%%%%%%%%%%%%%%%%%%%%%%%%%%%%%%%%%%%%%%%%%%%%%%%%%%%%%%
% Definición de la frase célebre para el capítulo de la
% bibliografía. Dentro normalmente se querrá hacer uso del entorno
% \begin{FraseCelebre}, que contendrá a su vez otros dos entornos,
% un \begin{Frase} y un \begin{Fuente}.
%
% Nota:
% Si no se quiere cita, se puede eliminar su definición (en la
% macro setCitaBibliografia{} ).
%%%%%%%%%%%%%%%%%%%%%%%%%%%%%%%%%%%%%%%%%%%%%%%%%%%%%%%%%%%%%%%%%%%%%%
% \setCitaBibliografia{
% \begin{FraseCelebre}
% \begin{Frase}
%   Y así, del mucho leer y del poco dormir, se le secó el celebro de
%   manera que vino a perder el juicio.\\ 
%   \textcolor{red}{(modificar en Cascaras$\backslash$bibliografia.tex)}
% \end{Frase}
% \begin{Fuente}
%   Miguel de Cervantes Saavedra
% \end{Fuente}
% \end{FraseCelebre}
% }

%%
%% Creamos la bibliografia
%%
\makeBib

% Variable local para emacs, para  que encuentre el fichero maestro de
% compilación y funcionen mejor algunas teclas rápidas de AucTeX

%%%
%%% Local Variables:
%%% mode: latex
%%% TeX-master: "../Tesis.tex"
%%% End:



% Apéndices
%\appendix
%\chapter{Título del Apéndice A}
\label{Appendix:Key1}

Los apéndices son secciones al final del documento en las que se agrega texto con el objetivo de ampliar los contenidos del documento principal.
%\chapter{Título del Apéndice B}
\label{Appendix:Key2}

Se pueden añadir los apéndices que se consideren oportunos.
%\include{Apendices/appendixC}
%\include{...}
%\include{...}
%\include{...}
\backmatter



%
% Índice de palabras
%

% Sólo  la   generamos  si  está   declarada  \generaindice.  Consulta
% TeXiS.sty para más información.

% En realidad, el soporte para la generación de índices de palabras
% en TeXiS no está documentada en el manual, porque no ha sido usada
% "en producción". Por tanto, el fichero que genera el índice
% *no* se incluye aquí (está comentado). Consulta la documentación
% en TeXiS_pream.tex para más información.
\ifx\generaindice\undefined
\else
%%---------------------------------------------------------------------
%
%                        TeXiS_indice.tex
%
%---------------------------------------------------------------------
%
% TeXiS_indice.tex
% Copyright 2009 Marco Antonio Gomez-Martin, Pedro Pablo Gomez-Martin
%
% This file belongs to TeXiS, a LaTeX template for writting
% Thesis and other documents. The complete last TeXiS package can
% be obtained from http://gaia.fdi.ucm.es/projects/texis/
%
% This work may be distributed and/or modified under the
% conditions of the LaTeX Project Public License, either version 1.3
% of this license or (at your option) any later version.
% The latest version of this license is in
%   http://www.latex-project.org/lppl.txt
% and version 1.3 or later is part of all distributions of LaTeX
% version 2005/12/01 or later.
%
% This work has the LPPL maintenance status `maintained'.
% 
% The Current Maintainers of this work are Marco Antonio Gomez-Martin
% and Pedro Pablo Gomez-Martin
%
%---------------------------------------------------------------------
%
% Contiene  los  comandos  para  generar  el índice  de  palabras  del
% documento.
%
%---------------------------------------------------------------------
%
% NOTA IMPORTANTE: el  soporte en TeXiS para el  índice de palabras es
% embrionario, y  de hecho  ni siquiera se  describe en el  manual. Se
% proporciona  una infraestructura  básica (sin  terminar)  para ello,
% pero  no ha  sido usada  "en producción".  De hecho,  a pesar  de la
% existencia de  este fichero, *no* se incluye  en Tesis.tex. Consulta
% la documentación en TeXiS_pream.tex para más información.
%
%---------------------------------------------------------------------


% Si se  va a generar  la tabla de  contenidos (el índice  habitual) y
% también vamos a  generar el índice de palabras  (ambas decisiones se
% toman en  función de  la definición  o no de  un par  de constantes,
% puedes consultar modo.tex para más información), entonces metemos en
% la tabla de contenidos una  entrada para marcar la página donde está
% el índice de palabras.

\ifx\generatoc\undefined
\else
   \addcontentsline{toc}{chapter}{\indexname}
\fi


% Generamos el índice
\printindex

% Variable local para emacs, para  que encuentre el fichero maestro de
% compilación y funcionen mejor algunas teclas rápidas de AucTeX

%%%
%%% Local Variables:
%%% mode: latex
%%% TeX-master: "./tesis.tex"
%%% End:

\fi

%
% Lista de acrónimos
%

% Sólo  lo  generamos  si  está declarada  \generaacronimos.  Consulta
% TeXiS.sty para más información.


\ifx\generaacronimos\undefined
\else
%---------------------------------------------------------------------
%
%                        TeXiS_acron.tex
%
%---------------------------------------------------------------------
%
% TeXiS_acron.tex
% Copyright 2009 Marco Antonio Gomez-Martin, Pedro Pablo Gomez-Martin
%
% This file belongs to TeXiS, a LaTeX template for writting
% Thesis and other documents. The complete last TeXiS package can
% be obtained from http://gaia.fdi.ucm.es/projects/texis/
%
% This work may be distributed and/or modified under the
% conditions of the LaTeX Project Public License, either version 1.3
% of this license or (at your option) any later version.
% The latest version of this license is in
%   http://www.latex-project.org/lppl.txt
% and version 1.3 or later is part of all distributions of LaTeX
% version 2005/12/01 or later.
%
% This work has the LPPL maintenance status `maintained'.
% 
% The Current Maintainers of this work are Marco Antonio Gomez-Martin
% and Pedro Pablo Gomez-Martin
%
%---------------------------------------------------------------------
%
% Contiene  los  comandos  para  generar  el listado de acrónimos
% documento.
%
%---------------------------------------------------------------------
%
% NOTA IMPORTANTE:  para que la  generación de acrónimos  funcione, al
% menos  debe  existir  un  acrónimo   en  el  documento.  Si  no,  la
% compilación  del   fichero  LaTeX  falla  con   un  error  "extraño"
% (indicando  que  quizá  falte  un \item).   Consulta  el  comentario
% referente al paquete glosstex en TeXiS_pream.tex.
%
%---------------------------------------------------------------------


% Redefinimos a español  el título de la lista  de acrónimos (Babel no
% lo hace por nosotros esta vez)

\def\listacronymname{Lista de acrónimos}

% Para el glosario:
% \def\glosarryname{Glosario}

% Si se  va a generar  la tabla de  contenidos (el índice  habitual) y
% también vamos a  generar la lista de acrónimos  (ambas decisiones se
% toman en  función de  la definición  o no de  un par  de constantes,
% puedes consultar config.tex  para más información), entonces metemos
% en la  tabla de contenidos una  entrada para marcar  la página donde
% está el índice de palabras.

\ifx\generatoc\undefined
\else
   \addcontentsline{toc}{chapter}{\listacronymname}
\fi


% Generamos la lista de acrónimos (en realidad el índice asociado a la
% lista "acr" de GlossTeX)

\printglosstex(acr)

% Variable local para emacs, para  que encuentre el fichero maestro de
% compilación y funcionen mejor algunas teclas rápidas de AucTeX

%%%
%%% Local Variables:
%%% mode: latex
%%% TeX-master: "../Tesis.tex"
%%% End:

\fi

%
% Final
%
%%---------------------------------------------------------------------
%
%                      fin.tex
%
%---------------------------------------------------------------------
%
% fin.tex
% Copyright 2009 Marco Antonio Gomez-Martin, Pedro Pablo Gomez-Martin
%
% This file belongs to the TeXiS manual, a LaTeX template for writting
% Thesis and other documents. The complete last TeXiS package can
% be obtained from http://gaia.fdi.ucm.es/projects/texis/
%
% Although the TeXiS template itself is distributed under the 
% conditions of the LaTeX Project Public License
% (http://www.latex-project.org/lppl.txt), the manual content
% uses the CC-BY-SA license that stays that you are free:
%
%    - to share & to copy, distribute and transmit the work
%    - to remix and to adapt the work
%
% under the following conditions:
%
%    - Attribution: you must attribute the work in the manner
%      specified by the author or licensor (but not in any way that
%      suggests that they endorse you or your use of the work).
%    - Share Alike: if you alter, transform, or build upon this
%      work, you may distribute the resulting work only under the
%      same, similar or a compatible license.
%
% The complete license is available in
% http://creativecommons.org/licenses/by-sa/3.0/legalcode
%
%---------------------------------------------------------------------
%
% Contiene la última página
%
%---------------------------------------------------------------------


% Ponemos el marcador en el PDF
\ifpdf
   \pdfbookmark{Fin}{fin}
\fi

\thispagestyle{empty}\mbox{}

Este texto se puede encontrar en el fichero Cascaras/fin.tex. Si deseas eliminarlo, basta con comentar la línea correspondiente al final del fichero TFGTeXiS.tex.

\vspace*{4cm}

\small

\hfill \emph{--¿Qué te parece desto, Sancho? -- Dijo Don Quijote --}

\hfill \emph{Bien podrán los encantadores quitarme la ventura,}

\hfill \emph{pero el esfuerzo y el ánimo, será imposible.}

\hfill 

\hfill \emph{Segunda parte del Ingenioso Caballero} 

\hfill \emph{Don Quijote de la Mancha}

\hfill \emph{Miguel de Cervantes}

\vfill%space*{4cm}

\hfill \emph{--Buena está -- dijo Sancho --; fírmela vuestra merced.}

\hfill \emph{--No es menester firmarla -- dijo Don Quijote--,}

\hfill \emph{sino solamente poner mi rúbrica.}

\hfill 

\hfill \emph{Primera parte del Ingenioso Caballero} 

\hfill \emph{Don Quijote de la Mancha}

\hfill \emph{Miguel de Cervantes}


\newpage
\thispagestyle{empty}\mbox{}

\newpage

% Variable local para emacs, para  que encuentre el fichero maestro de
% compilación y funcionen mejor algunas teclas rápidas de AucTeX

%%%
%%% Local Variables:
%%% mode: latex
%%% TeX-master: "../Tesis.tex"
%%% End:

\end{otherlanguage}
\end{document}
