\chapter{Conclusions and Future Work}
\label{cap:conclusions}

In Chapter \ref{cap:imp-checker} we implemented an algorithm which, when given a list of octagonal
hypothesis constraints and an octagonal thesis constraint, will determine whether
the thesis constraint can be obtained from the hypothesis. If this algorithm
determines that it is possible, we have also formally verified that the implication
holds true, and can therefore be relied on by different parts of the project.

We can now employ our algorithm to solve the problem introduced in Chapter
\ref{cap:state-of-the-art}. The Listings \ref{lst:evm-original} and \ref{lst:evm-optimized}
were shown to both leave the stack in the same configuration, but they could
not be considered semantically equivalent unless some conditions were met. In
particular, these conditions were:
\begin{align}
    \label{constr:T1} x_4 &= 0 \\
    \label{constr:T2} x_0 &\ge x_2 + 32
\end{align}

In Section \ref{sect:optimization-smart-contracts} we proved that \ref{constr:T1} and \ref{constr:T2} could be derived under the following constraints.

\begin{align*}
    x_4 &= x_5 \\
    x_5 &= 0 \\
    x_0 &\ge x_2 + 128
\end{align*}

Now, however, we are able to use the implication checker implemented in Chapter
\ref{cap:imp-checker} to test this. First, we need to transform our hypothesis
constraints into octagonal constraints. To do so, we follow the instructions
described in Chapter \ref{cap:definitions}.

\begin{align*}
    x_4 = x_5 &\Longrightarrow x_4 - _5 \le 0 \ \wedge\  x_5 - x_4 \le 0
    \\
    x_5 = 0 &\Longrightarrow x_5 \le 0 \ \wedge\ -x_5 \le 0
    \\
    x_0 \ge x_2 + 128 &\Longrightarrow x_2 - x_0 \le -128
\end{align*}

To ease the definition of multiple constraints, we make use of the following
convenience functions.

\begin{minted}{coq}
Program Definition mkadd_pp(x: nat)(y: nat)(d: Z) :=
  AddConstr (Build_term 1 x) (Build_term 1 y) d.
Program Definition mkadd_pn(x: nat)(y: nat)(d: Z) :=
  AddConstr (Build_term 1 x) (Build_term (-1) y) d.
Program Definition mkadd_nn(x: nat)(y: nat)(d: Z) :=
  AddConstr (Build_term (-1) x) (Build_term (-1) y) d.
Program Definition mkbnd_p(x: nat)(d: Z) :=
  BndConstr (Build_term 1 x) d.
Program Definition mkbnd_n(x: nat)(d: Z) :=
  BndConstr (Build_term (-1) x) d.
\end{minted}

With these functions, we define the list of constraints which represent the
context of our application.

\begin{minted}{coq}
Local Definition C := 
  (* Obtained from [x4 = x5] *)
  [ Octagon.mkadd_pn 4 5 0   (*  x4 - x5 <= 0 *)
  ; Octagon.mkadd_pn 5 4 0   (*  x5 - x4 <= 0 *)
  (* Obtained from [x5 = 0] *)
  ; Octagon.mkbnd_p 5 0      (*  x5 <= 0*)
  ; Octagon.mkbnd_n 5 0      (* -x5 <= 0*)
  (* Obtained from [x0 >= x2 + 128] *)
  ; Octagon.mkadd_pn 2 0 (-128) (*  x2 - x0 <= -128 *)
].
\end{minted}

This transformation we made for our hypothesis also needs to be made for our
thesis. In particular, \ref{constr:T1} transforms into the constraints $x_4 \le 0$
and $-x_4 \le 0$, which means we will need to check that both of them hold to
conclude that it holds for \ref{constr:T1}.

\begin{minted}{coq}
Local Definition checker: Constraint -> bool := 
    conj_imp_checker_fun (conj_trans_closure_checker (2 * length C)) C.

(* x4 = 0 *)
Compute checker (Octagon.mkbnd_p 4 0) && checker (Octagon.mkbnd_n 4 0).
(* x0 >= x2 + 32 *)
Compute checker (Octagon.mkadd_pn 2 0 (-32)).
\end{minted}
\vspace{-\baselineskip*3/2}
\begin{minted}[bgcolor=outputbg]{coq}
checker is defined

= true
     : bool

= true
     : bool
\end{minted}

\section{Future work}
\label{sect:future-work}

The original objective of this project was to develop an implication checker in Coq
and verify its soundness. The objective of the project has therefore been achieved,
however there is work to be done before it can be used in \verb|FORVES2| directly.
While the algorithm is implemented and proven to be correct, before it can be used
we need to connect it to the definitions found in \mintinline{coq}{constraints.v},
those defined in Chapter \ref{cap:definitions}.

The first definition to bridge would be those representing state, \mintinline{coq}{Octagon.model} 
and \mintinline{coq}{Constraints.assignment}. Both represent functions which transform naturals
to values, but \mintinline{coq}{Octagon.model} gives integers (\mintinline{coq}{Z}) while
\mintinline{coq}{Constraints.assignment} gives naturals (\mintinline{coq}{N}). The first
step would be to show that \mintinline{coq}{Octagon.model} contains 
\mintinline{coq}{Constraints.assignment}, and to provide an explicit injection.
\begin{minted}{coq}
Definition translate_model:
    Constraint.assigment -> Octagon.model.
\end{minted}

Afterwards, a transform of \mintinline{coq}{Constraints.constraint}s to 
\mintinline{coq}{Octagon.Constraint}s using the rules outlined in Chapter 
\ref{cap:definitions} would need to be defined. We can assume this transformation's
signature to be something similar to the following.
\begin{minted}{coq}
Definition translate_constraints: 
    Constraint.conjunction -> list Octagon.Constraint.
\end{minted}

\noindent Then, we would need to show that our \mintinline{coq}{translate_constraints} function
preserves the information contained in the constraints. That is, some model satisfies some 
constraints if and only if the translated model also satisfies the translated constraints.

\begin{minted}{coq}
Lemma translate_preserves_information(C: list Constraint.constraint) :
  forall (m: Constraint.assigment),
    let m_oct = translate_model m in
    let C_oct = translate_constraints C in
      Constraint.satisfies_conjunction m C = true <->
      Octagon.satisfies_constraints m_oct C_oct = true.
\end{minted}

Armed with these results we could define a \mintinline{coq}{Constraint.conj_imp_checker}
which would work by translating the constraints from the \mintinline{coq}{Constraint} module to
the \mintinline{coq}{Octagon}'s one and asking whether these imply the provided thesis constraints
using \mintinline{coq}{Octagon.conj_trans_closure_checker}. Notice that, since we have not defined
a translation between single constraints, our initial single thesis constraint may now correspond
to multiple ones. This can be solved by calling the checking function as many times as thesis 
constraints need to be proven, and ensuring that all of them hold. Using the 
\mintinline{coq}{translate_preserves_information} lemma we would then show that the soundness of 
\mintinline{coq}{Octagon.conj_trans_closure_checker} implies that of this checker.

Beyond the connection between the theories described in \mintinline{coq}{constraints.v} and 
\mintinline{coq}{octagon.v}, there  is also the possibility of improving the runtime of the
transitive closure algorithm. For instance, \cite{TransitiveClosure} mentions an algorithm which
uses multiple matrices to represent the collection of constraints. Beyond improving the algorithm's
efficiency, this representation would simplify greatly the implementation of the 
\mintinline{coq}{join} function.

One could also approach this from the perspective of generalizing our constraints container from
the algorithm. A \mintinline{coq}{Module Type} could be defined which implements the required 
operations and guarantees the necessary properties, and then the particular container could be
developed independently.

Finally, as mentioned in Chapter \ref{cap:definitions}, the \mintinline{coq}{imp_checker} invariant
only requires that if the checking function returns \mintinline{coq}{true} then the implication
holds. However, it would be useful to have the reciprocal: if the implication holds, then the checking
function returns \mintinline{coq}{true}. This could be used to develop a satisfiability checker.

Notice that our algorithm also constructs additive octagonal bounds of the form $x - x \le d$.
These constraints will hold as long as $d$ is a non-negative integer, regardless of the assignment
of $x$. In fact, \cite{HarveyStuckey} proves that deriving a constraint of the form $x - x \le d$
for a negative $d$ is sufficient and necessary to prove that the original constraints were 
unsatisfiable, that is, they are not satisfiable under any model.

If we had the completeness result, we could implement a satisfiability checker which checks 
whether the constraint $x - x \le -1$ is implied by the original constraints, for any variable $x$.
Then, if the checker is unable to say that the implication holds, through completeness we would
be able to assert that the constraints are satisfiable.

While completeness is a nice property to have, it has never been a requirement of the \verb|FORVES2|
project. Thanks to the SMTCoq project\cite{EMT+17} we can rely on external SMT solvers to prove
that a set of restrictions is satisfiable and verify their results. That is why in this project
we opted to focus in soundness over completeness. Nevertheless, it would be the most natural 
extension of the concepts discussed in this project.