\chapter{State of the art}
\label{cap:state-of-the-art}

A block-chain implements a distributed ledger, a database of transactions
between entities which is not hosted by any central authority but shared
among multiple computers in a network. The name comes from the mechanism
in which these transactions are stored: groups of transactions called
``blocks'' are linked together forming a ``chain''. Once a block of 
transactions has been recorded in the block-chain, no further changes can
be applied to it without needing to change all subsequent blocks, which in
real world scenarios is made to be unfeasible, thus ensuring the immutability
and transparency of the data.

Blockchain technology has diverse applications, the most famous one being
as a form of currency. One of these use cases of particular interest is 
implementing smart contracts. Smart contracts are self-executing contracts
whose terms are directly expressed as code. They run on a blockchain 
network and automatically execute actions when predefined conditions are
met, without the need to rely on a trusted intermediary since their
execution is verified by the decentralized network. The usage of smart
contracts has enabled the development of new and innovative decentralized
applications (DApps).

Ethereum is a prominent blockchain platform which stands out for its robust
support for smart contracts through the Ethereum Virtual Machine (EVM). The
EVM is a runtime environment that enables the execution of code written in
Ethereum's native language, Solidity. Solidity code is compiled into EVM 
bytecode, which is composed of instructions operating on a stack-based 
virtual machine. Each operation performed in the EVM incurs a cost measured 
in gas, Ethereum's unit of computational cost. This cost is covered by the 
user interacting with the smart contract and is payed in the Ethereum 
blockchain's native currency, of the same name. This serves as a mechanism 
to prioritize and meter the execution of code in the network, which ensures 
that the network remains efficient and secure, by encouraging developers to 
be mindful of their use of resources. This contributes to preventing spam 
attacks and resource abuse, ensuring the stability and sustainability of the 
Ethereum ecosystem.

The GASOL project implements a superoptimizer for EVM bytecode. Superoptimization 
is a compilation technique that searches, for a given jump-free sequence of 
instructions, a semantically equivalent sequence of instructions which is optimal
by some metric, like memory usage or execution cost. Since the superoptimizer 
requires these sequences of instructions to not have bifurcations, to optimize a
whole program it first extracts all the sequences of instructions which do not perform
jumps and optimizes those separately before reassembling them back together. 
In doing so, it remembers which conditions triggered those jumps so it can gain more
information on which states of the program are possible for each section.

The Coq proof assistant has been used extensively to verify projects in the industry.
\begin{itemize}
    \item \verb|CompCert| \cite{CompCert}, a certified compiler for the majority of the C language.
    \item \verb|Fiat-Crypto| \cite{FiatCryptoSP19}, a tool for cryptographic primitive code generation.
    \item \verb|ConCert| \cite{ConCert}, a framework for smart contract verification in Coq.
    \item \verb|Iris| \cite{Iris}, a higher-order concurrent separation logic framework, used for reasoning about safety of concurrent programs, as the logic in logical relations, to reason about type-systems, data-abstraction, \dots. It has been used in other projects such as RustBelt \cite{RustBelt}.
    \item \verb|Cosette| \cite{Cosette}, an automated prover for checking equivalences of SQL queries.
\end{itemize}

%% TODO: Describe what are octogonal constraints and available algorithms
The constraints we are interested on can be reduced to a special type of constraint called
octagonal constraints. Octagonal constraints are constraints of the type ${\pm x \pm y \le d}$ 
or $\pm x \le d$ where $x$ and $y$ are integer variables. In general we assume that $d$ is an
integer, since otherwise we could exploit the fact that the left hand side of both types of
constraints result in integers to derive a more strict inequality using $\lfloor d \rfloor$
instead. This new constraint $\pm x \pm y \le \lfloor d \rfloor$ is called the \emph{tightened}
constraint of $\pm x \pm y \le d$.

Octagonal constraints have been studied in depth in the literature. In particular, efficient
algorithms have been derived to compute the transitive closure of a set of constraints. The
transitive closure of a set of constraints $C$ is a set of constraints $C'$ where for every
pair of variables $x$ and $y$ defined in the constraints of $C$ we have a $d$ such that
the constraint $x + y \le d$ is in $C'$ and a $d'$ such that $x \le d'$ is in $C'$. Furthermore,
the choices of $d$ and $d'$ are the smallest possible such that they can still be derived from
the constraints in $C$.

In \cite{TransitiveClosure} an $O(n^3\log{n})$ algorithm is described to compute the tightened 
transitive closure of a set of octagonal constraints where $n$ is the number of integer variables. The 
same article cites \cite{HarveyStuckey} for providing an $O(n^4)$ algorithm which relies on the 
application of successive rules to iteratively construct the transitive closure.