\chapter*{Introducción}
\label{cap:introduccion}
\addcontentsline{toc}{chapter}{Introducción}

\chapterquote
{Ethereum, taken as a whole, can be viewed as a transaction-based state machine: we begin with a genesis state and incrementally execute transactions to morph it into some current state. It is this current state which we accept as the canonical “version” of the world of Ethereum.}
{Ethereum Yellow Paper \cite{wood2014ethereum}}

\section{Motivación}

En el campo de optimización de código, el \emph{bytecode} de la máquina virtual 
de Ethereum (EVM) destaca como un lenguaje con mucho de lo que beneficiarse
de la disciplina, debido a requisitos particulares del lenguaje que abren
nuevas posibilidades para su optimización. En lenguajes de programación de
propósito general el objetivo de las optimizaciones suele ser el tiempo de
ejecución o el uso de memoria. Sin embargo, los programas compilados a \emph{bytecode}
de la EVM, conocidos como contratos inteligentes, uno debe tener en consideración
nuevos factores como el tamaño del ejecutable compilado o el coste monetario
de su ejecución. Toda instrucción de la EVM tiene un precio, conocido como
\emph{gas}\footnote{Ver \url{https://www.evm.codes/}}, que debe ser pagado 
para ejecutarla, y estos precios pueden variar en gran medida según la instrucción.

Aún así, la optimización de \emph{bytecode} de la EVM no está tan extendido como
uno imaginaría inicialmente. Esto se debe en parte a los grandes riesgos bajo los
que los programas de la EVM suelen trabajar. Un error en un contrato inteligente
suele tener un mayor impacto que en otros programas, a menudo resultando en grandes
perdidas monetarias\footnote{Ver \url{https://www.gemini.com/cryptopedia/the-dao-hack-makerdao}}.
Por lo tanto es comprensible que los desarrolladores de contratos inteligentes no 
quieran responsabilizarse de los posibles cambios en la semántica de su programa que
una herramienta de terceros pueda introducir.

Para mitigar el riesgo, el proyecto \verb|FORVES|\footnote{FORmaly Verified EVM optimizationS 
\url{https://github.com/costa-group/forves2/}} pretende desarrollar un verificador
capaz de garantizar que una optimización de una secuencia de instrucciones de la EVM 
sin saltos conserva el mismo significado que su versión sin optimizar. \verb|FORVES|
consigue esto calculando el estado resultante de la ejecución de ambas secuencias y
simplificándolas para probar su equivalencia. Sin embargo, esto no es suficiente
para aplacar las preocupaciones de los desarrolladores de contratos inteligentes,
puesto que simplemente hemos trasladado la carga de la prueba del optimizador al
verificador. Es decir, si el verificador tuviera un error, este podría garantizar que 
una optimización errónea es válida.

La solución es volver a trasladar la carga de la prueba a otra herramienta en la que
sí podamos confiar. Para el proyecto \verb|FORVES| el verificador se ha escrito en
Coq, un asistente de pruebas y lenguaje de programación que permite a sus usuarios
definir una especificación formal y probar que su programa la cumple. Los beneficios
de usar esta herramienta es que la carga de la prueba se concentra en un subconjunto
de funcionalidades que han sido cuidadosamente estudiadas por matemáticos y sobre las
que el resto de sus funcionalidades se basan para garantizar su correctitud. Es
decir, para confiar en que un programa certificado por Coq es correcto solo hace falta
confiar que el núcleo de Coq es correcto, el cual es mucho más fiable\footnote{
Hay incluso un proyecto para verificar Coq en Coq. \cite{sozeau:hal-04077552}
}. Muchos otros proyectos se han aprovechado de este hecho para desarrollar programas 
críticos en los que se pueda confiar\cite{...}. Una breve descripción de estos proyectos, 
en inglés, se puede encontrar en el Capítulo \ref{cap:state-of-the-art}.

El objetivo de este proyecto es ayudar en el desarrollo de \verb|FORVES2|, un sucesor
de \verb|FORVES| que mejora sus capacidades de razonamiento permitiéndole tener en
cuenta la información del contexto para determinar si dos secuencias de instrucciones
son equivalentes. Las secuencias de instrucciones libres de saltos de la EVM que 
\verb|FORVES| recibe vienen normalmente de un programa más grande, el cual cuenta con
información adicional sobre como se usa esta secuencia y que tipo de valores recibe.
En particular, para realizar ciertas optimizaciones es necesario poder deducir ciertas
restricciones de la información del contexto. Veamos un ejemplo.

\begin{center}
\begin{tikzpicture}[auto,
  node distance = 12mm and 20mm,
  start chain = going below,
  box/.style = {draw,rounded corners,blur shadow,fill=white,
        on chain,align=center}]
 \node[box] (b2)    {\verb|x0|\mintinline{haskell}{ = ...} \\ \verb|x1|\mintinline{haskell}{ = ...} \\ \verb|x0| - \verb|x1| $\ge$ 128?};      
 \node[box, draw=blue!80] (b3)    {\mintinline{haskell}{MEM[x0] = ...} \\ \mintinline{haskell}{MEM[x1] = ...}};  
 \node[box,right=of b2] (b4)    {\dots};     
 \begin{scope}[rounded corners,-latex]
 \path 
  (b2) edge node{\color{red}{\texttt{false}}} (b4)
  (b2) edge node{\mintinline{coq}{true}} (b3);
 \end{scope}
\end{tikzpicture}
\end{center}

El bloque azul solo se puede alcanzar si la condición de salto se cumple, lo cual a su
vez nos indica que \texttt{x0}$\ge$\texttt{x1}$+128$. Esta información puede ser útil
cuando se optimiza una parte del código, y por lo tanto se debe tener en consideración
si queremos certificar que dos secuencias de instrucciones son semánticamente 
equivalentes. Por ejemplo, sabiendo que \texttt{x0}$\ge$\texttt{x1}$+128$ es cierto
podemos deducir que las regiones de memoria a las que apuntan \texttt{x0} y \texttt{x1}
son disjuntas, y por lo tanto los accesos a memoria se pueden reordenar sin alterar
el resultado final.

La información del contexto de interés para \verb|FORVES2| son restricciones de la forma
$x + d \le y + d'$ donde $x$ e $y$ representan variables que apuntan a valores de la EVM,
que a su vez son un subconjunto de números enteros, y $d$ y $d''$ son constantes enteras.
Usando restricciones como la anterior también se pueden derivar otras restricciones como
$x \ge  y$, $x = d$ o $x < y + d'$ usando la simetría de la relación de orden, la 
combinación de múltiples restricciones o el hecho de que si $x \le k$ con $x$ y $k$ 
enteros entonces $x < k + 1$.

Por lo tanto, el objetivo de este trabajo es desarrollar un resolutor de restricciones que
sea capaz de probar que una restricción dada se cumple para un contexto específico. Para
ello transformamos las restricciones a un tipo específico, llamadas restricciones octogonales,
y usamos el algoritmo de la clausura transitiva para comprobar si otra restricción se puede
derivar de las anteriores.

\section{Objetivos}

Como se mencionó anteriormente, el objetivo principal de este trabajo es desarrollar un
resolutor de restricciones que permita usar información del contexto para aplicar ciertas
simplificaciones en el proyecto \verb|FORVES| que necesitan de ciertas condiciones previas
para ser usadas. Además, al igual que en el resto del proyecto, dicho resolutor de
restricciones debe estar verificado en Coq. Para ello elegimos los siguientes subobjetivos.

\begin{itemize}
    \item Estudiar Coq como herramienta para el desarrollo de \emph{software} verificado.
    \item Modelizar el problema en Coq.
    \item Implementar un comprobador de implicaciones que sea capaz de determinar si una
          restricción puede ser derivada de información de contexto.
    \item Verificar el comprobador de implicaciones en Coq.
\end{itemize}

\section{Plan de trabajo}

Cada objetivo a su vez se dividió en distintas tareas y cada una de estas tareas se distribuyo
en el periodo de tiempo asignado al proyecto.

\begin{enumerate}
    \item Estudiar Coq como herramienta para el desarrollo de \emph{software} verificado. \\
    \begin{enumerate}[label={1.{\arabic*}.}]
        \item\label{tarea:1.1} Leer \emph{Logical Foundations} \cite{Pierce2017Logical} hasta el
              Capítulo 6 - \emph{Logic}.
        \item\label{tarea:1.2} Terminar \emph{Logical Foundations} \cite{Pierce2017Logical}.
    \end{enumerate}

    \item Modelizar el problema en Coq.
    \begin{enumerate}[label={2.{\arabic*}.}]
        \item\label{tarea:2.1} Definir la estructura de las restricciones relevantes.
        \item\label{tarea:2.2} Definir la noción de implicación de restricciones.
        \item\label{tarea:2.3} Definir qué es un comprobador de implicaciones.
    \end{enumerate}

    \item Implementar un comprobador de implicaciones que sea capaz de determinar si una
          restricción puede ser derivada de información de contexto.
    \begin{enumerate}[label={3.{\arabic*}.}]
        \item\label{tarea:3.1} Implementar una versión básica de un comprobador de implicaciones,
             para asegurarse de la definición es la deseada.
        \item\label{tarea:3.2} Implementar una versión más potente del comprobador de implicaciones
             usando un resolutor de restricciones.
    \end{enumerate}

    \item Verificar el comprobador de implicaciones en Coq.
    \begin{enumerate}[label={4.{\arabic*}.}]
        \item\label{tarea:4.1} Desarrollar la teoría para verificar el comprobador de implicaciones básico.
        \item\label{tarea:4.2} Desarrollar la teoría para verificar el comprobador de implicaciones definitivo.
    \end{enumerate}
\end{enumerate}

El periodo asignado para el trabajo fueron los meses de enero a mayo de 2024. Los dos primeros
meses se reservaron para familiarizarse con las herramientas a utilizar, como Coq o \verb|FORVES2|.
El mes siguiente se dedicó a modelizar el problema en Coq, decidiendo y estabilizando la versión
final de las definiciones que se iban a usar e implementando la versión básica del comprobador de
implicaciones para ver como estas definiciones interactuarían con el resto del proyecto. Durante
este mes se consideraron varios algoritmos para implementar el resolutor de restricciones. La
implementación del comprobador de restricciones definitivo se dejó para mayo, cuando se implementó
el algoritmo en Coq y se desarrolló la teoría necesaria para verificarlo. Esta teoría se acabó
implementando de forma independiente al resto de definiciones del proyecto, y pese a que 
originalmente consideramos incluir la integración de esta teoría con el resto del proyecto como 
una subtarea adicional, rápidamente nos dimos cuenta de que este esfuerzo estaba fuera del ámbito
del trabajo. Para más información ver la Sección \ref{sect:trabajo-futuro}.

El ciclo de desarrollo final se puede visualizar como diagrama de Gantt en la Figura \ref{diagrama-gantt}.

\begin{figure}
\setganttlinklabel{s-s}{START-TO-START}
\setganttlinklabel{f-s}{FINISH-TO-START}
\setganttlinklabel{f-f}{FINISH-TO-FINISH}
\begin{ganttchart}[
    canvas/.append style={fill=none, draw=black!5, line width=.75pt},
    hgrid style/.style={draw=black!5, line width=.75pt},
    vgrid={*1{draw=black!5, line width=.75pt}},
    title/.style={draw=none, fill=none},
    title label font=\bfseries\footnotesize,
    title label node/.append style={below=7pt},
    include title in canvas=false,
    bar label font=\mdseries\small\color{black!70},
    bar label node/.append style={left=2cm},
    bar/.append style={draw=none, fill=blue!50},
    bar incomplete/.append style={fill=barblue},
    bar progress label font=\mdseries\footnotesize\color{black!70},
    group incomplete/.append style={fill=groupblue},
    group left shift=0,
    group right shift=0,
    group height=.5,
    group peaks tip position=0,
    group label node/.append style={left=.6cm},
    group progress label font=\bfseries\small,
    link/.style={-latex, line width=1.5pt, linkred},
    link label font=\scriptsize\bfseries,
    link label node/.append style={below left=-2pt and 0pt}
  ]{1}{13}
  \gantttitle[
    title label node/.append style={below left=7pt and -10pt}
  ]{MESES:\quad Ene}{1}
  \gantttitlelist{"","","Feb","","","Mar","","","Abr","","","May"}{1} \\

  \ganttbar{Tarea \ref{tarea:1.1}}{1}{5} \\
  \ganttbar{Tarea \ref{tarea:1.2}}{6}{9} \\
  \ganttbar{Tarea \ref{tarea:2.1}}{6}{8} \\
  \ganttbar{Tarea \ref{tarea:2.1}}{6}{8} \\
  \ganttbar{Tarea \ref{tarea:2.1}}{6}{8} \\
  \ganttbar{Tarea \ref{tarea:3.1}}{7}{8} \\
  \ganttbar{Tarea \ref{tarea:4.1}}{8}{9} \\
  \ganttbar{Tarea \ref{tarea:3.2}}{10}{11} \\
  \ganttbar{Tarea \ref{tarea:4.2}}{11}{12} \\
  \ganttbar{Escribir la memoria}{13}{13}

  % \ganttlink[link type=s-s]{WBS1A}{WBS1B}
  % \ganttlink[link type=f-s]{WBS1B}{WBS1C}
  % \ganttlink[
  %   link type=f-f,
  %   link label node/.append style=left
  % ]{WBS1C}{WBS1D}
\end{ganttchart}
\caption{Diagrama de Gantt del trabajo.}
\label{diagrama-gantt}
\end{figure}

El trabajo se desarrolla en el repositorio de GitHub\footnote{
\url{https://github.com/costa-group/forves2}} del proyecto \verb|FORVES2|. Los archivos
relevantes al trabajo son 
\href{https://github.com/costa-group/forves2/blob/8ec2a66dd44e0c695668d04dbe08f68ef5b56fb4/constraints.v}{\texttt{constraints.v}}
y
\href{https://github.com/costa-group/forves2/blob/8ec2a66dd44e0c695668d04dbe08f68ef5b56fb4/octagon.v}{\texttt{octagon.v}},
donde el primero usa algunos términos definicios en otros archivos del proyecto.

