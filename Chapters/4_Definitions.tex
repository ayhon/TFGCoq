\chapter{Modelization of the problem in Coq}
\label{cap:definitions}

The purpose of \mintinline{coq}{FORVES} is to certify that two sequences of EVM instructions are 
semantically equivalent. To do so, we need to define what we mean by semantic equivalence.


\section{Semantic equivalence of programs}

We can represent a program as a function which transform state where by state we mean an assignment of 
variables to values. In generic programming languages state is usually visualized with a map or a
dictionary.

\begin{minted}{python}
state = {
    "x": 10,
    "y": 3,
}
\end{minted}

The reason we use maps to represent these objects is that they are memory efficient, since their 
cost in efficiency is proportional to the number of elements they assign. However, the canonical 
way in which we represent assignments in mathematics is through functions, and this is also the 
method preferred by Coq since it eases its use in proofs. Therefore, the state our program's 
transform can be though of as a function with the signature \mintinline{coq}{Variable -> Value}.

As we explained in Chapter \ref{sect:blockchain}, the Ethereum Virtual Machine is a stack-based
virtual machine with an added memory and storage models Since the Ethereum Virtual Machine is a 
low-level virtual machine, it works on the byte level, usually by grouping them in sequences of 
32 called words. These words cover all possible values our variables can take.

\begin{minted}{coq}
Definition EVMWord := word EVMWordSize. 
(* EVMWordSize = BytesInEVMWord * EVMByteSize
               =             32 * 8
               =               256             *)
\end{minted}

In a smart contract the values of variables live either in memory or in storage, while the stack is
instead reserved for temporary values. Therefore it is safe to assume that the values we are 
interested in will be stored in memory or storage. For the purposes of the model, whether a variable
is stored in memory or storage is not relevant, and so we make no such distinction. Since in the
execution of a program there is a finite number of variables, each can be represented by a natural
number. We can then represent the status of all variables in our program to be a function from 
\mintinline{coq}{nat} to \mintinline{coq}{EVMWord}, which is precisely how they are defined in 
\href{https://github.com/costa-group/forves2/blob/8ec2a66dd44e0c695668d04dbe08f68ef5b56fb4/constraints.v}
{\texttt{constraints.v}}.

\begin{minted}{coq}
Definition assignment : Type := nat -> EVMWord.
\end{minted}

EVM programs can then be represented as functions which take an initial state of type 
\mintinline{coq}{assignment}, transforms it, and returns a final state, also of type
\mintinline{coq}{assignment}. Since a program is just a function of type
\mintinline{coq}{assignment -> assignment} we can define semantic equality through the principle of
functional extensionality, which says that if for all possible inputs two functions output the same
values then the functions must be the same.

\begin{minted}{coq}
Axiom functional_extensionality : forall {X Y: Type}
                                    {f g : X → Y},
  (forall (x:X), f x = g x) → f = g.
\end{minted}

Functional extensionality is not native to Coq, since it cannot be proven directly. Therefore, if we
wanted to use it we would need to add it as an axiom. Fortunately, adding this axiom is known to keep
Coq's logic consistent, which means that we cannot prove a proposition simultaneously to be true and 
false. Since it is harmless, Coq provides it as a definition in its standard library, under 
\mintinline{coq}{Coq.Logic.FunctionalExtensionality}, although not in the form described before. Coq's functional
extensionality axiom is tweaked to also work with functions which return dependent types. Our 
\mintinline{coq}{functional_extensionality} proposition is then derived from this axiom.

\begin{minted}{coq}
Axiom functional_extensionality_dep : forall {A} {B : A -> Type},
  forall (f g : forall x : A, B x),
  (forall x, f x = g x) -> f = g.
\end{minted}

Unfortunately, proceeding with this approach is unfeasible since testing all possible assignments for
the complete program is too complex. Therefore, the approach followed by \verb|FORVES| is different.
Instead, it records the effects of the execution of both programs on a generic model of the state and
simplifies them to a canonical state to see if they are equivalent, by applying simplification rules.
Furthermore, \verb|FORVES| does not test the transformation performed by the whole program, but only of 
jump-free sequences of instructions. This had the added benefit of not having to worry about 
conditionals with the added cost that we can no longer make any assumptions about the stack. However 
this is easily fixed by extending our \mintinline{coq}{assignment} definition to also include 
variables representing elements in the stack, not only offsets in memory.

Furthermore, we may also not need to consider all states, since not all states are reachable at a 
particular point in the code. For instance, if a piece of code is only reachable through a
conditional statement which checks for an invariant $P$, we can assume that the invariant holds for the 
final assignment. Consider the following snippets of code.

\begin{minted}{c}
if(x > y){
    z = x-y;
}
\end{minted}
\begin{minted}{c}
if(x > y){
    z = abs(x-y);
}
\end{minted}

The result of executing the section inside the \mintinline{c}{if} statement in the first snippet is 
setting the variable \mintinline{coq}{z} to \mintinline{coq}{x-y}, while in the second snippet it is
set to \mintinline{coq}{abs(x-y)}. Both of these effects are not equivalent if we consider an initial
state of \mintinline{js}{{x:0, y:1}} since then ${x-y} = -1 \ne 1 = \lvert-1\rvert$. However,
the condition in the \mintinline{coq}{if} statement discards these types of states. We can represent
this by making the condition in the \mintinline{coq}{if} statement a requirement for the model to 
follow.

Therefore, for some optimizations we may want to ask for some special condition to be applied.
For example, in the previous example we may consider simplifying \mintinline{coq}{abs(z)} to
\mintinline{coq}{z} if we are able to show that \mintinline{coq}{z > 0}. Sometimes, these conditions
are not explicitly stated in the context, but can instead be derived from it. Consider an optimization
\mintinline{coq}{add_zero}, which says that if we have \mintinline{coq}{x0 + x1} and 
\mintinline{coq}{x0} is zero then this is equivalent to just having \mintinline{coq}{x1}. This 
optimization can only be applied if we have the condition \mintinline{coq}{x0 = 0}. Now, consider the
following snippet.

\begin{minted}{c}
if (x > 0) f();
else if (x < 0) g();
else {
    y = y + x;
}
\end{minted}

When we reach the \mintinline{coq}{else} branch the contextual information we have recorded is that 
$x \le 0$ and $x \ge 0$, but not $x = 0$. However, we know that having $x \le 0$ and $x \ge 0$ implies 
$x = 0$. Therefore, to apply the optimization \mintinline{coq}{add_zero} we need to be able to show 
that the current contextual information we have implies the conditions we need, not that it is present
explicitly.

\section{Representation of constraints}

How do we represent this in Coq? For starters, we need to define what type of contextual information
are we working with. Since the Ethereum Virtual Machine works at the byte level, all its comparisons
are arithmetic, which means the only conditions we are interested in are equalities and inequalities.

\begin{minted}{coq}
Inductive constraint : Type :=
  | C_LT (l r : cliteral) (* l <  r *)
  | C_EQ (l r : cliteral) (* l == r *)
  | C_LE (l r : cliteral).(* l <= r *)
\end{minted}

We call \mintinline{coq}{cliteral}s to the elements we compare. In general, we are interested in 
comparisons between variables, constants and variables offset by a constant. These are what our 
\mintinline{coq}{cliteral}s represent.

\begin{minted}{coq}
Inductive cliteral : Type :=
  | C_VAR (n : nat) (* x *)
  | C_VAL (n : N) (* c *)
  | C_VAR_DELTA (n: nat)(delta : N).(* x + c *)
\end{minted}

Armed with these definitions we can start to represent what we mean by contextual information. This 
information is obtained from the optimizer when it is performing the static analysis of the program, 
in particular from the branches in the code. After each conditional jump, if the jump was taken we 
can add the condition tested to the context, and if it was not taken we add the negation of the 
condition. If we have multiple nested conditions, these can be combined logically by using an 
\mintinline{coq}{and} connective. It could also happen that one same region of the code is reachable 
from different jump instructions, which themselves carry different contextual information. This is the 
case, for example, with function calls. In this case we can combine the contextual information through 
an \mintinline{coq}{or} connective. 

To ease the representation of contextual information, we represent these combined conditions in 
disjunctive normal form. In code, this is modeled as a list of lists, where the inner lists represent 
conjunctions and the outer list represents one big disjunction.

\begin{minted}{coq}
Notation conjunction := (list constraint).
Notation disjuntion := (list conjunction).
Definition constraints : Type := disjuntion.
\end{minted}

\section{Implication between constraints}

The type \mintinline{coq}{constraints} represents the context that is available at one point in the 
code. What we want to be able to do is check if some set of constraints imply a different one. But in 
order to define this we first need to define what it means that a set of constraints imply another.

In logic, implication is usually expressed by the symbol $\rightarrow$, although the symbol $\subset$ 
is sometimes also used for the same purpose. This last symbol sheds more light into how we might want 
to define implication in this case.

For any state, a condition \mintinline{coq}{c} may or may not hold. We say that a model satisfies a 
single constraint if after substituting the variables by the values assigned to them by the model the 
condition holds.

\begin{minted}{coq}
Definition satisfies_single_constraint 
    (model: assignment) (c: constraint) : bool :=
  let get_value := cliteral_to_nat model in 
  match c with
  | C_EQ l r => (get_value l =? get_value r)%N
  | C_LT l r => (get_value l <? get_value r)%N
  | C_LE l r => (get_value l <=? get_value r)%N
  end.
\end{minted}

We can extend this definition to conjunctions of constraints by requiring that the model satisfies all 
constraints at the same time.

\begin{minted}{coq}
Definition satisfies_conjunction 
    (model: assignment) (conj: conjunction): bool :=
  forallb (satisfies_single_constraint model) conj.
\end{minted}

Finally, we extend the definition to also hold for generic combinations of constraints in disjunctive 
normal form by specifying that at least one conjunction must hold for the whole formula to be true.

\begin{minted}{coq}
Definition satisfies_constraints 
    (model: assignment) (cs: constraints): bool :=
  forallb (satisfies_conjunction model) cs.
\end{minted}

Now, we can take inspiration from the $\subset$ notation of implication to define implication of 
constraints. We say that a set of constraints in disjunctive normal form \mintinline{coq}{cs} implies a constraints \mintinline{coq}{c} if,
for every model which satisfies \mintinline{coq}{cs}, it also satisfies \mintinline{coq}{c}. That is, if $State_{\texttt{cs}}$ is the set of
all states which satisfy $cs$ and $State_{\texttt{c}}$ is the set of all states which satisfy $c$, then
$State_{\texttt{cs}} \subset State_{\texttt{c}}$. In Coq, we would write

\begin{minted}{coq}
Definition imply(cs: constraints)(c: constraint) := 
  forall (m: assignment),
    satisfies_constraints m cs = true ->
    satisfies_single_constraint m c = true.
\end{minted}

A similar definition follows for conjunctions instead of terms in DNF.
\begin{minted}{coq}
Definition conj_imply(cs: conjunction)(c: constraint) := 
  forall (m: assignment),
    satisfies_conjunction m cs = true ->
    satisfies_single_constraint m c = true.
\end{minted}

\section{Implication checker}

We can finally define what we mean by an implication checker. In the end, an implication checker is 
just a function which takes some constraints as contextual information, which we will call the 
hypothesis, and a constraint we will call the thesis, and returns the boolean true if it is able to 
show that the hypothesis imply the thesis.

In a different programming language we would implement implication checkers as functions with the 
signature \mintinline{coq}{constaints -> constraint -> bool}, but in Coq we can go further and encode 
the implication checker's invariant as part of the type.

\begin{minted}{coq}
Record imp_checker: Type := 
  { imp_checker_fun: constraints -> constraint -> bool
  ; imp_checker_snd: forall (cs: constraints) (c: constraint),
      imp_checker_fun cs c = true -> imply cs c
  }.
\end{minted}

An implication checker is not just the function which checks for the implication, but also the proof of
that if the checking function says the implication is true, then it is a fact that the hypothesis 
imply the thesis.

To implement the implication checker, we need to handle the constraints in disjunctive normal form. 
However, we observe that in general we can consider the different conjunctions of constraints 
independently. The intuition behind it is that, if we need to prove that $A \vee B \rightarrow C$, 
then we need to prove $A \rightarrow C$ and $B \rightarrow C$ separately. This can actually be easily 
proven by using the identities $A \rightarrow B \equiv \neg A \vee B$ and the distributive property of 
$\wedge$ and $\vee$.

\begin{align*}
A \vee B \rightarrow C &\equiv \neg (A \vee B) \vee C \\
                       &\equiv \neg A \wedge \neg B \vee C \\
                       &\equiv (\neg A \vee C)\wedge (\neg B \vee C) \\
                       &\equiv (A \rightarrow C)\wedge (B \rightarrow C)
\end{align*}

% ~ (A \/ B) \/ C = ~A /\ ~B \/ C = (~ A \/ C) /\ (~ B \/ C)

In Coq we represent this by defining a different type of implication checker, one that only works with 
conjunctions, and showing that a conjunction implication checker can be used as a regular implication
checker and vice versa.

\begin{minted}{coq}
Record conj_imp_checker: Type := 
  { conj_imp_checker_fun: conjunction -> constraint -> bool
  ; conj_imp_checker_snd: forall (cs: conjunction) (c: constraint),
      conj_imp_checker_fun cs c = true -> conj_imply cs c
  }.
\end{minted}

That a \mintinline{coq}{imp_checker} can work as a \mintinline{coq}{conj_imp_checker} is obvious and 
not really useful, since a conjunction is trivially in disjunctive normal form, where there are no 
disjunctions. The interesting side of this equivalence is the converse, showing that given a 
\mintinline{coq}{conj_imp_checker} we can create an \mintinline{coq}{imp_checker}. We do this by 
defining the \mintinline{coq}{mk_imp_checker} function, of signature 
\mintinline{coq}{conj_imp_checker -> imp_checker}.

\begin{minted}{coq}
Program Definition mk_imp_checker 
    (checker: conj_imp_checker): imp_checker := {|
  imp_checker_fun (cs : constraints) c := 
    match cs with
    | [] => false
    | _ => 
        forallb (fun conj => conj_imp_checker_fun checker conj c) cs
    end
|}.
\end{minted}

Since we included the implication checkers invariant inside its type, Coq will not let us finish the definition
of this function until we have proven that this function preserves the invariant. That is why we need to use
the \mintinline{coq}{Program} prefix to this definition, to reassure Coq that we know that the definition is not finished 
with just the function, and that we will follow with the proof of the invariant.

As we saw before, given a conjunction implication checker we can define a more general implication checker
by running the conjunction implication checker on each conjunction in the hypothesis and making sure they 
all hold true. We now need to show that, given that the conjunction implication checker is sound, the
implication checker derived from it is sound as well. Let's look at the proof of this.

We use the \mintinline{coq}{Next Obligation} command to enter in proof mode after the \mintinline{coq}{Program} definition we used before.

\begin{minted}{coq}
Next Obligation.
  (* First we unfold the definition of imply and introduce the model
     in its definition *)
  unfold imply; intros model.
  (* Then we obtain the conjunction checker's checking function and 
     soundness proof from its constructor *)
  destruct checker as [checker checker_snd].
  (* Next Obligation automatically introduces as many terms as 
     possible. In particular, the goal it asks to prove is whether an 
     assignment called model satisfies the constraint c given that:
       H : the checker function we have derived says that cs imply c
     We will rename H to full_checker__cs_imp_c. Full checker is the
     checking function we have derived from the conjunction 
     implication checker's *)
  rename H into full_checker__cs_imp_c.
  (* We proceed over induction on c. The base case is not reachable,
     so we discriminate. We are only left with the inductive case. *)
  induction cs as [|c' cs' IHcs']; try discriminate.
  simpl in *.
  (* Since we are left in the case where cs = c' :: cs' and the full
     checker said that cs imply c, by definition that means that
     c' imply c and cs' imply c. Notice that:
        cs': list list constraint (DNF)
         c': list constraint (conjunction)
         c : constraint                           *)
  apply Bool.andb_true_iff in full_checker__cs_imp_c 
    as [checker__c'_imp_c checker__cs'_imp_c].
  (* Moreso, to show that the disjuntion c' \⁄ cs'' implies c we
     need to that the model satisfies c both when it satisfies c' 
     and it satisfies cs'. *)
  intros h; apply Bool.orb_true_iff in h as [c'_sat | cs'_sat].
  - (* If the model satisfies c' then since the conjugation checker
       said that c' implies c, we have our result.*)
    exact (checker_snd _ _ checker__c'_imp_c model c'_sat).
  - (* Otherwise, we apply the induction hypothesis on cs' *)
    unfold is_model in cs'_sat.
    (* We can assume that cs' is nonempty, otherwise the proof would
       have concluded earlier *)
    destruct cs' as [|c'' cs'']; try discriminate.
    exact (IHcs' checker__cs'_imp_c cs'_sat).
Qed.
\end{minted}

Thanks to this function, which we have now proven that correctly yields
implication checkers from conjunction implication checkers, serves to ease the 
development of implication checkers. To define a conjunction implication checker
we no longer need to deal with DNF formulas, just with conjunctions. We can later
derive the full implication checker from the conjunction one.

Before we conclude this chapter, we must highlight an important fact. We have only
required our implication checkers to be sound, but we have not said anything about
completeness. This means that if an implication checker returns false we cannot 
assume that the implication is in fact false. Proving completeness is quite a hard
endeavour, and was therefore considered out of scope for this project.

This means that there is some type of partial order between implication checkers, 
depending on if how close to completeness they are. We can implement some trivial
implication checkers which are pretty far away from completeness. 

One such example would be the implication checker that always returns false. But a
more interesting example is the implication checker which just checks whether the
thesis is in the hypothesis.

\begin{minted}{coq}
Program Definition inclusion_conj_imp_checker: conj_imp_checker := {| 
  conj_imp_checker_fun := fun cs c => existsb (eqc c) cs
|}.
(* Proof left as an exercise to the reader *)
Definition inclusion_imp_checker := 
    mk_imp_checker inclusion_conj_imp_checker.
\end{minted}

While we have defined our implication checkers to be best-effort, we of course would
like to have the best implication checker we could possible have. That is the 
purpose of the next chapter.