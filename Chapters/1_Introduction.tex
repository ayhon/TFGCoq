\chapter{Introduction}
\label{cap:introduction}

% \chapterquote{Frase célebre dicha por alguien inteligente}{Autor}
\chapterquote
{Ethereum, taken as a whole, can be viewed as a transaction-based state machine: we begin with a genesis state and incrementally execute transactions to morph it into some current state. It is this current state which we accept as the canonical “version” of the world of Ethereum.}
{Ethereum Yellow Paper \cite{wood2014ethereum}}

\section{Motivation}
\label{sect:motivation}

% A distinctive feature of Ethereum is that transactions are programs, smart contracts, and computing a state transition requires to run the contract code to compute the next state. This capability
% is provided by the Ethereum Virtual Machine (EVM) that can execute programs written in EVM
% bytecode.


%% Why EVM good for optimization %% 
In the field of code optimization, Ethereum Virtual Machine (EVM) bytecode stands as a language with 
much to benefit from, as it has some unique requirements which open up new dimensions for optimization.
For general purpose programming languages the optimization's focus are usually on execution time or 
memory usage. However, in the case of programs compiled to EVM bytecode, called \emph{smart contracts},
one must also take into consideration other factors, such as the size of the compiled binary or the cost
of the programs execution. Every EVM instruction requires a fee (called \emph{gas}\footnote{See 
\url{https://www.evm.codes/}}) to be paid for its execution, and these prices can vary greatly in 
magnitude.

% https://costa.fdi.ucm.es/papers/costa/AlbertGHR22.pdf

%% WHY EVM code optimization not widespread %%
However, the optimization of EVM bytecode is not as widespread as one could imagine. This is in part
due to the higher stakes that EVM programs tend to work under. A bug in a \emph{smart contract} usually
has a greater impact than most other programs, often resulting in large monetary loses\footnote{See 
\url{https://www.gemini.com/cryptopedia/the-dao-hack-makerdao}}. It is therefore understandable that 
\emph{smart contract} developers do not want to risk being responsible of the semantics of their code
being contaminated by a third party's tool.

%% FORVES verifier %%
To mitigate this risk, the \verb|FORVES|
\footnote{FORmaly Verified EVM optimizationS \url{https://github.com/costa-group/forves2/}}
project aims to develop a verifier which is able to guarantee that an optimization of a jump-free
sequence of EVM instructions (that is, without pieces of code that are conditionally executed)
retains the same semantics as the original unoptimized version. \verb|FORVES| works by computing the 
resulting state after the execution of both sequences and applying simplifying them to prove their
equivalence.
This, however, does not by itself suffice to appease the worries of the \emph{smart contract} 
developer, since we have merely pushed the burden of trust from the optimizer to the verifier. If the 
verifier were to have a bug, it could fail to recognize a valid optimization or, in the worst case, 
guarantee that an erroneous optimization is in fact valid, impacting its users.

%% WHY trust FORVES -> Coq %%
The solution is to, again, shift the burden of proof to another tool. For the \verb|FORVES| project the 
verifier has been written in Coq, a proof assistant and programming language which allows its users to 
define a formal specification and prove that their program adheres to it. The benefit of using a tool
such as Coq is that the burden of proof is placed on a small core subset of its features, which have
been deeply scrutinized by mathematicians, and out of which all the other functionalities are built.
Therefore, to trust that a Coq-certified program is correct is equivalent to trusting that the Coq  
kernel is correct, which is a much greater 
assurance\footnote{There are even efforts to verify Coq in Coq!\cite{sozeau:hal-04077552}}. 
Many different projects have taken advantage of this fact to develop pieces of critical software which
can be trusted upon \cite{ConCert,FiatCryptoSP19,CompCert,Iris,Cosette}. A brief description of these projects can be found in Chapter 
    \ref{cap:state-of-the-art}.

The purpose of this project is to aid in the development of \verb|FORVES2|, a successor of \verb|FORVES|
which improves on its reasoning power by allowing it to take into account contextual information to
decide if two pieces of code are equivalent. The jump-free sequences of instructions which \verb|FORVES|
takes as input are usually extracted from a complete program, which offers additional information about
how it is used and what kind of values it receives. In particular, to apply some optimizations it is
necessary to be able to derive certain constraints from this contextual information. Consider the 
following example.

\begin{center}
\begin{tikzpicture}[auto,
  node distance = 12mm and 20mm,
  start chain = going below,
  box/.style = {draw,rounded corners,blur shadow,fill=white,
        on chain,align=center}]
 \node[box] (b2)    {\verb|x0|\mintinline{haskell}{ = ...} \\ \verb|x1|\mintinline{haskell}{ = ...} \\ \verb|x0| - \verb|x1| $\ge$ 128?};      
 \node[box, draw=blue!80] (b3)    {\mintinline{haskell}{MEM[x0] = ...} \\ \mintinline{haskell}{MEM[x1] = ...}};  
 \node[box,right=of b2] (b4)    {\dots};     
 \begin{scope}[rounded corners,-latex]
 \path 
  (b2) edge node{\color{red}{\texttt{false}}} (b4)
  (b2) edge node{\mintinline{coq}{true}} (b3);
 \end{scope}
\end{tikzpicture}
\end{center}

The blue block is only reachable if the jump condition is met, which means that \verb|x0| $\ge$ 
\verb|x1| $+ 128$. This information can be useful when optimizing a piece of code, and must therefore
be taken into consideration when certifying that two sequences of instructions are semantically 
equivalent. For example, knowing that \verb|x0| $\ge$ \verb|x1| $+ 128$ could imply that the addresses
of memory pointed to by \verb|x0| and \verb|x1| are disjoint, and therefore that the memory accesses
can be reordered without altering the final result.

The contextual information which \verb|FORVES2| is interested in are constraints of the form 
$x + d \le y + d'$ where $x$ and $y$ represent variables which point to EVM values (which themselves
are a subset of the integers) and $d$ and $d'$ are integer constants. In particular from the previous
constraints we can also derive the constraints of the form $x \ge y$, $x = d$ or $x < y + d'$ by taking
advantage of the symmetry of the order relation, the combination of multiple constraints or the fact
that $x \le k$ for $x$ and $k$ integers implies that $x < k + 1$. 

Therefore, this project aims to develop a constraint solver which is able to prove that a given
constraint holds for a given context. To do so, we transform the constraints into octagonal constraints
and we employ the tightened transitive closure algorithm to finally check if a constraint can be derived
from it.

\section{Objectives}
\label{sect:objectives}

As previously stated, the main objective of this project is developing a constraint solver which would
allow the usage of contextual information to enable specific simplifications in the \verb|FORVES2| project
that require some conditions to hold true to be applied. Furthermore, in spirit with the rest of the
project, this constraint solver must be verified in Coq. To do so, we set the following sub-objectives:

\begin{enumerate}
    \item Study Coq as a tool for the development of verified software.
    \item Model the problem in Coq.
    \item Implement an implication checker which is able to determine whether a constraint can be derived from a set of contextual information.
    \item Verify the implication checker in Coq.
\end{enumerate}

\section{Work plan}
\label{sect:work-plan}

Each objective was further subdivided in various tasks. These tasks where then distributed between the aloted
time frame for the project.

\begin{enumerate}
    \item Study Coq as a tool for the development of verified software. \\
    \begin{enumerate}[label={1.{\arabic*}.}]
        \item\label{task:1.1} Read Logical Foundations \cite{Pierce2017Logical} up to Chapter 6 - Logic.
        \item\label{task:1.2} Complete Logical Foundations \cite{Pierce2017Logical}.
    \end{enumerate}
    
    \item Model the problem in Coq.
    \begin{enumerate}[label={2.{\arabic*}.}]
        \item\label{task:2.1} Define the structure of the desired constraints.
        \item\label{task:2.2} Define the correct notion of equivalence.
        \item\label{task:2.3} Define what is an implementation checker.
    \end{enumerate}
    
    \item Implement an implication checker which is able to determine whether a constraint can be derived from a set of contextual information.
    \begin{enumerate}[label={3.{\arabic*}.}]
        \item\label{task:3.1} Implement a basic version of an implication checker, to ensure the model works as desired.
        \item\label{task:3.2} Implement a more powerful implication checker algorithm through a constraint solver.
    \end{enumerate}
    
    \item Verify the implication checker in Coq.
    \begin{enumerate}[label={4.{\arabic*}.}]
        \item\label{task:4.1} Develop the theory to verify the basic implication checker.
        \item\label{task:4.2} Develop the theory to verify the final implication checker.
    \end{enumerate}
\end{enumerate}

The development of the project was planned to take place between January and May of 2024.
The first two months were reserved to get familiar with the tools to be used, such as the
Coq proof assistant and \verb|FORVES2|. The following month was dedicated to model the
problem in Coq, by deciding and stabilizing the definitions that were going to be used and
implementing a basic implication checker to test how these definitions would fit in the
bigger scope of the project. During this month various algorithms were also considered to
implement the constraint solver. The implementation of the final implication checker was
left for may, by developing the chosen algorithm in Coq and deriving the theory necessary
to verify it. This theory ended up being developed independently of the definitions of the
project, and we considered linking this theory with the rest of the project as an 
additional subtask, but we quickly realized this effort was out of scope for the project.
See Chapter \ref{cap:conclusions} for more information.

The final development cycle can be visualized in the following Gant diagram.

\setganttlinklabel{s-s}{START-TO-START}
\setganttlinklabel{f-s}{FINISH-TO-START}
\setganttlinklabel{f-f}{FINISH-TO-FINISH}
\begin{ganttchart}[
    canvas/.append style={fill=none, draw=black!5, line width=.75pt},
    hgrid style/.style={draw=black!5, line width=.75pt},
    vgrid={*1{draw=black!5, line width=.75pt}},
    title/.style={draw=none, fill=none},
    title label font=\bfseries\footnotesize,
    title label node/.append style={below=7pt},
    include title in canvas=false,
    bar label font=\mdseries\small\color{black!70},
    bar label node/.append style={left=2cm},
    bar/.append style={draw=none, fill=blue!50},
    bar incomplete/.append style={fill=barblue},
    bar progress label font=\mdseries\footnotesize\color{black!70},
    group incomplete/.append style={fill=groupblue},
    group left shift=0,
    group right shift=0,
    group height=.5,
    group peaks tip position=0,
    group label node/.append style={left=.6cm},
    group progress label font=\bfseries\small,
    link/.style={-latex, line width=1.5pt, linkred},
    link label font=\scriptsize\bfseries,
    link label node/.append style={below left=-2pt and 0pt}
  ]{1}{13}
  \gantttitle[
    title label node/.append style={below left=7pt and -10pt}
  ]{MONTHS:\quad Jan}{1}
  \gantttitlelist{"","","Feb","","","Mar","","","Apr","","","May"}{1} \\

  \ganttbar{Task \ref{task:1.1}}{1}{5} \\
  \ganttbar{Task \ref{task:1.2}}{6}{9} \\
  \ganttbar{Task \ref{task:2.1}}{6}{8} \\
  \ganttbar{Task \ref{task:2.1}}{6}{8} \\
  \ganttbar{Task \ref{task:2.1}}{6}{8} \\
  \ganttbar{Task \ref{task:3.1}}{7}{8} \\
  \ganttbar{Task \ref{task:4.1}}{8}{9} \\
  \ganttbar{Task \ref{task:3.2}}{10}{11} \\
  \ganttbar{Task \ref{task:4.2}}{11}{12} \\
  \ganttbar{Writing the memory}{13}{13}

  % \ganttlink[link type=s-s]{WBS1A}{WBS1B}
  % \ganttlink[link type=f-s]{WBS1B}{WBS1C}
  % \ganttlink[
  %   link type=f-f,
  %   link label node/.append style=left
  % ]{WBS1C}{WBS1D}
\end{ganttchart}




The project is developed in \verb|FORVES2|'s GitHub repository\footnote{\url{https://github.com/costa-group/forves2}}. 
The files relevant to this project are \verb|constraints.v| and \verb|octagon.v|, with the former using some terms
defined in other files in the project.


