\chapter{Introduction}
\label{cap:introduction}

% \chapterquote{Frase célebre dicha por alguien inteligente}{Autor}
\chapterquote
{Ethereum, taken as a whole, can be viewed as a transaction-based state machine: we begin with a genesis state and incrementally execute transactions to morph it into some current state. It is this current state which we accept as the canonical “version” of the world of Ethereum.}
{Ethereum Yellow Paper}

\section{Motivation}
\label{sect:motivation}

% A distinctive feature of Ethereum is that transactions are programs, smart contracts, and computing a state transition requires to run the contract code to compute the next state. This capability
% is provided by the Ethereum Virtual Machine (EVM) that can execute programs written in EVM
% bytecode.


In the field of code optimization, EVM bytecode stands as a language with much to benefit from, as it
has some unique requirements which open up new dimensions for optimization. For general purpose
programming languages the optimization's focus are usually on execution time, memory usage or code size.
However, in the case of programs compiled to EVM bytecode, called \emph{smart contracts}, one must also
take into consideration other factors, such as the size of the compiled binary or the cost of the
programs execution. Every EVM instruction requires a fee (called \emph{gas}\footnote{See 
\url{https://www.evm.codes/}}) to be paid for its execution, and these prices can vary greatly in 
magnitude.

% https://costa.fdi.ucm.es/papers/costa/AlbertGHR22.pdf

However, the optimization of EVM bytecode is not as widespread as one could imagine. This is in part
due to the higher stakes that EVM programs, called \emph{smart contracts}, tend to work under. A bug
in a \emph{smart contract} tends to have a greater impact than most other programs, often resulting
in large monetary loses\footnote{See \url{https://www.gemini.com/cryptopedia/the-dao-hack-makerdao}}.
It's therefore understandable that \emph{smart contract} developers do not want to risk
being responsible of the semantics of their code being contaminated by a third party's tool.

To mitigate this risk, the \verb|FORVES|
\footnote{FORmaly Verified EVM optimizationS \url{https://github.com/costa-group/forves2/}}
project aims to develop a verifier which is able to guarantee that an optimization of a jump-free
sequence of EVM bytecode retains the same semantics as the original unoptimized version. This, however,
does not by itself suffice to appease the worries of the \emph{smart contract} developer, since we've
merely pushed the burden of trust from the optimizer to the verifier. If the verifier were to have a
bug, it could fail to recognize a valid optimization or, in the worst case, guarantee that an erroneous
optimization is in fact valid, impacting its users.

The solution is to, again, shift the burden of proof to another tool. For the \verb|FORVES| project the 
verifier has been written in Coq, a proof assistant and programming language which allows its users to 
define a formal specification and prove that their program adheres to it. The benefit of using a tool
such as Coq is that the burden of proof is placed on a small core subset of its features, which have
been deeply scrutinized by mathematicians, and out of which all the other functionalities are built.
Therefore, to trust that a Coq-certified program is correct is equivalent to trusting that the Coq  
kernel is correct, which is a much greater assurance. 
% TODO: add link to article on verifying Coq in Coq: https://www.openaccessgovernment.org/proof-assistants-2/80852/
Many different projects have taken advantage of this fact to develop pieces of critical software which
can be trusted upon \citep{ConCert}
% TODO: add references to other tools verified in Coq

The purpose of this project is to aid in the development of \verb|FORVES2|, a successor of \verb|FORVES|
which improves on its reasoning power by allowing it to take into account contextual information to
decide if two pieces of code are equivalent. 

The \verb|FORVES| verifier was devised to certify optimizations of the \verb|GASOL| superoptimizer. 
Superoptimization is a compilation technique that searches, for a given jump-free sequence of 
instructions, a semantically equivalent sequence of instructions which is optimal in some metric, like
memory usage or execution cost. Since the superoptimizer requires these sequences of instructions to not
have bifurcations, to optimize a program it first extracts all the sequences of instructions which don't 
perform jumps and optimizes those separately before reassembling the program. In doing so, it remembers
which conditions triggered those jumps so it can gain more information on which states of the program
are possible for each section of the program. Consider the following example

\begin{center}
\begin{tikzpicture}[auto,
  node distance = 12mm and 20mm,
  start chain = going below,
  box/.style = {draw,rounded corners,blur shadow,fill=white,
        on chain,align=center}]
 \node[box] (b2)    {\verb|x0|\mintinline{haskell}{ = ...} \\ \verb|x1|\mintinline{haskell}{ = ...} \\ \verb|x0| - \verb|x1| $\ge$ 128?};      
 \node[box] (b3)    {\mintinline{haskell}{MEM[x0] = ...} \\ \mintinline{haskell}{MEM[x1] = ...}};  
 \node[box,right=of b2] (b4)    {\dots};     
 \begin{scope}[rounded corners,-latex]
 \path 
  (b2) edge node{\color{red}{\texttt{false}}} (b4)
  (b2) edge node{\mintinline{coq}{true}} (b3);
 \end{scope}
\end{tikzpicture}
\end{center}

The third block is only reachable if the jump condition is met, which means that \verb|x0| $\ge$ 
\verb|x1| $+ 128$. This information can be useful when optimizing a piece of code, and must therefore
be taken into consideration when certifying that two sequences of instructions are semantically 
equivalent. 


Consider the following two programs.

\begin{listing}[!ht]
\begin{minted}{haskell}
SWAP3 SWAP1 SWAP2 MSTORE SWAP1 SWAP2 ADD SWAP1 MSTORE
\end{minted}
\caption{Original EVM program}
\label{lst:evm-original}
\end{listing}

\begin{listing}[!ht]
\begin{minted}{haskell}
MSTORE MSTORE POP 
\end{minted}
\caption{Optimized EVM program}
\label{lst:evm-optimized}
\end{listing}

Even though we refer to them as ``Original'' and ``Optimized'', at first glance these two programs don't
appear to be equivalent. Consider the execution of the first program (\ref{lst:evm-original}).

\begin{minted}{haskell}
[x0,x1,x2,x3,x4,x5]
SWAP3    -- Exchange 0th with 3rd element
[x3,x1,x2,x0,x4,x5]
SWAP1    -- Exchange 0th with 1st element
[x1,x3,x2,x0,x4,x5]
SWAP2    -- Exchange 0th with 2nd element
[x2,x3,x1,x0,x4,x5]
MSTORE   -- Store 1st element at 0th element's offset
[x1,x0,x4,x5] -> MEM[x2] = x3
SWAP1
[x0,x1,x4,x5]
SWAP2
[x4,x1,x0,x5]
ADD      -- Exchange first 2 elements with their sum
[x4+x1,x0,x5]
SWAP1
[x0,x4+x1,x5]
MSTORE
[x5] -> MEM[x0] = x4+x1
\end{minted}

Finally, consider the execution of the second program (\ref{lst:evm-optimized})

\begin{minted}{haskell}
[x0,x1,x2,x3,x4,x5]
MSTORE
[x2,x3,x4,x5] -> MEM[x0] = x1
MSTORE
[x4,x5] -> MEM[x2] = x3
POP     -- Discard 0th element
[x5] 
\end{minted}

The effects that these two programs produce are very different. However, they become equivalent
if we have the following pieces of context:

\begin{itemize}
    \item \verb|x4| $=$ \verb|x5|
    \item \verb|x5| $=$ 0
    \item \verb|x0| $\ge$ \verb|x2| $+ 128$
\end{itemize}

The first two ensure that \verb|x4| $+$ \verb|x1| is in fact equivalent to \verb|x1| since from them we
can deduce that \verb|x4| $= 0$. The last one guarantees that the memory accesses to the offsets
referenced by \verb|x0| and \verb|x2| are disjoint, since the word size of the EVM is of 32 bytes (128 
bits), and therefore we can reorder both writes while preserving the semantics of the code. These two
snippets which at first seemed different can be shown to be equivalent if some sort of conditions always
apply.

The purpose of this project is to aid in the development of \verb|FORVES2|, a successor of 
\verb|FORVES| which is able to take into account this kind of contextual information to verify whether
two sequences of instructions are equivalent. To do so we need to be able to reason if a set of
constraints are met from the context, which is what this project sets out to do, by developing a 
certified ``implication checker'' which tests whether some constraints can be derived from the current
context.

\section{Objectives}
\label{sect:objectives}

\begin{itemize}
    \item Objective 1
    \item Objective 2
    \item \emph{and most importantly...} Objective 3!
\end{itemize}

\section{Work plan}
\label{sect:work-plan}

In order to achieve the previous objectives, the following plan was developed.

\begin{itemize}
    \item Define what contextual information is useful for the verifier
    \item Define the specification of an implementation checker
    \item Develop and certify a basic implementation checker
    \item Develop and certify a more complex implementation checker.
\end{itemize}