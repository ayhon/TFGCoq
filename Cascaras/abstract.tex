\chapter*{Abstract}

\section*{\tituloPortadaEngVal}

The optimization of Ethereum Virtual Machine (EVM) bytecode presents unique challenges and opportunities due to its 
distinct requirements, including considerations such as execution cost and compiled binary size. Despite its potential 
benefits, optimization in this realm remains limited, primarily due to the high stakes associated with EVM programs, or 
smart contracts, where even minor errors can result in significant financial losses. To address these concerns, the 
\verb|FORVES| project introduces a verifier, leveraging formal verification techniques to ensure optimized bytecode retains 
its original semantics. By employing Coq, a rigorously verified proof assistant and programming language, \verb|FORVES| 
enhances trust in optimization processes, shifting the burden of proof from the optimizer to the verifier. This project 
outlines the development of \verb|FORVES2|, an advanced iteration, which incorporates contextual information to assess code 
equivalence. Specifically, the project aims to devise an implication checker capable of verifying whether certain 
octagonal constraints hold in specific contexts, by employing a tightened transitive closure algorithm for validation.

\section*{Keywords}

\noindent Coq, Octagonal restrictions, Functional programming, Smart contracts, Ethereum, EVM, Blockchain
% \noindent 10 keywords max., separated by commas.



