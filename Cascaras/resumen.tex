\chapter*{Resumen}

\section*{\tituloPortadaVal}


La optimización del \emph{bytecode} de la máquina virtual de Ethereum (EVM) presenta retos y
oportunidades únicos debido a sus requisitos específicos, que incluyen consideraciones como 
el coste de ejecución y el tamaño del ejecutable compilado. A pesar de sus potenciales 
beneficios, la optimización en este ámbito sigue siendo limitada, principalmente debido a los 
altos riesgos asociados con los programas EVM, llamados contratos inteligentes, donde incluso
errores menores pueden resultar en pérdidas financieras significativas. Para abordar estas 
preocupaciones, el proyecto FORVES desarrolla un verificador, aprovechando las técnicas de 
verificación formal para garantizar que el código de bytes optimizado conserve su semántica
original. Al emplear Coq, un asistente de pruebas y lenguaje de programación rigurosamente 
verificado, FORVES aumenta la confianza en los procesos de optimización, trasladando la carga 
de la prueba del optimizador al verificador. Este trabajo ocurre en el entorno de \verb|FORVES2|,
un sucesor del proyecto anterior que incorpora información contextual para evaluar la equivalencia
del código. En concreto, el proyecto pretende diseñar un verificador de implicaciones capaz de 
comprobar si ciertas restricciones octogonales se cumple en contextos específicos, empleando para
su validación un algoritmo de cálculo del cierre transitivo.

\section*{Palabras clave}

\noindent Coq, Restricciones octogonales, Programación funcional, Smart contracts, Ethereum, EVM, Blockchain